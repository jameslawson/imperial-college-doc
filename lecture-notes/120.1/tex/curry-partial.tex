
\chapter{Currying and Partial Application}





\section{Point-free Style}

It is very common for functional programmers to write functions as a composition of other functions, never mentioning the actual arguments they will be applied to. For example, compare:

\begin{lstlisting}
sum = foldr (+) 0
sum' xs = foldr (+) 0 xs
\end{lstlisting}

These functions perform the same operation, 
however, the former is more compact, and is considered cleaner.
It is clearer to write let fn = f . g . h than to write let fn x = f (g (h x)).
This idea of knocking off arguments is called \textit{point-free style}. 

In fact, this can be compared to pipelining in unix shell scripting

\begin{lstlisting}
grep '^X-Spam-Level' | sort | unique | wc -l
length . nub . sort . filter (isPrefixOf "X-Spam-Level")
\end{lstlisting}