
\chapter{Division}




\section{Introducing Division in Rings}

What do we know about division from group theory?
When we add addition and subtraction into the mix, things become a bit more confusing. 
Division in rings is centred around the equation $ra = rb$. In particular, we want to know 
when we can cancel $r$ and when we cannot. 

It turns out if we have $r$, then it falls into certain exactly \textit{three} disjoint cases. 
In two cases, we can cancel $r$ but in one case we cannot. The rest of our work in this chapter 
will be looking at these three cases and finding 

\begin{itemize}	
\renewcommand{\labelitemi}{$\Box$}
\item \textbf{Invertible.} In the first case, $r$ is invertible. Right away, we can see that we can 
cancel $r$ in $ra = rb$ to get $a = b$. This is obvious because we can left-multiply 
both sides of $ra = rb$ by $r^{-1}$ to get $a =b$. This first case we have seen from group theory. We didn't 
need any addition/subtraction to see that $a$ and $b$ must be equal.
\item \textbf{Non-invertible Cancellable.} In the second case, $r$ does not have a multiplicative inverse 
but can still be cancelled. This case is not as obvious as the first and we will look at it a bit more 
later.
\item \textbf{Non-cancellable.} In the third case, is the bad case where we cannot cancel $r$ from $ra = rb$. 
We would like to find certain properties that tell us when we are in this case. 
\end{itemize}




\begin{figure}[h]
\begin{tikzpicture} [
	every node/.style={node_sty},
	edge from parent/.style={edge_sty},
	edge from parent path={(\tikzparentnode.south) -- ++(0, -0.2cm) -| (\tikzchildnode.north)},
	level/.style = {level distance = 0.6cm},
	level 1/.style={sibling distance=6cm},
	level 2/.style={sibling distance=3cm}
]
	\node (root) {$ra = rb$}
    child{
    	node (can) {$r$ invertible}
		child {node(inv) [text width=6em] {$r$ cancellable}}
	}
	child{
		node (ncan) {$r$ non-invertible}
		child {node (nz) [text width=6em] {$r$ cancellable}}
		child {node (z) [text width=8em] {$r$ non-cancellable}}
	};
	
\end{tikzpicture}
\end{figure}

\frmrule


\begin{example}
Here is an example of case 1. In $(\mathbb{R}, +, \times)$ we have the equation $5x = 5y$.
We can simply multiply both sides by the multiplicative inverse of 5 which is $\frac{1}{5}$, this gives us $x=y$. 
\end{example}

\frmrule

\begin{example}
Here is an example of case 2. Suppose in $(\mathbb{Z}, +, \times)$ we have the equation $2x = 2y$. This equation tells us that 
doubling the integer $x$ is the same as doubling the integer $y$. Intuitively we can tell that 
$x = y$. In this ring, it wouldn't make sense to double two \textit{different} integers and get the \textit{same result}. 
$2$ is not invertible in $(\mathbb{Z}, + \times)$ yet we know we should be able to cancel it from $2x = 2y$ to get $x = y$.
\end{example}

\frmrule

So cancelling things is not as simple as multiplying by inverse elements. The previous example 
demonstrated that we can cancel elements that are not invertible. This is part of the reason why 
division in rings is more complex than division in groups. 

\begin{example}
Case 3 is the most interesting. Rearranging gives us $r(a-b) = 0_R$. The next step
you would \textit{expect} would be to say that either $r = 0$, or $a-b = 0$. 
This step is called \textit{the zero-factor property}.
However, in general, doing this step here is wrong. It turns out 
that \textit{sometimes} $r(a-b) = 0_R$ without either $a$ or $a-b$ being zero at all! 
For example look at $(\mathbb{Z}_6, +_6, \times_{6})$.
We have $3 \times (4-2) = 0$. Yet $r = 3$ is not zero and $4-2$ is not zero. 
So going from $3 \times (4-2) = 0$ to saying $r = 0$ or $a - b = 0$ is wrong. 
\end{example}


\frmrule

The zero factor property is $r(a-b) = 0_R$ implies $r = 0_R$ or $a-b = 0_R$
This can be rewritten in logic as not($r(a-b) = 0_R$ and not ($r = 0_R$ or $a-b = 0_R$))

\highlightdef{$r$ can be cancelled iff the \textit{zero factor property} holds}

Now in the non-cancellable case, if we assume the zero-factor property, we will get a contradition.
And so we will see that zero-factor property is not true. 
Hence  $r(a-b) = 0_R$ and not ($r = 0_R$ or $a-b = 0_R$)
in other words $r(a-b) = 0_R$ and $r \neq 0_R$ and $a-b \neq 0_R$.

This is the condition that tells us that we cannot cancel $r$.

\highlightdef{\textbf{Non-cancellable Property}:  $r(a-b) = 0_R$ and $r \neq 0_R$ and $a-b \neq 0_R$}
The non-cancellable property tells us when we cannot cancel $r$ from $r(a-b) = 0_R$ when we 
it is because $r$ and $a-b$ multiply to give zero (and neither of them are 0). 

\frmrule

\begin{example}
If a ring has no multiplicative identity, then there is no notion
of a unit. How does this affect our three cases?
\end{example} 




\section{Units and Zero Divisors}




\highlightdef{$r \neq 0_R$ is a \textbf{Zero-divisor} iff $\exists a \exists b$ st $a-b \neq 0_R$ and $r(a-b) = 0_R$}
In other words, $r$ is called a \textit{zero-divisor} is there is at least one case 
where it cannot be cancelled in $r(a-b) = 0_R$. This is when $r$ times $a-b$ gives zero but neither $r$ nor 
$a-b \neq 0_R$ is zero. Suppose we let $s = a - b$, then we can rewrite the definition as follows: 

\highlightdef{$r \neq 0_R$ is a \textbf{Zero-divisor} iff $\exists s$ st $s \neq 0_R$ and $rs = 0_R$}

This form is equivalent but a simpler to work with for later proofs. However it no longer captures the 
idea that we are trying to cancel $r$ from the equation $ra = rb$.

\frmrule

\begin{example}
Show that 4 is a zero-divisor in 
\end{example}


\frmrule

Zero divisors and units are the key to understanding cancellation. It turns out that
the three cases of cancellation vs non-cancellation of $r$ correspond \textit{exactly} with whether 
$r$ is (i) just a zero-divisor, (ii) just a unit or (iii) neither. This is shown below.
From now on, we only concern ourselves with $r$ that is neither zero nor one. We will treat 
these two special cases at the end. 

\begin{figure}[h]
\begin{tikzpicture} [
	edge from parent/.style={edge_sty},
	edge from parent path={(\tikzparentnode.south) -- ++(0, -0.2cm) -| (\tikzchildnode.north)},
	level/.style = {level distance = 0.6cm, every node/.style={draw, node_sty}},
	level 1/.style={sibling distance=6cm},
	level 2/.style={sibling distance=3cm}
]
	\node [draw, node_sty] (root) {$ra = rb$}
    child{
    	node (inv) {$r$ invertible}
		child {node(can1) [text width=6em] {$r$ cancellable}}
	}
	child{
		node (ninv) {$r$ non-invertible}
		child {node (can2) [text width=6em] {$r$ cancellable}}
		child {node (ncan) [text width=8em] {$r$ non-cancellable}}
	};
	\node[below=0.2cm of can1, align=center] {\textit{$r$ is a unit}\\\textit{$r$ is not a zero divisor}};
	\node[below=0.2cm of can2, align=center] {\textit{$r$ is not a unit}\\\textit{$r$ is not a zero divisor}};
	\node[below=0.2cm of ncan, align=center] {\textit{$r$ is not a unit}\\\textit{$r$ is a zero divisor}};
\end{tikzpicture}
\end{figure}

Before we prove anything this claim, what is intersting is that \textit{there is no fourth case} 
corresponding to when $r$ is \textit{both} a unit and a zero-divisor. In fact, we 
can prove that such an element cannot possible exist in a ring. That is, 
for any element $r$, we have that:

\highlightdef{$r$ cannot be \textit{both} a unit \textit{and} a zero-divisor}

In other words being a zero divisor of some kind and a unit at the same time isn't possible. 
These events are mutually exclusive. You either one, or the other, or neither. 
Before looking at the actual three cases, we shall prove that we don't
need to consider this fourth case at all (because it doesn't make sense).

\frmrule

\textbf{Proof:}\\
Assume that $r$ is both a \textit{unit} and a \textit{zero-divisor}. 
By definition of zero-divisor we have $\exists a \exists b$ st $a-b \neq 0_R$ and $r(a-b) = 0_R$. 
But we assumed that $r$ is a unit and so $r^{-1}$ exists. Using this, we can multiply $r(a-b) = 0_R$ on 
both sides to get $a-b = 0_R$. This contradicts our assumption that $a-b \neq 0_R$. 
Hence $r$ cannot be both a \textit{unit} and a \textit{zero-divisor}. $\square$

\frmrule

So being invertible and being a zero-divisor are two \textit{disjoint} cases.
Having one less case does indeed makes things simpler, but we still have three cases 
to prove. We need to show that the three cases of: (i) \textit{$r$ is just a unit}, 
(ii) \textit{$r$ is neither a unit nor a zero-divisor}, (iii) \textit{$r$ is just a zero-divisor}, 
correspond precisely to where they have been put on the diagram in regard to cancellation.  

\begin{itemize}
\renewcommand{\labelitemi}{$\Box$}
\item \textbf{Case 1}, \textit{$r$ is a unit} $\wedge$ \textit{$r$ is not a zero-divisor} $\rightarrow$ \textit{$r$ cancellable}  
\item \textbf{Case 2}, \textit{$r$ is not a unit} $\wedge$ \textit{$r$ is not a zero-divisor} $\rightarrow$ \textit{$r$ cancellable}   
\item \textbf{Case 3}, \textit{$r$ is not a unit} $\wedge$ \textit{$r$ is a zero-divisor}$\rightarrow$ \textit{$r$ not cancellable}   
\end{itemize}

\frmrule


Recall that $(x \boxtimes)$ is using Haskell's \textit{section notation} and denotes 
the function $f(y) = x \boxtimes y$. Similarly, $(\boxtimes x)$ denotes $f(y) = y \boxtimes x$. To check 
that an element is a left zero-divisor, 


\frmrule


\section{Non-commutative Ring Division}

\begin{example}
Support xyz = 0. Explain why x is a right zero divisor.
\end{example}

Notice how we didn't need to add brackets to xyz. Multiplication in rings is always associative.
So x(yz) is the same as (xy)z. 

\frmrule

\begin{example}
Give a formal definition of a \textit{left proper zero divisor}. 
\end{example}

\frmrule

If our multication isn't commutative then we need to be careful as to whether we use $f_x$ or $g_x$. 
However if multication is commutative, then every left zd is also going to be a right zd.
And every right zd will also be a left zd. So saying left/right is redundant.
So, for zds in \textit{commutative rings}, we say that 
they are simply that $r$ is a \textit{a zero divisor}.  


\frmrule


\highlightdef{$r$ cannot be both a \textit{left-unit} and a \textit{left zero-divisor}}

Assume $r$ is a \textit{left-unit} and a \textit{left zero-divisor}.
By definition of a left zero-divisor we have $\exists a \exists b$ st $a-b \neq 0_R$ and $r(a-b) = 0_R$. 
But $r$ is also a left-unit so has a \textit{left-inverse} $r_L$ exists. 
Using this, we can multiply $r(a-b) = 0_R$ on the left to give $(a-b) = 0_R$
This contradicts our assumption that $a-b \neq 0_R$. 
Hence $r$ cannot be both a \textit{unit} and a \textit{zero-divisor}.

\highlightdef{$r$ cannot be both a \textit{right-unit} and a \textit{right zero-divisor}}

This intuitively follows by symmetry. This can be proved is the same way the left case was proved.





%\highlightdef{\textbf{$\mathbb{Z}_d$ unit condition}: x is a unit iff $\text{hcf}(x,d) = 1$ \\
%\textbf{$\mathbb{Z}_d$ zd condition}: x is a zd iff $\text{hcf}(x,d) \neq 1$  }




\section{Fields and Domains}

Take any nontrivial element $r$,
\hightlightdef{A \textit{Domain} is a ring where every $r$ is not a zero-divisors.}
\hightlightdef{A \textit{Field} is a ring where every $r$ is not a zero-divisors and every $r$ is a unit.}

\textit{Fields} and \textit{Domains} both satisfy the \textit{zero-factor property}. 
That is, when we have $r(a-b) = 0$, we can factor to give: $r = 0$ \textit{or} $a - b = 0$. 
And this works for \textit{all} elements $r \in R$. In other words, 
fields and domains are great because we can \textit{always cancel elements}. 
The difference between fields and domains is that fields are slightly better because 
every element is invertable as well.

One way to think about domains and fields is to take any arbitrary nontrivial element, $r$ 
of the ring and say what cancellation case $r$ can fall into.  
From this thinking we have that $r$ can fall into:
\highlightdef{
\textbf{Domain}: case 1 or case 2\\
\textbf{Field}: definitely case 1
}
Notice that Domains and Fields have no zero-divisors at all. 
So it is \textit{impossible} for $r$ to fall into case 3. 

Quite intuitively, 
\highlightdef{
Every field is a domain, but not every domain is a field.
}

An domain cuts out one area. A field cuts out two areas.
In a domain, you won't get any zd's but you may get some non-units.


For historical reasons, commutative Domains are better known as \textit{Integral Domains}. 
Why Integral? This is likely because the ring of integers is an domain. 
When we multiply numbers together,we don't get zero - unless we are multiplying by zero. 
So saying integral is a reminder that domains are a bit like $\mathbb{Z}$








% $F = F^{*} \cup {0}$
% $F = F^{*} \cup F_{\text{pnu}} \cup  \left\{0\right\}$

\frmrule

\begin{example}
Find what's wrong with the following diagram for a field.
\end{example}

\frmrule


\frameans{You may assume $R \neq {0}$}{Zero is inside zero-divisors.}


