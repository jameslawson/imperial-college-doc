
\chapter{Bonding}




\section{Introducing Chemical Bonding}


\highlightdef{\textbf{Chemical Bond} Atoms }

\highlightdef{\textbf{Molecule}: group of \textit{two or more atoms} stuck together by \textit{chemical bonds} }

\highlightdef{\textbf{Compound}: molecule that contains \textit{at least two different elements} }
All compounds are molecules but not all molecules are compounds.

\frmrule

\begin{example}
Hydrogen is the Chemical Element with $Z = 1$, with symbol H

Two Hydrogen atoms join together by a chemical bond to form $H_2$. \\
$H_2$ is indeed a \textit{molecule} because is has two atoms stuck together by a chemical bond. \\
$H_2$ is a not a \textit{compound} because the two atoms that stuck together are \textit{the same}. 
They are both Hydrogen.
\end{example}

\frmrule

\begin{example}
Hydrogen is the Chemical Element with $Z = 1$, with symbol H\\
Oxygen is the Chemical Element with $Z = 8$, with symbol O

Two hydrogen atoms and a oxygen atom all become stuck together form $H_2O$. \\
$H_2O$ is indeed a \textit{molecule} because is has two atoms stuck together by a chemical bond. \\
$H_2O$ is indeed a \textit{compound} because there are two different elements involved in the chemical bond. 
The molecule has hydrogen molecules \textit{and} has oxygen. 
\end{example}

\frmrule

\begin{example}
A molecule of pentange can be represented as shown: \\
(a) What do the \textit{letters} on the diagram represent? \\
(b) What do the \textit{lines} on the diagram represent? \\
(c) Explain whether or not this molecule is a \textit{compound}.
\end{example}

\frmrule


\highlightdef{\textbf{Ionic Bonding}: atoms joining together by \textit{transferring electrons}}

\highlightdef{\textbf{Covalent Bonding}: atoms joining together by \textit{sharing electrons}}






\section{Ionic Bonding I}


\highlightdef{\textbf{Ion}: atom or molecule in which the number of electrons $\neq Z$}


\highlightdef{\textbf{Negative Ion}: atom or molecule in which the number of electrons $> Z$}

\highlightdef{\textbf{Positive Ion}: atom or molecule in which the number of electrons $< Z$}


Negative ions are often called \textit{anion} (\textit{an-ion}). 
(an comes from the greek word for down, \textit{ano}. 
up because the electron count is above the normal value of $Z$).

Positive ions are often called \textit{cation} (\textit{cat-ion}). 
(cat comes from the greek word for down, \textit{kata}. 
down because the electron count is below the normal value of $Z$). 


\highlightdef{Postive and Negative Ions attract to each other}

\begin{example}
Explain whether the two ions shown are attracted to each other.
\end{example}


\begin{example}
Show that the two ions shown repel each other.
\end{example}


\highlightdef{\textbf{Ionic Bonding}: Ions attracted to each other}


\section{Ionic Bonding II}


\highlightdef{\textbf{Dot and Cross Diagram}: diagram showing how electrons are transferred 
in ionic bonding between atoms of two different elements}

\frmrule

\begin{example}
Magnesium is a Chemical element with $Z = 12$ and symbol Mg. \\
Oxygen is a Chemical element with $Z = 8$ and symbol O. 

Using a dot-and-cross diagram, 
show an ionic bond between an atom of Magnesium and an atom of Oxygen 
where \textit{two} electrons are transferred from the Magnesium atom to 
the Oxygen atom.
\end{example}


\frmrule

\begin{example}
Sodium is a Chemical element with $Z = 11$ and symbol Na. \\
Chlorine is a Chemical element with $Z = 17$ and symbol Cl. 

Using a dot-and-cross diagram, 
show an ionic bond between an atom of Sodium and an atom of Chlorine 
where \textit{one} electron is transferred from the Sodium atom to 
the Chlorine atom.
\end{example}

\frmrule

\begin{example}
Draw dot-and-cross diagram to
show an ionic bond between an atom of Lithium and Chlorine.
\end{example}



\section{Ionic Bonding III}



\highlightdef{\textbf{Ionic Compound} a \textit{compound} in which the atoms are joined by \textit{ionic bonding}}


\frmrule

\begin{example}
Draw the dot-and-cross diagram for the ionic compound: 
\end{example}

\frmrule

\begin{example}
The dot-and-cross diagram below
shows an ionic bond between an atom of ..... and an atom of ....... 
where ..... electron(s) is transferred from the ...... atom to 
the ...... atom. 
\end{example}

\section{Covalent Bonding I}



\section{Covalent Bonding II}


\section{Covalent Bonding III}



\section{Relative Atomic Mass of Molecules}





