
\chapter{Atoms and Elements}




\section{The Atom}



\highlightdef{$A = \text{mass number} = \text{protons+neutrons}$}

$Z$ is called the 
\textit{atomic number} or sometimes the \textit{proton number}. 


$N$ is called the \textit{neutron number}. This, as its name 
suggests, denotes the \textit{number of neutrons}.


$A$ is called the \textit{mass number}. 
$A$ denotes the total number of protons \textit{and} the total 
number of neutrons. It is called the \textit{mass} number
because it is the protons and neutrons that give an atom its mass. Although 
electrons do have mass, they are \textit{extremely light} (that is, 
they have a very small mass). 
Notice that $A = N + Z$. 

\frmrule

$Z =$ Zahl (German word for number, the no. of \textit{protons} is the most important number) \\
$N =$ Neutron ($N$ counts \textit{neutrons}) \\
$A =$ Almost All (protons and neutrons made up \textit{almost all} of the mass of an atom) \\


\frmrule

\begin{example}
Fill in the following blanks. \\
The number of protons in an atom is ........ ($Z/N/A$) \\
The number of neutrons in an atom is ........ ($Z/N/A$) \\
The number of electrons in an atom is ........ ($Z/N/A$) \\
The number of protons, electrons and neutrons in an atom is ........ ($Z/N/A$) \\
The number of electrons and neutrons in an atom is ........ ($Z/N/A$) \\
The number of electrons the nucleus is .......... ($Z/N/A/$zero)
\end{example}

\frmrule

\begin{example}
Do we need a letter like $E$ to count the number of electrons in an atom, 
or is there another letter that already exists the does the same job?
\end{example}

\frmrule

\begin{example}
Fill in the following blanks. \\
We can see electrons under a microscope (true/false) \\
We can see protons under a microscope (true/false) \\
We can see neutrons under a microscope (true/false) \\
We can see the nucleus under a microscope (true/false) \\
We can see the atom under a microscope (true/false) 
\end{example}

\frmrule



\section{$Z-N$ Diagram}

How many protons, electrons and neutrons can an atom have? 
Scientists have seen, to this date, that atoms can have around 
1-100 protons, 1-100 electrons, 1-100 neutrons. Although the 
number of protons, electrons and neutrons can lie in these 
ranges, not every combination has been found. For example,
the combination of 2 protons, 2 electrons and 50 neutrons 
has not been discovered and may not ever exist. 


We can plot all combinations that have been found 
on a diagram. This is called the $Z-N$ diagram. 
The $Z-N$ plots all possible combinations and tells us 
which combations are actual atoms that can exist.
The $Z-N$ diagram gives is a helpful visual aid for 
looking at, just exactly what types of atoms there are 
in the universe (that we know of). Below shows the $Z-N$ diagram. 

The $N-Z$ diagram is a grid with two axis. Notice that some boxes filled in and 
some are not filled-in.

\begin{itemize}
\item The $y$-axis plots $Z$, the proton number. 
\item The $x$-axis plots $N$, the neutron number. 
\item If a coordinate is filled-in, it means that we are sure that 
that particular combination \textit{does indeed exist}. 
\item If a coordinate is not filled-in, 
it means that either the combination of protons and neutrons
\textit{doesn't exist}, or
that the combination has \textit{not yet been discovered}.
\end{itemize}

\frmrule

\begin{example}
What are we not plotting the number of electrons?

Remember, the number of electrons is \textit{always equal} to 
the number of protons, so we don't need to plot electrons, 
we can just look at the $y$-axis, i.e. the number of protons. 
\end{example}

\frmrule

\begin{example}
What are we not plotting the mass number, $A$?

Remember, $A = Z + N$. So we can look at an element 
at $(x,y)$ and find the mass number from $x + y$. 
\end{example}

The $Z-N$ diagram is useful for us 
It tells us what atoms are known to exist. Just what 

\highlightdef{The $Z-N$ diagram tells us what atoms exist}


\frmrule

\begin{example}
Look at the $Z-N$ diagram. \\
Can we have an atom with 0 neutrons, 1 proton?
\end{example}

\frmrule

\begin{example}
Look at the $Z-N$ diagram. \\
(a) Can we have an atom with 3 neutrons, 3 protons? \\
(b) Can we have an atom with 1 neutrons, 1 proton? \\
\end{example}



\section{Elements}

We will not study of protons, neutrons and electrons. 
This falls under the subject of \textit{particle physics} that is studied 
in physics. Instead, we study \textit{elements}. 

All substances are made up of atoms. Quite often, we get a variety of 
atoms. However some substances are special. 
For some subtances, \textit{all} of their atoms that have the same 
number of electrons. These substances, or groups of identical-$Z$ atoms
are called \textit{elements}. 

\highlightdef{\textbf{Element}: Atoms with the same value of $Z$}

\begin{example}
Fill in the blanks with one choice. \\
$X$ and $Y$ have have the same value for ...... ($A/Z/N$) 
\end{example}

\frmrule 

\begin{example}
The atoms of a substance all have same no. of protons/neutrons/electrons.  (true/false) \\
\end{example}

The substance of an element has all of its atoms having
\highlightdef{atoms with the same $y$-coordinate on the $Z-N$ diagram}


\begin{example}
Why do we use symbols to tell the difference between elements?
\end{example}


\frmrule 

\highlightdef{same value of $Z =$ same element \\
$=$ same number of electrons $=$ same chemical properties}

\frmrule 


\highlightdef{\textbf{Isotopes}: Atoms with the same value of $Z$, but different value for $N$}
Isotopes all have the same value of $Z$ and so they are all the \textit{same element}, but
crucially they have a \textit{different number of neutrons}. 

\frmrule


\begin{example}

\end{example}

\frmrule

\highlightdef{\textbf{A-Z Notation}: $\nucl{A}{Z}{X}$. writing isotopes of element $X$ using \textit{hardly any space}: }

It gives us a succinct way to give us all the information we 
need about the atom. It gives us a short, yet complete way to 
write down an isotope. As well as giving us $A$ and $Z$, the notation 
also gives us corresponding element's symbol (so we don't have to take $Z$ and look it up).

\section{Periodic Table}


\highlightdef{\textbf{Periodic Table}: A table \textit{organising elements conveniently} for Chemistry}

\highlightdef{The period table only plots $Z$}
This is unlike the $Z$-$N$ diagram. The period table takes 
the $y$-axis of the $Z$-$N$ diagram and arranges them in a particularly interesting 
way that is convenient for chemistry. 

\highlightdef{\textbf{Blocks}: Large groups of elements }


\highlightdef{\textbf{Groups}: The \textit{columns} of the Periodic Table }


\highlightdef{\textbf{Periods}: The \textit{rows} of the Periodic Table }



\frmrule 

\begin{example}
Sodium and Chlorine are in the same period. ............. (True/False)
\end{example}



\section{Enery Shells}



\highlightdef{\textbf{Energy Shell}: electrons can put into groups called \textit{energy shells} }


\section{Mass of Atoms}


\begin{example}
Suppose we have 6 black balls that have mass $1.01kg$ and 6 white balls that have mass $1.02kg$. \\
(a) What is the average mass of the balls. \\
(b) Is the average far from $1.01kg$ and $1.02kg$. 
\end{example}

\frmrule


Suppose now that we have a similar situation, but now balls were made very small (same size as a particle). 
And also suppose that their weights differ by an incredibly minute amount. 
The average mass is still a good approximation for the size of a single ball. 

\frmrule


\highlightdef{ \textbf{Atomic Mass }: $m_{\text{a}}$ is approx. mass of a proton/neutron }
%  relative isotopic mass

The actual value chosen is $m_{\text{a}}$ found from the mass of carbon-12. That is, 
we take the mass of carbon-12, and, realising that the majority of its mass is from 
the 12 particles in its nucleus, we divide by 12 to get an approximate value for 
the mass of a proton/neutron. Remember that protons and neutrons have a slightly 
different mass. So this is indeed an approximation.



\frmrule

If somebody asks you: \textit{what is the mass of hydrogen?}, 
you should respond with: \textit{which hydrogen atom}?. 

\highlightdef{ Atoms of of the same element have different masses  }

For elements where there are a lot of different isotopes, we find that:
the mass for isotope 1 of $X$ can be is quite different to the mass for isotope 2 of $X$.


\frmrule

\begin{example}
Hydrogen is the Chemical Element with $Z = 1$, with symbol H \\
Hydrogen has three known isotopes: $N = 1$, $N = 2$, and $N = 3$. \\
\end{example}

For a given element $X$, we define:
\highlightdef{ \textbf{Relative atomic mass} of $X$: 
$A_r(X) = \frac{\text{average mass for an isotope of } X}{\text{average mass of a proton/neutron}}$}

The relative atomic mass, despite its name, gives us a picture of the average number of 
protons/neutrons for atoms of that element. The calculation takes 
into



$A_r(X) = \frac{\text{average mass for an isotope of } X}{\text{average mass of a proton/neutron}}$ \\
$= \frac{m_1 p_1 + m_2 p_2 + ... + m_n p_n }{m_a}$ \\
$= \frac{m_1}{m_a}p_1 + \frac{m_2}{m_a}p_2 +  ... + \frac{m_n}{m_a}p_n$ \\

\frmrule

\begin{example}
Boron is the Chemical Element with $Z = 5$, with symbol B \\
Boron has two known isotopes: $N = 5$, and $N = 6$. \\


\begin{table}[h] 
\centering
\begin{tabular}{crr} 
\hline
Isotope & Relative atomic mass & Abundance\\
\hline 
Boron-10 & 10.0 & 18.7\\ 
Boron-11 & 11.0 & 81.3\\ 
\hline 
\end{tabular} 
\end{table} 



In the table below, you are given the relative atomic masses of each isotope and their percentage abundance. \\
Find the relative atomic mass of Boron.

$A_r(B) = A_r(\nucl{10}{5}{B}) \times 0.187 + A_r(\nucl{11}{5}{B}) \times 0.813$\\
$= 10.0 \times 0.187 + 11.0 \times 0.813 = 10.8$

\frmrule


\end{example}