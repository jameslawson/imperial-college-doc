\chapter{Vim}


\section{Movement}


\frmrule 

\textit{Search to Move}

\highlightdef{\textbf{Search}: Press $/$ then regex search pattern}

\highlightdef{\textbf{Next}: Press $n$. \textbf{Back}: Press $N$}




\frmrule 

\textit{Using Relative Numbers}




\section{Maps}

Similar to how Unix has a \lstinline{.cshrc} file
(\textit{C} \textit{S}hell \textit{R}un \textit{C}ommands file), 
Vim has a \lstinline{.vimrc} file, a Vim \textit{R}un \textit{C}ommands file.
You will most likely need to create it - it is not usually there by default.


\begin{sidenote}{vimtutor}
 
\end{sidenote}




\begin{sidenote}{Paste to match indentation}
\lstinline{]p}. This pastes the contents of a buffer just like \lstinline{p} does, 
but it \textit{automatically adjusts the indent} to match the line the cursor is on. 
This is useful for moving code around.
\end{sidenote}


\section{Changing}

Single Character - r \\
Up to The End of Line - c\$ \\
Entire LIne - cc 
