\chapter{Find, Grep, Sed, Awk}



\section{Regular Expressions}

\frmrule 

\textit{Vanilla regular expressions}

Vanilla regular expressions
$E ::= (E)* | (E)? | (E|E|...|E)$

Quite often 
we include 

$E+$ matches $E$ repeated $r$ times where $1 \leqslant r$.\\
Note that $E+ \equiv E(E)*$.

\frmrule 

\textit{Syntactic Sugar: Counting Occurences}

We introduce syntactic sugar for counting occurences. 

$E\{n\}$ matches $E$ repeated $r$ times where $r = n$.\\
$E\{n,\}$ matches $E$ repeated $r$ times where $n \leqslant r$.\\
$E\{n,m\}$ matches $E$ repeated $r$ times where $n \leqslant r \leqslant m$.\\

Notice that:\\
$E\{n\} \equiv EEE...E$\\
$E\{n,\} \equiv EEE...E(E)*$\\
$E\{n,m\} \equiv EEE...E|EEE...EE|...|EEE...EEEEE$\\
This shows that these new operators for counting occurences can be expressed in terms 
of the vanilla regular expression 
operators. And so these new operators be thought of as \textit{syntactic sugar}. 
They help make things look shorter and more meaningful to read. 

We now define:
$E ::= (E)* | (E)+ | (E|E|...|E) | E\{n\} | E\{n,\} | E\{n,m\}$\\

\frmrule 

\textit{Dot or Period: sugar for matching any character exactly once}

$. \equiv (c_1 | ... | c_N)$ 
for each $c_i \in \Sigma$

\frmrule 

\textit{Syntactic sugar: character classes}

Suppose we have a set of alphabets $\mathcal{A} = \{[a_0,...a_k]\}$ 
where an alphabet is a ordered list of characters $[a_0,...a_k]$. Then 
we can define syntactic sugar for characters belonging to mixtures of alphabets. 
We call mixtures of alphabets \textit{character classes} and use $[C]$ to match 
against a class $C$. 

$E ::= (E)* | (E)+ | (E|E|...|E) | E\{n\} | E\{n,\} | E\{n,m\} | [C] | [\wedge C] $\\
$C ::= (A_1|A_2|...|A_t)+ $\\
$A_i ::= a_i\text{-}a_j | a_0 | ... | a_k $ for each alphabet $A_i \in \mathcal{A}$

Each character class description $[C]$ can be replaced by an alternation $[C] \equiv (c_1|...|c_s)$
where we have one alternative for each character $c_i$ in the class $C$. 
Similarly $[\wedge C]$ can be replaced with $[\wedge C] (d_1|...|d_s)$ where 
where we have one alternative for each character $d_i$ in $\Sigma - C$. 
So character classes are, like the occurence counting operators, \textit{syntactic sugar}.

\frmrule 

\textit{Syntactic sugar for certain character classes}


Digits: $\backslash  d \equiv [0-9]$  $\backslash  D \equiv [\wedge 0-9]$ \\
Alphanumerical:  $\backslash  w \equiv [A-Za-z0-9 \_]$  $\backslash  W \equiv [\wedge A-Za-z0-9\_]$ \\
Spaces:  $\backslash  s \equiv [\textsc{fnrtvo}]$  $\backslash  S \equiv [\textsc{fnrtvo}]$ 

A common pattern is to use lowercase to match positively (using $[C]$) 
and to use \textit{uppercase} to match \textit{negatively} (using $[\wedge C]$).
Also not that the underscore character is commonly included in the alphanumerical class.
Because the character classes were syntactic sugar for alternatives, $[C] \equiv (c_1|...|c_s)$, 
these character class shorthands are too, syntactic sugar for some alternation.


\frmrule 

\textit{Handling multiline input with $\wedge$ and $\$$}

\frmrule 

\textit{Parenthesized Substring Matches}

\frmrule 

\textit{Back references}


\highlightdef{\textbf{Forward Slash Delimeters}: A lot of programming languages use 
\textit{forward slashes} as delimeters for regular expressions: $/regex/$ }


\section{Sed}


\highlightdef{\textbf{Sed}: s$/$regex$/$replacement string$/$flags inputfile}