\chapter{Git Internals I}




\section{Introducing Git}


\frmrule

\textit{What does git stand for?}

\highlightdef{Git has \textit{no particular abbreviation}}

\frmrule

\textit{Git vs GIT vs git}

At the command line we must, by default, type \lstinline{git} (all lowercase), but 
otherwise, \textit{all three} ways to 
write Git (Git, GIT, git) are popular and acceptable.
In this book, we will stick to using Git (and not use GIT or git).

\frmrule 

\chapter{Hashing}
\section{Git Objects}


\frmrule 

\textit{Git objects}

In Git, an \textit{object} has particular concrete definition.

\highlightdef{\textbf{Object}: An \textit{object} is a hash of  }

A git object is either a \textit{blob object}, 
a \textit{tree object}, or a \textit{commit object}. 


\begin{figure}[h]
\begin{tikzpicture} [
  every node/.style={node_sty},
  edge from parent/.style={edge_sty},
  edge from parent path={(\tikzparentnode.south) -- ++(0, -0.2cm) -| (\tikzchildnode.north)},
  level/.style = {level distance = 0.6cm},
  level 1/.style={sibling distance=3.5cm, text width=1.5cm}
]
  \node (root) {Object}
    child{node (b) {Blob}}
    child{node (t) {Tree}}
    child{node (c) {Commit}};
\end{tikzpicture}
\end{figure}

Blob objects are often called \textit{blobs}. 
Likewise, tree objects are often called \text{trees} 


\frmrule 

\textit{Git as a content-addressable system}

\highlightdef{Git is a \textit{content-addressable system}}


Git hashes the \textit{contents} of objects. 
And so, when Git
places a file into a data structure called the \textit{object store}. 
It does this hashing based on the \textit{hash of the data} and not,
for example, on the hash of the filename. 

The files are not stored according to their filename, 
but rather by the hash of the data they contain. 
We call these files, \textit{blob objects}


and commit objects are often called \text{commits}. 

\begin{sidenote}{SHA1}
SHA1 (or SHA-1, with the dash between SHA and 1) is a \textit{cryptographic hash function}. 
\end{sidenote}





\frmrule 

\textit{Filepath of Objects - Overview}

Inside the \lstinline{.git} folder is a 
subfolder called \lstinline{objects}

\highlightdef{Each object's content address is 160 bits $=$ 40 hex digits }


\highlightdef{\textbf{Filepath}: Objects are stored in the path \lstinline{.git}$/$\lstinline{objects}$/h_0h_1/h_2,...,h_{39}$  }
where $h_0...h_{39}$ are the fourty hexidecimal digits 
that are the result of hashing the object's contents. 

Git uses the first 2 hex digits of the hash as a directory name, then the remaining 38 as a filename. 
This is so that we don't end up with too many files in a single directory. 
By inserting a / after the first two digits, we improve the efficiency for a lot of filesystems
that slow when many files in the same directory. 

The exact preimage of the hash is determined by whether 
the object is a blob object, tree object or commit object as the 
Depending on the type of object, we 
have a different structure of plaintext to hash.
The next examples demonstrate.  

\frmrule 

\frmrule 


\textit{Identical files have no effect on the Object Store}





\frmrule 

\textit{SHA1 hash collisions, will in practice, never happen }

The probability is so small. 

\begin{sidenote}{Birthday Paradox and Hash Collisions}

\end{sidenote}


\frmrule

\begin{example}
What is the maximum number of folders that can be inside 
\lstinline{.git/objects}?
\\
\textit{Note: by folders, we mean direct child folders.}\\
\textit{Hint: recall that we use hex digits $h_0h_1$ of 
hash $h_0...h_{39}$ of a git object's contents for the 
folder name. The remaining digits are used for the filename.}

There are 16 choices for $h_0$ and 16 choices 
for $h_1$. So by the multiplication principle 
of counting, we have 256 choices for $h_0h_1$. 
So there at most 256 possible folders. 
\end{example}

\section{Blobs}

\begin{example}
Show that a \textit{blob object} with contents \lstinline{sweet} has a file path of:\\
\lstinline{.git/objects/aa/823728ea7d592acc69b36875a482cdf3fd5c8d}

\begin{itemize}
\item
If we take the SHA1 hash of:
$$[blob,\textsc{sp},6,\textsc{nul}] + [sweet,\textsc{lf}]$$
then we find the hash is:
$$h = aa823728ea7d592acc69b36875a482cdf3fd5c8d$$


So the path is \lstinline{.git}$/$\lstinline{objects}$/h_0h_1/h_2,...,h_{39}$
which is equal to \lstinline{.git/objects/aa/823728ea7d592acc69b36875a482cdf3fd5c8d}.

Here, \textsc{sp},\textsc{nul} and \textsc{lf} denote 
the space, zero-byte and linefeed characters respectively. We 
can find the hash using for example,\\ \lstinline{printf "blob 6\000sweet\n" | sha1sum}
\end{itemize}

\end{example}



\begin{sidenote}{The blob constant}
The constant $\textsc{blob}$ $=$ 100644 is actually a Unix file permission flag.\\
Show that this flag corresponds to owner: read/write, group+other read.

100644 = 0b11000100100100100\\
So splitting gives: 011|000|100|100|100|100. \\
\end{sidenote}

\section{Tree}

\frmrule 

\begin{example}
Show that a \textit{tree object} with a tuple:
$(rose,\textsc{normal\textunderscore file},h_0)$
has path:\\
\lstinline{.git/objects/05/b217bb859794d08bb9e4f7f04cbda4b207fbe9}\\
where $h_0 = aa823728ea7d592acc69b36875a482cdf3fd5c8d$.\\
You may assume $\textsc{blob}$ is a constant equal to 100644.

\begin{itemize}
\item
If we take the SHA1 hash of:
$$[tree,\textsc{sp},32,\textsc{nul}] + 
[\textsc{blob},myfile,\textsc{nul},h_0]$$
then we get:
$$h = aa823728ea7d592acc69b36875a482cdf3fd5c8d$$


So the path is \lstinline{.git}$/$\lstinline{objects}$/h_0h_1/h_2,...,h_{39}$
which is equal to \lstinline{.git/objects/aa/823728ea7d592acc69b36875a482cdf3fd5c8d}.

\end{itemize}
\end{example}


\section{Commits}




\frmrule 

% "commit 158" NUL
% "tree 05b217bb859794d08bb9e4f7f04cbda4b207fbe9" LF
% "author Alice <alice@example.com> 1234567890 -0800" LF
% "committer Bob <bob@example.com> 1234567890 -0800" LF
% LF
% "Shakespeare" LF
This is the first commit, so there are no parent commits, 
but later commits will always contain at least one line identifying a parent commit.
This is-a line, with special/separated/words (and some more).

\frmrule 


\begin{sidenote}{DEFLATE}
Objects in git are 
compressed using $\textsc{deflate}$. The $\textsc{deflate}$ algorithm is a 
variation of the popular compression algorithm LZ77 
(The \textit{Lempel-Ziv 1977 Algorithm}).  $\textsc{deflate}$ is also used 
for compression in $\textsc{png}$ images.
\end{sidenote}

\section{Tree Hierarchies}

\frmrule 

\textit{Tree objects as a list of tuples}

The previous example was for a tree object with \textit{one} tuple.
But, in general, tree objects have many tuples. They can be seen a \textit{list} of tuples.

\highlightdef{\textbf{Tree Object}: a \textit{tree object} is a list of tuples: $[($filetype, filename, hash$)]$}

The filetype of the tuple can vary. But we consider two cases:
\begin{itemize}
\item
$(n,\textsc{blob},h_0)$  where $n$ 
is the name of the file, $h_0$ is the hash of the blob object.
\item
$(n,\textsc{tree},h_0)$ where $n$ 
is the name of the directory, $h_0$ is the hash of the tree object.
\end{itemize}

In both cases $(n,t,h_0) \rightarrow [tree,\textsc{sp},32,\textsc{nul}] + [t,n,\textsc{nul},h_0]$
where $t$ is the filetype ($t \in \{\textsc{normal\textunderscore file}$, $\textsc{directory}\}$). 
Using these two types of tuple, we can build up a \textit{hierarchy} of tree objects. 

One tuples: $[tree,\textsc{sp},s,\textsc{nul}] + [t,n,\textsc{nul},h_0]$\\
Two tuples: $[tree,\textsc{sp},s,\textsc{nul}] + [t_0,n_0,\textsc{nul},h_0] + [t_1,n_1,\textsc{nul},h_1]$\\
Three tuples: $[tree,\textsc{sp},s,\textsc{nul}] + [t_0,n_0,\textsc{nul},h_0] + [t_1,n_1,\textsc{nul},h_1] + [t_2,n_2,\textsc{nul},h_2]$\\
$n$ tuples: $[tree,\textsc{sp},s,\textsc{nul}] +  [t_0,n_0,\textsc{nul},h_0] + ... + [t_{n-1},n_{n-1},\textsc{nul},h_{n-1}]$

That is, tree, with a space, followed by null, 
then followed by $n$ lots of $[t,n,\textsc{nul},h]$, 
one for each tuple where $n \geqslant 0$. 

\frmrule 

\begin{example}
Show that a \textit{tree object} with a tuples:\\
$[(house,\textsc{tree},h_0),(car,\textsc{blob},h_0),(dog,\textsc{blob},h_0)]$\\
has path:
\lstinline{.git/objects/05/b217bb859794d08bb9e4f7f04cbda4b207fbe9}
where \\$h_0 = aa823728ea7d592acc69b36875a482cdf3fd5c8d$,\\
$h_1 = aa823728ea7d592acc69b36875a482cdf3fd5c8d$,\\
$h_2 = aa823728ea7d592acc69b36875a482cdf3fd5c8d$.\\
You may assume $\textsc{blob}$ and $\textsc{tree}$ are constants equal to 100644
and , respectiely.
\end{example}

\frmrule 

\textit{Computing the hash of an empty list}



\frmrule 

\textit{Showing Hierarchies using Object Store Diagrams}


\highlightdef{\textbf{Tree Hierarchies}: each tree object's list of tuples: $[($filetype, filename, hash$)]$
represents the \textit{immediate children} }

That is, each one tuple for each direct child of the tree object. 
Each child may be either a tree object (recursive case) or a 
blob object (base case).


\frmrule 

\begin{example}
For the following directories and files shown, give the corresponding 
object store diagram.
\end{example}

\frmrule 


\begin{sidenote}{Hash Chains}
Notice that to build our Hierarchies, we are using a hash $h_0$ to find a new hash $h$. 
This is an example of a \textit{hash chain}. Let $h(x)$ be a 
hash function, then a \textit{hash chain}
is $h(g(x))$ where $g(x)$ has $h(x')$ inside. 
Note that this definition is \textit{recursive} for $g(x)$ may itself, 
be a hash chain. 

For us, our $h(x)$ is \\
and our $g(x)$ is\\
\end{sidenote}




\frmrule 



\begin{example}


\end{example}


\frmrule 


\highlightdef{File and directory names are \textit{only stored in tree objects}.  }

A common mistake is to think that a filename is 
hashed to address a blob object. It is only the \textit{content} of the 
file that is used to form the address. This is why we call the addresses 
\textit{content} addresses.



\begin{sidenote}{git ls-tree}
For any git project, you can 
see the tuples of a tree object using \lstinline{ls-tree} (list tree). 
Two useful flags are:\\
(i) \lstinline{-r}: recursively show all subtrees (not just 
the immediate children). \\
(ii) \lstinline{-t}: output the trees while recursing 
\end{sidenote}


\frmrule 

\begin{example}
For the following blob-tree hierarchy, give an expression for 
the hash values $h_1$, $h_2$, $h_3$ of tree objects $T_1$, $T_2$, 
$T_3$ respectively. 
\end{example}

\frmrule

\textit{Summary}
\begin{itemize}
\item A single tree object contains one or more tree entries, each of which contains a SHA-1 pointer to a blob or subtree with its associated mode, type, and filename.
\end{itemize}

\chapter{Commit Graphs}

\section{Commit Tree}

\highlightdef{\lstinline{hash-object} helps us find the hash for objects}

\lstinline{echo 'test content' | git hash-object -w --stdin}

\begin{itemize}
\item -w - tells hash-object to store the object. Without this flag, 
the command simple tells us what the hash is
\item --stdin tells hash-object to read the content from stdin. Without this flag, 
hash-object expects a path to a file.
\end{itemize}

\highlightdef{\lstinline{cat-file} displays an object for a given hash}

\lstinline{git cat-file -p d670460b4b4aece5915caf5c68d12f560a9fe3e4}

\begin{itemize}
\item -p - tells cat-file to figure out the type of the content and display 
it nicely when printed in the console
\end{itemize}

\lstinline{git cat-file -t 70460b4b4aece5915caf5c68d12f560a9fe3e4}
\lstinline{blob}

\section{Branches}

\highlightdef{\textbf{Commit Graph}: Vertices = commits, Edges = parent field }
\highlightdef{\textbf{Git Branch}: A path in the commit graph}
\highlightdef{The commit graph will always be a connected directed acyclic graph}

We often called directed acyclic graphs, DAGs. 
One important result about DAGs is the following. 

\highlightdef{Every dag has a natural partial order}

If we let $C$ be the set of commit objects, and define a relation on $C$ 
where $x \succeq y$ if there is a path from $x$ to $y$. Then, following 
from a commit graph being a DAG, we can show 
that $\succeq$ is a \textit{partial order} on $C$.
For a git commit graph, the partial order $\succeq$ is an ordering 
of \lstinline{time}. That is, $c_1 \succeq c_2$ iff $c_1$ is an earlier 
commit than $c_2$.

This is indeed a partial order because it is reflexive. 
Trivially $c \succeq c$ because there is a path from $c$ to $c$. 
The relation is antisymmetric. If we were to have both
$c_1 \succeq c_2$ and $c_2 \succeq c_1$ then there would be a 
path from $c_1$ to $c_2$ and a path from $c_2$ to $c_1$. 
This is a DAG, so there cannot be any cycles, so we must have $c_1 = c_2$.
Finally the relation is transitive. If we have $c_1 \succeq c_2$, and 
$c_2 \succeq c_3$, then there is a path from $c_1$ to $c_2$ and 
a path from $c_2$ to $c_3$. We can join these paths togther to give 
a path from $c_1$ to $c_3$. Hence we have $c_1 \succeq c_3$. 

\highlightdef{One commit may be the parent of one or more commits}

\highlightdef{\lstinline{git branch} creates a new branch}
\highlightdef{Each branch points to a commit}

\frmrule

\textit{Emphasising that $\succeq$ is not necessarily a total order}

Note that $\succeq$ is not necessarily a total order. It is possible 
for there to be two commits $x$ and $y$ and for there not to be a 
path from $x$ to $y$ and for there not to be a path 
from $y$ to $x$. In this case we have neither $x \succeq y$ nor do we 
have $y \succeq x$. We cannot say whether commit $x$ was before $y$ nor 
can we say $y$ was before $x$. In this case we say $x$ and $y$ are incomparible. 

\begin{example}
For the following commit graph show the effect of executing:
\lstinline{git branch dev2}
\end{example}

\frmrule

\highlightdef{HEAD points to the \textit{current branch}}

\highlightdef{\lstinline{git checkout} moves the HEAD pointer}

\begin{example}
For the following commit graph, show the effect of executing 
\lstinline{git checkout dev1}
\end{example}

\frmrule 

\begin{sidenote}{checkout -b}
The following: 
\lstinline{git checkout -b dev2} \\
is the same as: 
\lstinline{git branch dev2}, followed by 
\lstinline{git checkout dev2}
\end{sidenote}


\section{Logs}


\frmrule 

\textit{Introducting the most basic using of} \lstinline{git log}



\highlightdef{\lstinline{git log} shows all commits that are reachable from HEAD}

That is it shows all commits $c$ such that $c_{HEAD} \succeq c$.
Notice that these commits could belong to \textit{different branches}


\frmrule

\begin{example}
For the following commit graph, which commits will be shown by \lstinline{git log}
\end{example}

\frmrule

\textit{Using git log to show reachable commits from a branch tag that isn't HEAD}

\highlightdef{\lstinline{git log} $\langle$ branch name $\rangle$}

\frmrule

\begin{example}
For the following commit graph, which commits will be shown by \lstinline{git log dev2}
\end{example}


\section{Resetting}

\section{Reverting}


\section{Merging}

When you try to merge one commit with a commit that can be reached by following the
first commit’s history, Git simplifies things by moving the pointer forward because there
is no divergent work to merge together. This kind of merge is called a "fast forward".

\highlightdef{\textbf{Fast-forward}: When we try to merge $c_1$ and $c_2$, 
where $c_1 \succeq c_2$ }

What happens if we try to merge two commits there are incomparible/divergent?
Then we have the opposite case of the fast-forward. This is known as the \textit{three-way 
merge}.


\highlightdef{\textbf{Three-way Merge}: When we try to merge $c_1$ and $c_2$, 
where $c_1 \cancel{\succeq} c_2$ }

Because the commit on the branch you're on cannot reach the commit 
of the branch you’re merging in, we cannot do a simple fast-forward;
Git has to do some work. In this case, Git does what is called a \lstinline{three-way merge}.
It uses the two snapshots pointed to by the branch tips, but is also uses 
the \lstiline{least common ancestor} of the two.

A total of three commits are used in three-way merge, $c_1$, $c_2$, 
$c_1 \sqcup c_2$ (hence the three in three-way merge).Git looks at three
 commits to generate the final state of the merge. 

\frmrule

\begin{example}


\end{example}

\frmrule

\textit{More about least common ancestors for commit graphs}

The formal definition of the least common ancestor of $c_1$, $c_2$.\\
The least common ancestor of $c_1$ and $c_2$ is commit $c$ such that: \\
(i) $c \succeq c_1$, $c \succeq c_2$ \\
(ii) There is no $c'$ such that $c \succeq c'$ and $c' \succeq c_1$ and $c' \succeq c_2$ \\

We can call $c$ the join of $c_1$ and $c_2$. 
Figure 3-16 highlights the three snapshots that Git uses to do its merge in this case. 
For any two commits, $c_1$, $c_2$, there always exists the least common ancestor 
$c_1 \sqcup c_2$. 

\begin{sidenote}{Branching and Merging  used to be hard}
Git determines the best common ancestor to use for its merge base;
this is different than CVS or Subversion (before version 1.5),
where the developer doing the merge has to figure out the best merge base for
themselves. This makes merging a heck of a lot easier in Git than in these other systems.
\end{sidenote}

\section{Merge Conflicts}


\highlightdef{Git uses the \lstinline{diff3} algorithm for its three-way merge}

\frmrule 

The first thing diff3 does is to call the two-way comparison tool diff to
find maximum matchings (or longest common subsequences) between $c_1 \sqcup c_2$ and $c_1$
and between $c_1 \sqcup c_2$ and $c_2$.

\frmrule 

\textit{Looking at the longest common subsequence problem}

Recall the \textit{longest common subsequence problem}.
That is, we want to find a new sequence which can be obtained from the first original
sequence by deleting some items, and from the second original sequence by deleting other
items. We also want this sequence to be as long as possible. 

Another way to look at the problem. We want to add spaces to both sequences to that 
we maximise the number of matching characters in idintical indices.

\frmrule

For a given code repository, there are a finite number of commits, 
a finite number of trees and a finite number of blobs. 
There are a finite number of differing lines for the files of a git repository. 
To simply our work, we can assign each unique line a natural number $\{1,2,3,...,k\}$.

\begin{tikzpicture}
\tikzstyle{n}=[draw,font=\small];
\node (c1) [n]{c1};
\node (c2) [n, right=1cm of c1]{c2};
\node (c3) [n, above right=0.5cm and 1cm of c2]{c3};
\node (c4) [n, right=1cm of c3]{c4};
\node (c5) [n, right=1cm of c4]{c5};
\node (c6) [n, below right=0.5cm and 1cm of c2]{c6};
\node (c7) [n, right=1cm of c6]{c7};
\node (c8) [n, right=1cm of c7]{c8};
\node (c9) [n, right=6cm of c2]{c9};
\draw[<-] (c1)--(c2)--(c3)--(c4)--(c5)--(c9); 
\draw[<-] (c2)--(c6)--(c7)--(c8)--(c9);
\end{tikzpicture}


\highlightdef{$\lll\kern-0.75ex\lll,\ggg\kern-0.75ex\ggg$, is the output of diff3 }

A non-crossing matching is a relation $M$ on $\{1,2,3...,k\}$ such that:
\begin{itemize}
\item \textbf{Partial Function} $i\ M\ j$ implies there is no $j'$ such that $i\ M\ j'$ 
\item \textbf{Injection} $i\ M\ j$ implies there is no $i'$ such that $i'\ M_A\ j$
\item \textbf{Non-crossing} $i\ M\ j$ implies there is no $i', j'$ such that 
($i' < j, j' > j$ or $i' > i, j' < j$) and  $i' M j'$
\end{itemize}

We have two matching functions $M_A$ and $M_B$. 
We require $i M_A j$ implies $O[i] = A[j]$ and $i M_B j$ implies $O[i] = B[j]$

A \textit{chunk} (from $A$, $O$ and $B$) is a triple 
$H = ([a_i..a_j], [o_i..o_j], [b_i,..b_j])$ 
of a span in A, a span in O and a span in B. 
We require that at least one of the three spans is non-empty. 
The size of a chunk is the sum of the lengths of the three spans. 
We write $A[H]$ for $A[a_i,..a_j$, and similarly, 
$O[H] = O[o_i,..o_j$ and $B[H] = B[b_i..b_j]$.

A \textit{stable chunk} is a chunk in which all three spans have the same length
 and corresponding indices are matched in all three - i.e., a chunk
($[a..a+k-1],[o..o+k-1],[b..b+k+1])$) for some $k > 0$, with some
$M_A[o+i,a+i] = M_B[o+i, b+i] = true$ for each $0 \leq i < k$. 
That is a stable chunk correspinds to a span in $O$ that is matched in both 
$M_A$ and $M_B$.




An \textit{unstable chunk} is one that is not stable. 





An unstable chunk $H$ is classified as follows:


\section{Rebasing}



\highlightdef{\textbf{Semilattice Homomorphism}: $f(x \sqcup y) = f(x) \sqcup f(y)$ }
If S and T both include a least element 0, then f should also be a
monoid homomorphism, i.e. we additionally require that
$f(0) = 0$.

\highlightdef{A git rebase is a semilattice homomorphism}

allows you to write history differently. Often users use git rebase to 
have what they feel is a cleaner history.

\chapter{Indirection}

\section{Git References}

\begin{figure}[h]
\begin{tikzpicture} [
  every node/.style={node_sty},
  edge from parent/.style={edge_sty},
  edge from parent path={(\tikzparentnode.south) -- ++(0, -0.2cm) -| (\tikzchildnode.north)},
  level/.style = {level distance = 0.6cm},
  level 1/.style={sibling distance=4.5cm, text width=3.5cm}
]
  \node (root) {Git References}
    child{node (b) {References}}
    child{node (t) {Symbolic References}};
\end{tikzpicture}
\end{figure}

\begin{itemize}
  \item \textbf{Reference} 
  you can use that name as pointer rather than the awkward 
  to remember raw sha1 value.
  These are called \textit{references} or simply 
  \textit{refs}. Each reference is a file and is stored 
  in the \lstinline{.git/refs} directory. 
  The contents are simply the sha1 value. 
  \lstinline{cat my_ref}\\
  \lstinline{cat 1a410efbd13591db07496601ebc7a059dd55cfe9}
  \item \textbf{Symbolic Reference}
    It doesn’t generally contain a sha-1 value but rather a
    pointer to another reference. 
    \lstinline{cat .my_symref}\\
    \lstinline{ref: refs/my_ref}
\end{itemize}

\highlightdef{
  \textbf{Reference}: a file storing a sha-1 value under a simple name\\
  \textbf{Symbolic Reference}: a file storing a pointer to a reference 
}

we have \textit{two} layers of indirection. 

\frmrule 

\textit{Updating git references}

\highlightdef{
  \textbf{Update Reference}:
  \lstinline{git update-ref} $\langle filepath \rangle$ $\langle sha1 \rangle$\\
  \textbf{Update Symbolic Reference}
  \lstinline{git symbolic-ref} $\langle sym.filepath \rangle$ $\langle ref.filepath \rangle$
}

\frmrule 

\textit{Symbolic refs can only point to refs in the refs folder}

\lstinline{git symbolic-ref HEAD test}\\
\lstinline{fatal: Refusing to point HEAD outside of refs/}

\frmrule

\textit{Git branches and references}

A branch in Git is simply a reference to a commit 
that represents the front of a line of work. 

\highlightdef{
  Git branch pointers are maintained via references
}
When you run commands like git branch (branchname), Git
basically runs that update-ref command to add the SHA-1
of the last commit of the branch you’re on into whatever
new reference you want to create.  

\frmrule 

\textit{HEAD is a symbolic reference}

The HEAD file is a symbolic reference to the branch 
you're currently on. The HEAD file is a special file 
that's part of git's internals and is always created 
repo initialisation. It is located at \lstinline{.git/HEAD}

When we run \lstinline{git checkout test}, git updates 
the contents of \lstinline{.git/HEAD} from 
\lstinline{ref: refs/heads/master} to \lstinline{ref: refs/heads/test}

\frmrule 

\textit{The difference between HEAD and master}

master is a reference to the end of a branch 
(SVN would call this the 'trunk' of the graph) 

HEAD is a symbolic reference. It may point to master 
or \textit{it may not}. It points to whatever branch 
we are currently workig on. When we commit, 
our changes are commited to the branch HEAD points to

\highlightdef{master is a reference, HEAD is a symbolic reference}

When we know the internals of git, the difference between 
master and HEAD quite clear. However, when we use a GUI 
often master and HEAD appear in the same list and can cause
confusion. 

\section{Tags}

\highlightdef{\textbf{Tag}: like a reference \textit{but it never moves}}

\highlightdef{\textbf{List tags}: \lstinline{git tag}}

\highlightdef{Tags are useful to mark \textit{release points}}


\begin{figure}[h]
\begin{tikzpicture} [
  every node/.style={node_sty},
  edge from parent/.style={edge_sty},
  edge from parent path={(\tikzparentnode.south) -- ++(0, -0.2cm) -| (\tikzchildnode.north)},
  level/.style = {level distance = 0.6cm},
  level 1/.style={sibling distance=4.5cm, text width=3.5cm}
]
  \node (root) {Tags}
    child{node {Lightweight}}
    child{node {Annotated}};
\end{tikzpicture}
\end{figure}

\begin{itemize}
  \item \textbf{Lightweight} A lightweight tag is very 
    much like a branch that doesn’t change — it’s
    just a pointer to a specific commit.
  \item \textbf{Annotated}: Annotated tags, however, are heavier.
    They’re checksummed; contain the tagger name,
    e-mail, and date; have a tagging message; and can be signed and verified with GNU
    Privacy Guard (GPG). Thee additioal annotations make it heavier. So heavy, 
    that we store it as a git object in the Git database
\end{itemize}

\highlightdef{
  \textbf{Create Lightweight Tag}: git tag $\langle name \rangle$\\
  \textbf{Create Annotated Tag}: git tag -a $\langle name \rangle$
  -m $\langle message \rangle$
}

\frmrule 

\textit{Updating our definition of a git object to include annotated tags}



\section{Extended SHA1 Syntax}

A revision parameter typically, but not necessarily, names a
commit object. It uses what is called an extended SHA1 syntax. Here are
various ways to spell object names. The ones listed near the end of this
list name trees and blobs contained in a commit.

\highlightdef{
  $n$\textbf{th Prior}: $\langle ref.name \rangle @ \{ n \}$ \\
  $n$\textbf{th Parent}: $\langle ref.name \rangle \wedge \{ n \}$
}

Tidle-n is equivalent to n carets. 
\highlightdef{
  $\langle ref.name \rangle \sim n \equiv \langle ref.name \rangle \wedge \wedge ... \wedge$
}

\frmrule 

\begin{example}
For the following commit graph, state which commit
the following extended syntax identifiers
refer to:
(a) master$\wedge$ \\
(b) master$\wedge2$ \\
(c) master$\wedge\wedge 2\wedge$ \\
(d) master$\tidle 4$ \\
(e) master$\wedge\wedge\wedge\wedge\wedge$
\end{example}


\frmrule 

\begin{example}
Explain the effect of the following: \\
\lstinline{git reset --hard HEAD~1}
\end{example}


\chapter{Remotes}

\section{Working with Remotes}

\highlightdef{
  \textbf{Add Remote}:
  \lstinline{git remote add} $\langle name \rangle$ 
  $\langle url \rangle$
}

\frmrule 

\begin{example}
  \lstinline{git remote add origin https://github.com/<username>/first_app.git}
\end{example}

\frmrule

\highlightdef{
  \textbf{List Remotes}:
  \lstinline{git remote}
}
This lists all the remotes by their shortname. 
You can also add the flag \lstinline{-v} to show the urls for each 
remote in the list. 

\frmrule 

\highlightdef{
  \textbf{Inspecting Remotes}:
  \lstinline{git remote show} $\langle remote.name \rangle$
}

\frmrule 

\section{Remote Graphs}

Remote branches are references to the state of branches on your remote repositories.
They’re local branches that you can’t move; they’re moved automatically whenever you do
any network communication. Remote branches act as bookmarks to remind you where the
branches on your remote repositories were the last time you connected to them.

\highlightdef{Remote branch names are of the form 
$\langle remote \rangle / \langle branch\rangle$}



\highlightdef{
  A Git clone gives you your own master branch and origin/master
  pointing to origin’s master branch.
}

\highlightdef{
  \lstinline{git clone} $\langle url \rangle$ 
  automatically adds a remote called \lstinline{origin} 
  with url, $\langle url \rangle$
}


\begin{example}
  Complete the diagram to show the result 
  of \lstinline{git clone} on host $H_2$. 
  Show the list of remotes and show all branches and the commits they 
  point to.
\end{example}

\frmrule 


\begin{example}
  Complete the diagram to show the result of
  two commit graphs when three commits are added 
  to the graph on $G_1$. 
\end{example}

\frmrule 

\highlightdef{\textbf{Distributed Histories} when $H_1$ updates his graph, 
  the remote references on $H_2$ \textit{don't move}
}

\frmrule 

\textit{Introducting git fetch as a command that synchronises branches}

\highlightdef{\lstinline{git fetch} updates remote references}




\frmrule 

\section{Pushing}


\highlightdef{
  Git push allows one host to write to a graph of one of his remote
}

\highlightdef{
  \lstinline{git push} $\langle remote.name \rangle$
  $\langle branch.name \rangle$
}





\frmrule 

\textit{Pushing tags}

By default git push doesn't send the tags up, you also need push 
to include all the tags in the push by using the \lstinline{--tags} flag.


\highlightdef{\textbf{Tracking Branch} }

\highlightdef{\lstline{-u} in \lstinline{git push -u} $\langle bname \rangle$
will cause the local branch to automatically track the remote branch }

\frmrule 

Git pull:
\begin{itemize}
  \item Adds (origin, url) to the remotes list
  \item creates a branch called master
  \item sets up your local master branch to track the remote master branch
\end{itemize}

\section{Pulling}


%\chapter{Staging Area}

%\section{Detecting Changes}

%\section{Stashing}
%dsd
















