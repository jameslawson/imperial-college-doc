
\chapter{Counting}








\section{Balls and Bins I}


\highlightdef{Number of Balls: $n$, Number of Bins: $b$}


\begin{example}
\textbf{Distinct balls - distinct bins}:
Suppose that $n$ balls are distinct and that their order within a bin does not
matter. Argue that the number of ways of placing the balls in the bins is $b^n$.
\end{example}



\textbf{DD - binary bins}:
Show that the number of possible configurations is $P(b,n)$.

\textbf{DD - fixed size}:
Show that the number of possible configurations is a \textit{multinomial}.


\section{ID - Binary Bins}


Suppose that the balls are indistinguishable, and hence their order within a bin 
is not accounted for. Show that the number of possible configurations is $C(b, n)$.

\highlightdef{\textbf{BinaryString}: The number configurations a 
string of length $n$ using two symbols that has $b$ of one symbol is $C(b,n)$ }


\textit{BinaryString} and \textit{ID - Binary Bins} are very much 
the same problem.  

\frmrule

\highlightdef{$C(n,k) = C(n,n-k)$}



\highlightdef{The number of integers satisfying $1 \leqslant x_1 < x_2 < ... < x_k \leqslant n$ is $C(n,k)$}


\section{ID}

Suppose that the balls are indistinguishable, and hence their order within a bin 
is not accounted for. The number of possible configurations is $C(n+b-1, n)$.

Every configuration can be encoding into a string of dots and seperators. 
Each string must have $n$ dots, because we must use all $n$ balls. 
Each string must have $b-1$ seperators, because for $b$ boxes we need $b-1$ seperators.
So we have reduced the problem to finding the number of configurations of a
a length $n + b-1$ string of two symbols, where we need $n$ of one symbol. 
This is a reduction to \textit{binaryString}. So the solution is $C(n + b -1, n)$. 

\frmrule

\begin{example}

\end{example}

\frmrule

\begin{example}
\textit{Chemistry - Counting Energy States} \\
Let $N$ represent the number of atoms and $q$ represent the number of quanta of energy. 
Represent quanta by dot. Use seperators to show divide atoms.
Each string must have $q$ dots, because we must use all the energy and there is $q$ quanta. 
Each string must have $N-1$ seperators, because for $N$ atoms we need $N-1$ seperators.

So the number of ways or arranging the energy is $C(N + q - 1, q)$. 
\end{example}

\frmrule

\section{ID - Nonempty Bins}


Suppose that the balls are indistinguishable, and that no bin may be left empty. 
Assuming $n \geqslant b$ (enough balls to fill bins),
show that the number of possible configurations is $C(n-1, b-1)$.

\frmrule

A common Balls and Bins application involves enumerations with letters.
\highlightdef{
Indistinguishable Balls = Sequences of Letters\\
Indistinguishable Balls = Unordered set of Letters}

\frmrule

\begin{example}
Express the following problem as a Balls-to-Bins problem. 
\end{example}

\frmrule 



\section{Identities}


Pick $k$ people out of $n$ with 1 chosen to be president. 
\highlightdef{$nC(n-1,k-1) = kC(n,k)$}

\frmrule 

Choosing $k$ people from $m+n$ people. 
Any given configuration has $j$ people from $M$ group 
and $k-j$ people form the $N$ group. By multiplication 
principle, there are $C(m,j)C(n,k-j)$ different configurations 
with $j$ people from $M$ group. So adding up configurations 
for $j = 0, 1, ..., k$ we have:

\highlightdef{$C(m+n, k) = \sum^{k}_{j=0}C(m,j)C(n,k-j)$}
This identity is known as \textit{Vandermonde's Identity}.

\begin{example}
Give a combinatorial argument to show $C(3n,n) = \sum^{n}_{r=0}C(n,r)C(2n,n-r)$.
\end{example}


\frmrule

\highlightdef{$C(n, k) = C(n-1,k-1) + C(n-1, k)$}
This identity is known as \textit{Pascals's Identity}.

\frmrule

\highlightdef{$\sum^{n}_{k=0}C(k, r) = C(n+1,r+1)$}
This identity is known as \textit{Chu's Theorem}.


\frmrule

\section{Balls and Bins - Stacks}

Sometimes known as the \textit{Flagpole Problem}

\begin{example}
\textbf{DD - w/binorder}:
Suppose that the balls are distinct, the bins are distinct, 
but now, the balls in each bin are ordered. Each bin 
acts like a stack and stores the order in which the ball was placed in the bin. 
Prove that there are exactly $(b+n-1)!/(b-1)!$ ways to place the balls in the
bins. (Hint: Consider the number of ways of arranging $n$ distinct balls and $b-1$
indistinguishable sticks in a row.)
\end{example}


