
\chapter{Capital Asset Pricing Model}



\section{Introducing CAPM}

\highlightdef{
Risk is measured by $\beta_i$ not the variance of the asset
}


\highlightdef{
You are only rewarded with a \textit{positive excess expected return} \\
for \textit{risk that you cannot diversify away}
}


\highlightdef{
The return of asset $i$ is determined by \textit{how it fits into 
the market portfolio}.\\ Not by it's characteristics alone.
}




\section{Security Market Line}

For a given asset, $i$, we define the \textit{Market Security Line}: 
\highlightdef{\textbf{Security Market Line}: $(\overline{r_i},\beta_i)$-graph 
with plot: $\overline{r_i} =  (\overline{r}_M - r_f) \beta_i + r_f$ }

This is a graph of the form $y = (b-c)x + c$. 
The gradient is $(b-c)$ and the $y$-intersept is $c$. 
In other words, the Market Security line has gradient $(\overline{r}_M - r_f)$ 
and $y$-intersept $r_f$.

For any given value of $\beta_i$, that is, any value on the $x$-axis, 
there is a corresponding value of $r_i$ - a corresponding value on the $y$-axis. 
In other words, knowing well the $i$th asset \textit{correlates with the market}, 
is the same as knowing, the average return rate $\overline{r_i}$. 
But we also know $\overline{r}_f$. So really, knowing $\beta_i$, the market 
correlation for asset $i$, is the same as knowing the expected \textit{excess} rate 
of return: $r_i - \overline{r}_f$ given for asset $i$ in market $M$. 

Take care to notice the difference between the \textit{Capital} Market Line and 
the \textit{Security} Market Line. Both of them relate expected returns with 
risk. The difference is that the CML looks at an \textit{entire portfolio} whereas 
the SML only looks at only an \textit{individual asset}.  

\highlightdef{
\textbf{CML}: return vs risk for a \textit{portfolio} \\
\textbf{SML}: return vs risk for an \textit{asset}
}



\section{Systematic Risk}



Recall that covariance is \textit{bilinear}. 
That is, $\text{cov}[aX + bY,Z] = a\text{cov}[X,Z] + b\text{cov}[Y,Z]$,  
and $\text{cov}[Z,aX + bY] = a\text{cov}[Z,X] + b\text{cov}[Z,Y]$. 

\[
\begin{array}{ll}
 \text{cov}[r_i, r_i] & = \text{cov}[(r_M-r_f)\beta_i + r_f + \epsilon_i, \;\;\;(r_M-r_f)\beta_i + r_f + \epsilon_i] \\
 &= \beta_i^2 \text{cov}[r_M, r_M] + \beta_i^2 \text{cov}[r_f, r_f] + \text{cov}[r_f, r_f] + \text{cov}[\epsilon_i, \epsilon_i]  \\
 &\quad + 2( \beta^2_i \text{cov}[r_M, r_f] + \beta_i \text{cov}[r_M, r_f] + \beta_i \text{cov}[r_M, \epsilon_i] \\
 &\quad\quad  - \beta_i \text{cov}[r_f, r_f] - \beta_i \text{cov}[r_f, \epsilon_i] + \text{cov}[r_f, \epsilon_i]) \\
\end{array}
\]

Now $\text{cov}[r_f, r_f] = \sigma^2_f = 0$ because, a risk-free asset, by definition, has zero risk and so
 \textit{zero variance}. Risk-free assets also have no affect on the market, so 
 $\text{cov}[r_M, r_f] = 0$. We have shown that CAPM implies $\text{cov}[r_M, \epsilon_i] = 0$
and also $\text{cov}[r_f, \epsilon_i] = 0$. Hence the above equation simplifies to:
$\text{cov}[r_i, r_i] = \beta_i^2 \text{cov}[r_M, r_M] + \text{cov}[\epsilon_i, \epsilon_i]$. 
In other words, $\sigma^2_i = \beta_i^2 \sigma^2_M + \text{var}[\epsilon_i]$. 

\highlightdef{
\textbf{Risk Formula}: $\sigma^2_i = \beta_i^2 \sigma^2_M + \text{var}[\epsilon_i]$
}

We are familiar with idea that the risk of an asset $i$ is associated its variance $\sigma^2_i$. 
What the \textit{Risk Formula} reveals is that this risk has \textit{two terms} and crucially, 
one term depends on the market, and another does not depend on the market. 
We call these two forms of risk, \textit{systematic risk} and \textit{nonsystematic risk} 
respectively. 

\begin{itemize}
\item \textbf{Systematic Risk}
This is given by the term $ \beta_i^2 \sigma^2_M$. 
This is the risk that the asset has when taking the market into account. 
This risk caanot be reduced unless $\beta_i = 0$, ($\beta^2_i \sigma^2_M = 0$ iff $\beta_i = 0$)
but of course, when $\beta_i = 0$, the asset would be risk-free (that is completely risk-free). 
So for any asset that isn't risk-free, $\beta_i > 0$ and this systematic risk is indeed present. 
Diversification cannot reduce this risk because any combination of assets will contain this risk. 
The risk is independent of the choice of assets put into the mix and so will be present
in any combination. 
\item \textbf{Nonsystematic Risk} 
This is given by the term $ \text{var}[\epsilon_i]$. 
This risk is uncorrelated with the market and exists purely with the asset. 
This risk can be reduced by \textit{diversification}. That is, reduce 
risk, without changing the portfolio's mean return, by mixing and matching 
assets in a different way. This form of risk is also called the
\textit{idiosyncratic risk} or the \textit{specific risk}. 
\end{itemize}

Nonsystematic risk can be shown on the capital market line. 
If $\epsilon_i = 0$ then $r_i = \overline{r}$. The asset 
falls perfectly on the CML. However if $\epsilon_i > 0$, the risk is higher, so the 
value of $\sigma$ increases. The result is that the 
$x$-coordinate increases and we start to see that
the coordinate drifts to the \textit{right} of the CML. 

\section{CAPM Pricing Formula}

We can use the CAPM formula to give a \textit{pricing formula}. 
Suppose we have only one asset. So instead of writing $\beta_i$, $r_i$, 
we write just $\beta$, $r$.

$r = \frac{Q-P}{P}$. 
$\overline{r} = r_f + \beta(\overline{r}_M - r_f)$. 

Combining gives:
$$P = \frac{Q}{1 + r_f + \beta(\overline{r}_M-r_f)}$$

This formula has an interesting interpretation. 
In particular, it looks as through we are taking $Q$ 
and, rather than discounting it by a factor of $1 + r_f$, 
we discount it by an \textit{adjusted factor} of 
$1 + r_f + \beta(\overline{r}_M-r_f)$. 
We need to increase the reduction by $\beta(\overline{r}_M-r_f)$
to account for the uncertainty in payoff $\overline{Q}$. 
After all, it is a random variable and we cannot be sure
for it has a risk associate with the market. 

Another way is to say, using a non-adjusted rate $\frac{1}{1+r_f}$, 
what's adjusted payoff $Q'$. 

$$P = \frac{1}{1+r_f}\left(\overline{Q} - \frac{\text{cov}(Q,r_M)}{\sigma^{2}_{M}}(\overline{r}_M - r_f) \right)$$.

From this we have that the adjusted payoff 
$Q' = \left(\overline{Q} - \frac{\text{cov}(Q,r_M)}{\sigma^{2}_{M}}(\overline{r}_M - r_f) \right)$.


A firm can use the CAPM to decide which projects it
should carry out.
It is natural to define the NPV of a project that costs P
and generates a random payoff Q as

$$\text{NPV} = -P + \frac{1}{1+r_f}\left(\overline{Q} - \frac{\text{cov}(Q,r_M)}{\sigma^{2}_{M}}(\overline{r}_M - r_f) \right)$$

It is appropriate for the firm to select the group of
projects that maximize NPV.