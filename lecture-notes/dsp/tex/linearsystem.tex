\chapter{Linear Systems}



\section{Introducing Linear Systems}




\highlightdef{\textbf{Linear System}: A system is \textit{linear} when }



\highlightdef{\textbf{Time-invariant System}: A system is \textit{time-invariant} when:\\
for all $q$, $y[n] = Hx[n]$ implies $y[n-q] = Hx[n-q]$ 
}

This expresses the idea that the system behaves the same \textit{no matter when} the input is applied. 

A system that is \textit{not} time-invariant is called time-varying.
Time-invariant systems are sometimes called \textit{shift-invariant} systems. 



\frmrule 

\begin{example}
Show that the \textit{Moving Average} system is time-invariant.

Assume $y[n] = Hx[n]$. \\
$RHS = Hx[n-q] = \frac{1}{2}(x[n-q] + x[n-q-1]) = y[n-q] = LHS$ \\
\end{example}

\frmrule 

\begin{example}
Show that the \textit{Decimation} system is \textit{time varying}.

Assume $y[n] = Hx[n]$. \\
$RHS = Hx[n-q] = x[2(n-q)]$ by def our assumption and def of $H$. \\
$LHS = x[n-q]$. \\
Clearly, $x[2(n-q)] \neq x[n-q]$ for some $q$, hence 
the system is \textit{not} time-invariant. It is \textit{time varying}. 

\end{example}


\frmrule 

\highlightdef{\textbf{Linear Time-invariant System}: A system is \textit{linear time-invariant} when:\\
it is both linear \textit{and} time-invariant.
}



\section{Properties of LTI Systems}



\highlightdef{Toeplitz property: $h_{i,j} = h_{i+q,j+q}$}

All the entries in a Toeplitz matrix can be expressed as 
in terms of the entries of either 
(i) the 0th column \\
(ii) the 0th row.

(i) \\
$(H)_{n,m} = h_{n,m} = h_{n-m,0}$ by Toeplitz property\\
$= h[n-m]$. 

(ii)\\
$(H)_{n,m} = h_{n,m} = h_{0,m-n}$ by Toeplitz property\\

\section{Convolution}