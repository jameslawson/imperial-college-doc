
\chapter{Parsing III: Bottom-up}



\section{Introducing Bottom-up Parsing}

\textit{Contrasting top-down with bottom-up parsing }



\frmrule 




\section{Unique Handles}


\textit{Defining the handle of a right sentential form } 

For grammar $G$ with start non-terminal $S$, 
\highlightdef{\textbf{Sentential Form}:
any string $\alpha$, such that $S \Rightarrow^{*} \alpha$ is called 
a \textit{sentential form}. }

The start symbol is a sentential form because trivially, $S \Rightarrow^{*} S$. 
Any terminal string in $L(G)$ is a sentential form because $x \in L(G)$ implies $S \Rightarrow^{*} x$. 
More importantly, any \textit{half-finished derivation} is a sentential form. 
A sentential form consists of a sequence of terminals and non-terminals that can 
be legally derived from $S$ using rules form $G$. Any derivation corresponds to 
a sentential form (and vica versa). 

\highlightdef{\textbf{Sentence}:
any sentential form $\alpha$ where $\alpha \in T^{*}$ is called a \textit{sentence}. }

A sentence is a \textit{fully-finished derivation}. In other words, 
a sentence is a sentential form that is a terminal string in $L(G)$. It is a 
sentential form comprised only of \textit{terminals}, $\alpha \in T^{*}$.


\frmrule 

\highlightdef{\textbf{Handle}: if $S \Rightarrow^{*} \alpha X \gamma \Rightarrow \alpha \beta_i \gamma$ 
then $\beta_i$ is a handle of the\\ sentential form $\alpha \beta_i \gamma$ at the position following $\alpha$ }

It is convenient to say $\beta_i$ is a handle of $\alpha \beta_i \gamma$ when 
\textit{it is clear} what production and position are meant. 
Suppose we have a right sentential form $= \alpha \beta_i \gamma$ that is not the start symbol,
then such a rsf can have have one or many \textit{handles}. 
We will explore examples which show this and think about how 
we can find a way of enforcing a rsf to have \textit{exactly one} handle. 

\frmrule 

\begin{example}
Consider the grammar:\\
$S \rightarrow AaS | b$ \\
$A \rightarrow c | d | B$ \\
$B \rightarrow AgC | AhC | DgC | DhC$ \\
$C \rightarrow c | d | D $ \\
$C \rightarrow eBf $ \\
(a) Show that $cab$ is a sentential form of the grammar. \\
(b) Show that $cab$ has \textit{two} handles. 
\end{example}

\frmrule 

(b) \\
There are actually \textit{two} derivations, 
$S \Rightarrow AaS \Rightarrow caS \Rightarrow cab$\\
$S \Rightarrow AaS \Rightarrow Aab \Rightarrow cab$\\
And they differ in the last step. Which means we have two handles. \\
$c$ is a handle of $cab$ at position 0, \\
$b$ is a handle of $cab$ at position 1, 

\frmrule 

\begin{example}
Consider the grammar:\\
$S \rightarrow AaS | b$ \\
$A \rightarrow c | d | B$ \\
$B \rightarrow AgC | AhC | DgC | DhC$ \\
$C \rightarrow c | d | D $ \\
$C \rightarrow eBf $ \\
For the following, (i) show that each is a sentential form of the grammar.
and (ii) find the \textit{total number of handles} each sentential form has. \\
(a) $Aab$ (b) $AaS$ (c) $b$  (d) $Aacab$
\end{example}



\frmrule 

\textit{A sentential form $\neq S$ must have at least one handle}




\frmrule 

\textit{A sentential form $\neq S$ may have several handles}

\frmrule 

\textit{Idea: Using a rightmost derivation could help enforce unique handles}


\frmrule 




\highlightdef{\textbf{Right Sentential Form}:
any string $\alpha$, such that $S \Rightarrow^{*}_{rm} \alpha$ \\
is called a \textit{right sentential form} (RSF). }

That is, a \textit{right sentential form} is simply a sentential form that 
is a result of using a rightmost derivation. 
We can similarly define a \textit{left} sentential form 
a sentential form that 
is a result of using a \textit{leftmost} derivation 
($\alpha$ s.t. $S \Rightarrow^{*}_{lm} \alpha$). 




%\highlightdef{\textbf{Handle}: if $S \Rightarrow^{*}_{rm} \alpha X \gamma \Rightarrow_{rm} \alpha \beta_i \gamma$ 
%\\ then $\beta_i$ is a handle of the rsf $\alpha \beta_i \gamma$ at the position following $\alpha$ }

First notice that when we say $\alpha X \gamma \Rightarrow_{rm} \alpha \beta_i \gamma$ 
we used $\Rightarrow_{rm}$ to specify a one-step \textit{rightmost derivation}. We didn't 
use $\Rightarrow$ (any derivation). What is implies is that 
$\gamma$ has no non-terminals. Why? Well, according to $\alpha X \gamma \Rightarrow_{rm} \alpha \beta_i \gamma$
we replaced the rightmost non-terminal $X$ with $\beta_i$. 
If there were any non-terminals in $\gamma$, then $X$ would not've been the rightmost non-terminal.

So $\gamma$ is a sequence of zero or more \textit{terminals}. 
To make this idea clearer, rather than writing gamma, we can invent a symbol to denote 
zero or more terminals (like $w$ for \textit{word}) and then write $\alpha X w \Rightarrow_{rm} \alpha \beta_i w$.
So some definitions will define a handle as:
if $S \Rightarrow^{*}_{rm} \alpha X w \Rightarrow_{rm} \alpha \beta_i w$, 
then $\beta_i$ is a handle of the rsf $\alpha \beta_i w$. This definition (with $w$) 
and the original (with $\gamma$) are \textit{equivalent}. 

In other words, a \textit{handle} of a rsf is a string $\beta_i$ in the rsf
that can replace the rightmost non-terminal $X$.


%\frmrule 

%\textit{Looking a handles from the perspective of a reverse rightmost derivaiton}
%In other words a \textit{handle} 
%\highlightdef{A handle corresponds to a step in a rightmost derivation, 
%\textit{from a reverse perspective}  }

% Notice that we have two rsfs, , but the definition is 
% talking about  - that is the later one in the rightmost derivation, which as we will see,
% will be the earlier one in the LR parse.


\frmrule 

\textit{A rsf $\neq S$ must have at least one handle}




\frmrule 

\textit{A rsf $\neq S$ may have several handles}


\begin{example}
For the simple grammar:\\
$S \rightarrow Tat | TaT$\\
$T \rightarrow t$\\
(a) Show that $tat$ is a rightmost sentential form\\
(b) Show that $tat$ has \textit{two} handles.
\end{example}



\frmrule 

\textit{Idea: Making the grammar unambiguous could help enforce unique handles}

\frmrule 


\textit{Exploring the conditions for unique handles}

In turns out that the conditions for unique handles are \textit{very specific}. 
The table below shows how many handles a sentential could have. There is 
only one special case where we can guarantee a sentential form has a unique
handle, and that is when (i) $G$ is \textit{unambiguous}, 
(ii) the sentential form is a \textit{rsf} (iii) the sentential form is \textit{terminal}.

\begin{table}[h]
    \centering
    \begin{tabular}{l|lll}
                        & sf & rsf  & terminal rsf\\ \hline
    $G$ ambiguous       & $\geqslant 1$ & $\geqslant 1$ & $\geqslant 1$ \\
    $G$ unambiguous     & $\geqslant 1$ & $\geqslant 1$ & exactly $1$ \\
    \end{tabular}
     \caption{Conditions required for unique handles.} 
\end{table}

In all other cases it is possible that a sentential form has a non unique handle. 
Intuitively, using rightmost sententials reduce the chances of non-uniqueness. 
And, using an unambiguous also intuitively reduces the chances of non-uniqueness. 
These were \textit{good ideas}, but there were not strong enough. 
We also need a third condition, that the rsf is \textit{terminal}. 


\frmrule 

\textit{A rsf $\neq S$ of an umambiguous grammar may have several handles}

\begin{example}
This grammar is unambiguous:\\
$S \rightarrow A | a$ \\
$A \rightarrow AA | B$ \\
$B \rightarrow AA$ \\
(a) Show that $AA$ is a right sentential form. \\
(b) Show that $AA$ has more than one handle.

(a) $S \Rightarrow_{rm} A \Rightarrow_{rm} AA$\\
(b) $S \Rightarrow_{rm} A \Rightarrow_{rm} B \Rightarrow_{rm}$ AA. \\
So $AA$ has two handles, $A$ and $B$, in position 0. 
\end{example}


\frmrule 

\textit{A terminal rsf $\neq S$ of an umambiguous grammar has exactly one unique handle}

Recall that G is unambiguous then all rightmost derivations are \textit{unique}. 
Unique rightmost derivations are important for \textit{two} reasons. 
First reason, we can only replace \textit{one non-terminal} at a time. 
This is true for all rightmost derivations. Second reason, 
there is only \textit{one alternative} that will fit into the proof. 
Although $X$ may have several altnatives, $\beta_1$, ..., $\beta_n$, 
only one of these will fit. That is what a \textit{unique} rightmost derivation is.

This means that if we have a rsf (that isn't the start symbol): 
$S \Rightarrow^{*}_{rm} \alpha \beta_i \gamma$, 
then when we look at the steps, $S \Rightarrow_{rm} \gamma_1 \Rightarrow_{rm} \gamma_2 ... \gamma_k \Rightarrow_{rm} \alpha \beta_i \gamma$, 
to take us from $\gamma_i$ to $\gamma_{i+1}$ there is firstly, a unique position to replace a 
nonterminal (by the fact it is \textit{rightmost}), 
and secondly a unique alternative (by the fact that is is rightmost \textit{unique}). 

Suppose we have a rsf that isn't the start symbol: $\alpha \beta_i \gamma$.
Since it is a \textit{rightmost} sentential form, we know that it is part 
of a rightmost derivation. So in the previous step, there was unique position to replace a 
nonterminal. A second, because the grammar is unambiguous, this 
rightmost derivation is \textit{unique}, so there is only one 
\textit{alternative} that we can replace the rightmost nonterminal with 
(it is necessary that all other alternatives cannot fit). 

Assume $\beta_i$ was not the handle.
Hence $\beta_i$ must be the handle.

\highlightdef{If $G$ is unambiguous, then every terminal RSF$\neq S$ has a \textit{unique} handle}


This means that every derivation is completely unique. 
There is no room for choice. At every step, there is exactly 
one non-terminal to replace and exactly one alternative of this non-terminal that must be chosen. 
Bottom-up parsing exploits this property. By taking the rightmost derivation 
of a unambiguous grammar, we can work backwards from some rightmost sentential form 
and always find our way back to $S$. 


\frmrule 

\textit{Introducing Handle-pruning as using handles to construct a bottom-up parse} 


Given a unambiguous grammar $G$, 
the process of constructing a bottom-up parse from these unique handles of rightmost sentential forms
is called \textit{handle pruning}. 

Assume that input $w$ in $L(G)$. Then $w$ has a unique rightmost derivation. 
$$S \Rightarrow_{rm} \gamma_1 \Rightarrow_{rm} ... \Rightarrow_{rm} \gamma_k \Rightarrow_{rm} w$$ 

The handle pruning algorithm reconstructs this derivation via the algorithm:
\begin{lstlisting}
for i = n to i
  find: find handle beta_i in gamma_i
  prune: replace beta_i with X to generate gamma_{i-1}
\end{lstlisting}


\frmrule 

\begin{example}
Given the following unambiguous grammar. \\
Using handle pruning, find the bottom-up parse of $w = $.
\end{example}

\frmrule 

\textit{Highlighting the similarity between the handle-pruning algorithm 
and the basic bottom-up parsing algorithm} 




\frmrule 


%\textit{Refine handle to keep track of positions} 

%\highlightdef{\textit{Redefine:} \textbf{Handle}: if $S \Rightarrow^{*}_{rm} \alpha X \gamma \Rightarrow_{rm} \alpha \beta_i \gamma $ 
%then }


\section{Recognising Handles I}





\frmrule 

\textit{Compare the problem of finding a unique handles in LR parsing with the problem 
predicting rules in LL parsing }

(i) identify the unique handle $\beta$:
$\gamma_i = \alpha \beta\gamma$\\
(ii) determine which production to use to give $\gamma_{i-1} = \alpha X \gamma$

\frmrule 

\textit{Recognising handles is a hard problem. }

\highlightdef{There are no known efficient algorithms to recognise handles }

However there are a certain subclass of CFGs where we can efficiently 
recognise handles. There are called \textsf{LR}($k$) grammars. 
With such grammars, we can find and prune handles by only looking 
ahead at most $k$ symbols past the unique handle. 

We will still only consider unambiguous grammars 
and only consider terminal rightmost sentential forms. 
If we do this, then the handles are indeed unique. 

\frmrule 



\textit{Formal definition of} \textsf{LR}$(k)$ \textit{grammars }


Formally, a grammar $G$ is $\textsf{LR}(k)$ iff:\\
$S \Rightarrow^{*}_{rm} \alpha X xz \Rightarrow_{rm} \alpha \beta xz$ \\
$S \Rightarrow^{*}_{rm} \gamma      \Rightarrow_{rm} \alpha \beta xy$ \\
$|x| = k$\\
implies \\
$\gamma = \alpha X xz$, and $z=y$

Suppose we are only allowed to scan at most $k$ characters 
past the unique handle of our rsf $\alpha \beta \gamma$.
We don't know what is remaining so we could have two different 
strings: $\alpha \beta xy$, $\alpha \beta xz$.
Suppose we know that for the ending $xz$ we find $X \rightarrow \beta$. 
What happens in the case where the ending was $xy$. 
An  $\textsf{LR}(k)$ grammar tells us that this case is exactly 
the same.

In other words, we only need to consider at most $k$ symbols after the handle,
to know realise that the proof trees are the same therefore know
the handle to be applied.

Informally, we can say that a grammar $G$ is $\textsf{LR}(k)$ if 
given a rsf $\gamma$, we can decide find and prune by scanning 
$\gamma_i$ from left-to-right going no further than $k$ symbols 
past the unique handle, $\beta_i$. 

\frmrule 

\textit{Formal definition of} \textsf{LR}$(0)$ \textit{grammars }

Formally, a grammar $G$ is $\textsf{LR}(0)$ iff:\\
$S \Rightarrow^{*}_{rm} \alpha X z \Rightarrow_{rm} \alpha \beta z$ \\
$S \Rightarrow^{*}_{rm} \gamma      \Rightarrow_{rm} \alpha \beta y$ \\
implies \\
$\gamma = \alpha Xz$, and $z=y$

\frmrule 



\frmrule 

\begin{example}
Show that the following if-then-else grammar is \textit{not} \textsf{LR}($0$).\\
$S \rightarrow ictS | ictSeS | a$ 

We have $S \Rightarrow^{*}_{rm} ictS \Rightarrow_{rm} icta$ \\
and have $S \Rightarrow^{*}_{rm} ictSea \Rightarrow_{rm} ictaea$, \\
yet $ictS \neq ictSea$ (and furthermore $a \neq ea$), so the grammar is 
\textit{not}  \textsf{LR}($0$).
\end{example}

\frmrule 

\begin{example}
Show that the following polish postfix grammar \textit{is} \textsf{LR}($0$).\\
$S \rightarrow SSa | Sb$ 
\end{example}

\frmrule


\textit{Introducing Maximally Right Handles }

\begin{example}
Consider the ambiguous grammar: \\
$S \rightarrow Sa | Saa | a$\\
We have exactly two handles of sentential form $Saa$\\
$S \Rightarrow Saa$, $S \Rightarrow Sa \Rightarrow Saa$ 

The first handle $Saa$ and the second is $Sa$. Consider 
where the right end of each handle lies in the sentential form. 
The first handle's right end is further right than the second's. 
We say that $Saa$ is a \textit{Maximally Right Handle} for this reason. 
\end{example}

\highlightdef{\textbf{Maximally Right Handle}: $\beta_i$ is a \textit{maximally-right} handle 
of a sentential form $\gamma$ if $\beta_i$ is a handle of $\gamma$ and 
its right-end is maximally far to the right of $\gamma$ when compared with 
other handles of $\gamma$}

Maximally right handles are \textit{not necesarily unique}. We may have 
two handles that who have the same right end that is the furthest right.  
Note we cannot say \textit{the} rightmost handle. There may be several maxima. 
However when we have a unambiguous grammar and consider a terminal right sentential form, 
then we know there is only \textit{one} handle. In this case, such a handle 
is the only handle and so it is trivially the maximally-right handle.

\frmrule 

\textit{Introducing viable prefixes and relating them to handles }


\highlightdef{\textbf{Viable Prefix}: We say $\alpha$ is a \textit{viable prefix} (of $G$) if 
(i) there is some sf $\alpha\beta$ (ii) $\alpha$ does not continue past the rightend 
of any maximally right handles of $\alpha\beta$ }

This definition is a little tricky. We can translate this to first-order logic: 
$$viable(\alpha) \leftrightarrow \exists y: rsf(x) \wedge \forall y: (handle(x,y) \wedge maxRight(x,y) \rightarrow \alpha.last < y.last)$$
The first important point is that it is of the form $\exists\forall$. 
We need just \textit{some} rsf, $x$, such that $\alpha$ does not continue past \textit{all} handles, $y$
that are maximally right of $x$. 

% By this definition, it is always possible to add terminal symbols 
% to the end of $\alpha$ to obtain a right sentential form. 

\frmrule 

\begin{example}
For the following grammar,\\
$S \rightarrow cAd$\\
$A \rightarrow ab | a$

Determine whether the following are viable prefixes. \\
(a) $cAd$ (b) $\epsilon$ (c) $c$ (d) $cA$
\end{example}

\frmrule 

\begin{example}
For the following grammar,\\
$S \rightarrow cAd$\\
$A \rightarrow ab | a$

Determine whether the following are viable prefixes. \\
(a) $cab$ (b) $cad$ (c) $cabd$ (d) $S$
\end{example}


\frmrule 

\begin{example}
For the following grammar,\\
$S \rightarrow AaS | b$\\
$A \rightarrow c | d | B$\\
$B \rightarrow AgC | AhC | DgC | DhC$\\
$C \rightarrow c | d | D $\\
$D \rightarrow eBf $ \\
Determine whether the following are viable prefixes. \\
(a) $Aab$ (b) $ca$ (c) $AgCS$
\end{example}

\frmrule

\begin{example}
For the previous grammar, is there a longest viable prefix if finite-length?
(\textit{Hint:} $S \Rightarrow AaS \Rightarrow AaAaS \Rightarrow AaAaAaS \Rightarrow ...$)
\end{example}


\frmrule



\textit{The start symbol $S$ is always a viable prefix }

Notice that $\alpha = S$ is a prefix of some right-sentential form, 
namely $\alpha\beta = S$. 
Recall that the start symbol doesn't have any handles. 
Therefore,
$\alpha = S$ doesn't continue past 
the right end of the maximally right handles of the rsf $\alpha\beta S$ 
(because the rsf has \textit{no} handles and so has no maximally right handles.)

This is an example of a \textit{vacuous truth} that we saw a lot of in first 
year logic.
$$\exists y: rsf(x) \wedge \forall y: (handle(x,y) \wedge maxRight(x,y) \rightarrow \alpha.last < y.last)$$
If $handle(y)$ is false, then lhs of the implies is false for all $y$, 
and so the whole for all is true. 

\frmrule 


\textit{The viable prefix of a terminal rsf can be extended to a valid rsf by appending terminals }

By adding zero or more Ts to the end of a viable prefix (shifting), a handle 
can be constructed and reduced.

\frmrule 

\textit{Valid Splits}

Suppose we have a handle $\beta_i$ and $X \rightarrow \beta_i$. 
Let us split $\beta_i$ into two parts, $\beta_i = \beta_{i}' \beta_{i}''$. 
$X \rightarrow \beta_{i}' \beta_{i}''$. We put a dot inbetween them 
to store this split: $X \rightarrow \beta_{i}' \cdot \beta_{i}''$

We say the split $X \rightarrow \beta_{i}' \cdot \beta_{i}''$ is a \textit{valid split} for viable prefix $\alpha\beta_{i}'$ 
if $S \Rightarrow^{*}_{rm} \alpha X \tau \Rightarrow \alpha \beta_{i}' \beta_{i}'' \tau$.

For a split to be valid for a vaiable prefix, we have that:\\
(i) \textit{there is some rightmost derivation involving that prefix}, \\
this is just by definition of a viable prefix, \\
(ii) \textit{the last step of the rightmost derivation uses the rule seen given the pair} \\
(iii) \textit{the viable prefix stops at the point where the split has the dot}

\frmrule 

\begin{example}
For the following grammar, \\
(a) Give the rightmost derivation of ,\\
(b) For each right sentential form, give the viable prefixes\\
(c) For each viable prefix, give the corresponding \textit{valid split}.
\end{example}

\frmrule 

\begin{example}
For the right sentential form $abcAde$ given by, \\
$S \Rightarrow^{*}_{rm} aAde \Rightarrow_{rm} abcAde$. \\
(a) state the viable prefixes \\
(b) for each viable prefix, find the corresponding \textit{valid splits}.
\end{example}

\frmrule 

\begin{example}
Assume that the $X \rightarrow \delta \cdot$ is a \textit{valid pair} for some viable prefix. \\
Describe the form of the right sentential form that prefix is viable for. 

\begin{itemize}
\item By def. we must have $S \Rightarrow^{*}_{rm} \alpha X \tau \Rightarrow \alpha \delta \tau$. 
The dot of the valid pair tells us that the viable prefix is $\alpha \delta$. That is, the viable prefix 
extends \textit{as far right as possible} (goes all the way up to the rgiht end of handle $\delta$.
\end{itemize}
\end{example}

\frmrule 

\textit{A viable prefix may have several valid splits}



\frmrule 

Notice that valid split + viable prefix = handle. 
Note that a split can be valid for several viable prefixes. And, 
a given viable prefix may have several valid splits. But the pair (valid split, 
viable prefix) uniquely defines the handle of a right sentential form.  

\highlightdef{
Knowing the valid items for a given viable prefix allows
a handle to be found 
}
and by finding handles, we can thus find the reversed rightmost derivation.


\frmrule 

\textit{Valid Pairs}

To reduce the amount of valid splits for a given viable prefix, 
we add more information to valid splits to make them more distinct. 

We say that the tuple $(X \rightarrow \beta_{i}' \cdot \beta_{i}'', x)$ is a \textit{valid pair} for viable prefix $\alpha\beta_{i}'$ 
if $S \Rightarrow^{*}_{rm} \alpha X \tau \Rightarrow \alpha \beta_{i}' \beta_{i}'' \tau$.
and $\textsc{first}_k(\beta_{i}'' \tau) = x$ 


\frmrule 

\begin{example}
Assume that the $(X \rightarrow \delta \cdot , a)$ is a \textit{valid pair} for some viable prefix. \\
Describe the form of the right sentential form that prefix is viable for. 

By def. we must have $S \Rightarrow^{*}_{rm} \alpha X \tau \Rightarrow \alpha \delta \tau$. 
The dot of the valid pair tells us that the viable prefix is $\alpha \delta$. That is, the viable prefix 
is as far right as possible
The $a$ of the valid pair tells us that the sentential form must have $\textsc{first}(\tau) = \{a\}$. 
\end{example}

\frmrule 

The previous example shows that
the pair $(X \rightarrow \delta \cdot , a)$ tells us that 
the reduction $X \rightarrow \delta$ can only happen when 
the next input token, $\textsc{first}(\tau)$, is $a$. 

Suppose the question had instead: $(X \rightarrow \delta \cdot \delta', a)$.
Then such a pair does \textit{not} express that what follows the handle must be an $a$, ($\textsc{first}(\tau) = \{a\}$).
Instead it expresses that $\textsc{first}(\delta'\tau) = \{a\}$. 
Items of the form $(X \rightarrow \delta \cdot, a)$ are known as \textit{complete} items.

\frmrule 



\frmrule 


\frmrule 





\textit{Defining $\textsf{LR}(1)$ in terms of valid pairs }

Recall the formal definition of $\textsf{LR}(1)$:\\
$S \Rightarrow^{*}_{rm} \alpha X x\tau \Rightarrow_{rm} \alpha \beta_i x\tau$\\
$S \Rightarrow^{*}_{rm} \gamma \tau' \Rightarrow_{rm} \alpha \beta_i x\tau''$\\
implies\\
$\alpha X = \gamma$\\
$\tau' = x\tau''$

We can prove that a grammar is $\textsf{LR}(1)$ iff\\
for all viable prefixes,
if set of corresponding valid splits, $I$,
contains a complete item $X \rightarrow \delta \cdot, \{a_1,...,a_n\}$, 
then\\ 
(i) no $a_i$ appears immediately to the right of the dot in any item of $I$\\
(ii) if $B \rightarrow \beta \cdot, \{b_1,...,b_k\}$ is another complete item in $I$, then 
$a_i \neq b_j$ for all $i \in [1,n], j \in [1,k]$. \\

Here $I$ is the image set of $f(\cdot, v) = I$ for a fixed viable prefix $v$. 

\frmrule 

\textit{Valid Splits help to find handles }

Suppose we are trying to find handles in the rsf 
$\alpha \beta_{i}' \beta_{i}'' \tau$

Suppose that $X \rightarrow \beta_{i}' \cdot \beta_{i}''$ is 
the \textit{only} valid split for viable prefix $\alpha\beta_{i}'$, 
then 
\begin{itemize}
\item $\beta_{i}'' = \epsilon$. Then $\beta_{i}'$ is probably the handle. So we reduce 
by $X \rightarrow \beta_{i}'$
\item $\beta_{i}'' \neq \epsilon$. Then we shift
\end{itemize}
If valid prefix $\alpha\beta_{i}'$ has more than one valid split, then 
we use the lookahead symbol to decide (assuming that there are more lookaheads available).
If there are no more lookaheads to help us decide then we fail. 

\frmrule 



% \highlightdef{The set of viable prefixes is a regular language}

% The set of all viable prefixes of any context-free language is itself a regular language. 
% This fact is non-obvious but is important. 

% It means that we can build a DFA to recognise the set of viable prefixes. 


\begin{sidenote}{$\textsf{LL}(k)$ Grammars vs $\textsf{LR}(k)$}
LR parsers are generally \textit{more powerful} than LL parsers. 
This is because LR parsers generally see \textit{more input symbols} 
when deciding which production to take. LR parsers have seen 
a prefix and have $k$-symbol lookahead to choose a reduction.  LL parsers 
only have the lookahead when it does a prediction. 
As a result, the set of parsable LR(k) grammars is a \textit{super}set of LL(k). 
Because LR parsers can see more, they usually 
detect errors \textit{earlier} than LL parsers. 
\end{sidenote}


\section{Recognising Handles II}



valid pair + viable prefix = handle. 


Any sequence of labels along a path in the %LR(0)
automaton, 
starting at its start state, is a \textit{viable prefix};

the set of items reached at the end of that path contains the 
\textit{valid pairs} for that prefix. 


Hence any sequence of labels gives us: valid pair + viable prefix, 
hence gives us a handle. 

\frmrule 

\begin{example}
We want the automaton to accept when we have found a \textit{viable prefix}.
Which states are the accepting states? Explain your answer. 
\end{example}

\frmrule 

Drawing conventions for LR DFAs:
(i) For the purpose of recognizing the set of
viable prefixes, all drawn states are
considered accepting.
(ii) Error transitions and error states are not
drawn.

\frmrule 

\textit{State Closure Rule}

If $A \rightarrow \alpha \cdot B \beta$ is in $I$ then 
$B \rightarrow \cdot \gamma$ is in $I$

\textit{Proof}: \\
We have $A \rightarrow \alpha \cdot B \beta$ is in $I$,  
therefore, $A \rightarrow \alpha \cdot B \beta$ is a valid pair for some 
right sentential form, say $\rho$. \\
To be a valid pair means that we have:
$$S \Rightarrow^{*}_{rm} \rho A \delta \Rightarrow_{rm} \rho\alpha B\beta\delta$$
But we have the rule $B \rightarrow \gamma$. 
If $B$ is the rightmost non-terminal, then an application of the rule gives that:
$$S \Rightarrow^{*}_{rm} \rho\alpha B\beta\delta \Rightarrow_{rm} \rho\alpha \gamma \beta\delta$$
For the right sentential form $\rho\alpha \gamma \beta\delta$, 
we see that $B \rightarrow \cdot \gamma$ is a valid pair for viable prefix $\rho\alpha$

\frmrule 

\textit{Transition Rule}

Let $I,I'$ be the set of valid items for viable prefixes $\rho\alpha$ and $\rho\alpha Y$. \\
If $X \rightarrow \alpha \cdot Y \beta$ is in $I$ then 
$X \rightarrow \alpha Y \cdot \beta $ is in $I'$.

\textit{Proof}: \\
Assume $X \rightarrow \alpha \cdot Y \beta$ is in $I$.  
Then $X \rightarrow \alpha \cdot Y \beta$ is a valid pair for $\rho$. \\
To be a valid pair means that we have:
$$S \Rightarrow^{*}_{rm} \rho X \delta \Rightarrow_{rm} \rho\alpha Y\beta\delta$$

We are given that $\rho Y$ is a viable prefix. 
So there must be some right sentential form:
$$S \Rightarrow^{*}_{rm} \rho Y Z \delta'$$

\frmrule 

% \frmrule 

% No ambiguous grammar is LR(k) for any fixed k

% \frmrule 


% An \textsf{LR}($k$) parser must be able to recognize the occurrence of the right
% hand side of a production after having seen all that is derive
% d from
% that right hand side with
% k
% symbols of lookahead.

% \frmrule 



% \section{Shift-Reduce Parsing}


% \frmrule

% \textit{Introducing Shift-Reduce Parsers}:


% \begin{prooftree}
% \def\defaultHypSeparation{\hskip .01in}
% \AxiomC{}
% \LeftLabel{\textsc{shift}}
% \UnaryInfC{$(\alpha\gamma,a\rho) \leadsto (\alpha\gamma a,\rho)$}
% \end{prooftree}


% \begin{prooftree}
% \def\defaultHypSeparation{\hskip .01in}
% \AxiomC{}
% \LeftLabel{\textsc{reduce}}
% \RightLabel{$X \rightarrow \gamma \in R$}
% \UnaryInfC{$(\alpha\gamma,\rho) \leadsto (\alpha X,\rho)$}
% \end{prooftree}

% \highlightdef{\textbf{Shift}: }
% \highlightdef{\textbf{Reduce}: }
% \highlightdef{\textbf{Shift-Reduce Parser}: a \textit{shift-reduce parser} is an implementation of the handle-pruning 
% algorithm that uses
% uses state $(\sigma, \rho)$ where $\sigma$ is the stack and $\rho$ is unread input}


% \section{Item Push-down Automaton}


% \highlightdef{\textbf{Pushdown Automaton}: Like a FSA but \textit{augmented with a stack} }
% A pushdown automaton is like a FSA (which it be a NFA, DFA). 



% \frmrule 


% \textit{Introducing context-free items. Interpretation of their meaning in context of PDA}



% \frmrule 




% \section{Recognizing Derivation Handles I}




% \highlightdef{\textbf{Viable Prefix}: $\alpha$ is a \textit{viable prefix} iff 
% there is a $\omega$ such that\\ $\alpha\omega$ is a stack of the 
% shift-reduce parser}

% \highlightdef{\textbf{LR(0) Item}: An \textit{LR item} is a rule 
% with a dot ($\cdot$) in some position on the rhs}

% \highlightdef{All the LR items of $X \rightarrow \gamma$ give all possible $(\sigma, \rho)$ 
% where we can apply the \textsc{reduce} rule }

% We didn't have to use a dot. We could've used a bar (|) or a pointer/arrow.
% From literature, dot is the most common choice. This dot is often called the \textit{cursor}.


% \highlightdef{Whatever is on the stack must be the prefix of the rhs of some production 
% (or productions)}



% \section{Recognizing Derivation Handles II}