
\chapter{Space Complexity}



\section{Introducing Space Complexity}

Clearly space is costly as well as time.
But have seen that

\highlightdef{Bound on times implies bound on space}

Putting a bound on the number of steps taken 
by a Turing machine also puts a bound on the 
number of tape squares we can write symbols to. 

So why bother measuring space?

\begin{itemize}   
\renewcommand{\labelitemi}{$\Box$}
\item \textbf{New Classification} 
Space complexity classes have nice closure properties
\item \textbf{Parallel computation} 
Space complexity classes have connection with parallel computation.
sequential logspace $\sim$ parallel logtime
\end{itemize} 

\frmrule



A multi-tape Turing machine has several heads that 
can move independently. This should not be confused with 
a multi-\textit{track} Turing machine that only has one 
head. 

\begin{itemize}   
\renewcommand{\labelitemi}{$\Box$}
\item \textbf{Work Tapes} 
There may be \textit{zero or more} work tapes.
\item \textbf{Input Tape} 
An IO-TM has exactly one \textit{input tape}. 
The head for this tape can move freely, but we 
cannot modify the symbols on the squares of this tape. 
This tape is \textit{read-only}. 
\item \textbf{Output Tape} 
An IO-TM has exactly one \textit{output tape}. 
The head for this tape can \textit{only move right}, and we 
will not use the tape for reading.
The squares of this tape are strictly write only.  
The tape is \textit{write-only}. 
\end{itemize} 


\highlightdef{\textbf{IO k-Tape TM}: $k-2$ work tapes, 1 input tape, 1 output tape}



An IO-TM decides a language $L$ in space $f(n)$ 
iff every input, $w$, of length $|w| = n$, 
the TM uses $f(n)$ squares on each work tape.
The contents of the output tape IO-TM 

Now using this definition, we define $\text{SPACE}(f(x))$.
$L$ is in SPACE($f(n)$) if $L$ is decided by an IOTM operating
within space $f(n)$. That is to say, for any langauge 
\begin{itemize}   
\renewcommand{\labelitemi}{$\Box$}
\item \textbf{Work Tapes} 
There may be \textit{zero or more} work tapes.
\item \textbf{Input Tape} 
An IO-TM has exactly one \textit{input tape}. 
The head for this tape can move freely, but we 
cannot modify the symbols on the squares of this tape. 
This tape is \textit{read-only}. 
\item \textbf{Output Tape} 
An IO-TM has exactly one \textit{output tape}. 
The head for this tape can \textit{only move right}, and we 
will not use the tape for reading.
The squares of this tape are strictly write only.  
The tape is \textit{write-only}. 
\end{itemize} 

The previous definition is the same as 
the definitions of $\text{TIME}(f(x))$ except that now we are 
using a IO-TM instead of the TM definition we gave in the 
computational models chapter. 

\frmrule

A ND-IOTM $M$ computes $F(x)$ in time $f(x)$ iff on input $x$,
\begin{itemize}   
\renewcommand{\labelitemi}{$\Box$}
\item All computations either have $F(x)$ on the output tape
and halts in the \textsc{yes} state \textit{or}
halt the rejecting state in the \textsc{no} state.
\item At least one computation must halt 
in the \textsc{yes} state with $F(x)$ and 
\item For all such computations that do halt 
in the \textsc{yes} state, the $F(x)$ must be unique 
($F(x)$ is a function and has one unique output).
\item The work tapes use space $\leqslant f(|x|)$.  
\end{itemize} 


\frmrule

One question is whether it is reasonble to include the 
input space and the output space as part of the space needed 
to do the work. After all, surely all the work is done on the 
work tape? It doesn't matter whether or not we include 
the input tape space and the output tape space. 
The difference changes by a factor of $k-2$, 
and as with time, we can ignore constant factors. 
The speedup is linear which makes no difference to it 
being a polynomial. 


\highlightdef{\textbf{LOGSPACE}: $\text{LOGSPACE} = \text{SPACE}(\log(n))$ }

We abbreviate \textsc{logspace} by L.

\frmrule

\begin{example}

Languages are often decided by maintaining counters with values within
poly bound in input length n.
Suppose we have $m$ counters whose digits have a 
p-bound $c_i \leqslant p(n)$.
Then the overall space usage $\sum c_i \leqslant m\log(p(n))$. 
Note that we do not always have to put the counters on different work tapes. 
If each counter has a fixed number of bits allocated, we could string them along a single work 
tape. But typically, it is easier to have a counter per tape.
Hence the language is $O(\text{log}(n))$ and so it in \textsc{logspace}. 
The problem associated to this language has a 
logarithmic space complexity.

\end{example}




\section{Time-Space Relationships}


We want to compare space and time usage:
For instance, we want to argue that a bound on time usage implies a
bound on space usage. 

\highlightdef{\textbf{Proper Function}: can be computed 
by an IO-TM within time and space\\ given by the value of the function}
Also called a \textit{proper complexity function}. 

$f:\mathbb{N} \rightarrow \mathbb{N}$ is a \textit{proper function} iff

\begin{itemize}   
\renewcommand{\labelitemi}{$\Box$}
\item \textbf{Nondecreasing} 
$f$ is a nondecreasing function. That is, $f(n+1) \geqslant f(n)$ for all n
\item \textbf{IO-TM} 
There is a $k$-tape IO-TM which for every input, $w$, 
outputs $1^{f(|w|)}$s on the output tape, decides operates within time $O(|w| + f(|w|))$ 
and space $O(f(|w|))$. 
\end{itemize} 


\frmrule

\begin{example}
Show that the addition of two proper functions, $f + g$, is a proper function.

\textbf{(a)} $f$ is nondecreasing so ................... for all n \\
\textbf{(b)} $g$ is nondecreasing so ................... for all n \\
\textbf{(c)} Hence $(f + g)(n+1) = f(n+1) + g(n+1) \geqslant$ .......... $+$ ......... $ = $ ....... for all n. \\
So $f + g$ is nondecreasing. 

Next we need to find a $k$ tape IO-TM which for 
every input, $w$, of length $|w| = n$, outputs \textbf{(d)} ............ 
on the output tape and operates within time \textbf{(e)} ............. 
and space \textbf{(f)} ............. 


\end{example}



\frameans{Complete the proof}
{
\textbf{(d)} $1^{(f+g)(\abs{w})}$. \\
\textbf{(e)} $O(\abs{w} + (f+g)(\abs{w}))$. \\
\textbf{(f)} $O((f+g)(\abs{w}))$ 
}

We already have IO-TMs that compute 

\begin{tikzpicture}
[n/.style={draw, fill=white,inner sep=11pt}]
    \node[n] (m) {$M_f$};
    \node[n, right=1cm of m] (m2) {$M_g$};
    \begin{pgfonlayer}{background} 
        \node[draw, fill=black!10, fit={(m2) (m)}, drop shadow, inner sep=5pt] (m1) {};
    \end{pgfonlayer} 
    \coordinate (in) at ($(m)+(-1.5cm,0.0cm)$);
    \coordinate (out) at ($(m2)+(2.5cm,0.0cm)$);
    \draw[->] (in) -- node[above]{} (m);
    \draw[->] (m) -- node[above]{} (m2);
    \draw[->] (m2) -- node[above]{} (out);

    \node[above of=m1] {$M_{f+g}$};
\end{tikzpicture}

Idea: \\
$M_{f+g}$ is a IO-TM with $(k_f - 2) + (k_g - 2)$ work tapes.

$M_{f+g}$ first runs $M_f$ on input $w$. 
The result is $1^{f(\abs{w})}$ on $M_f$'s output tape.
$M_{f+g}$ first runs $M_g$ on input $w$. 
The result is $1^{g(\abs{w})}$ on $M_g$'s output tape.
$M_{f+g}$ concatenates $M_f$ and $M_g$ uses copying
to copy from $M_f$'s output tape then from $M_g$'s output tape
to produce a concatenation of $1^{f(\abs{w} + g(\abs{w})}$ 
on $M_{f+g}$'s output tape.

This runs in total time $= [O(\abs{w}+f(\abs{w}))] + [O(\abs{w}+g(\abs{w}))]  
+ [f(\abs{w}) + g(\abs{w})]$ which is $O(\abs{w}+f(\abs{w})+g(\abs{w}))$. 
This uses space $= \max{f(\abs{w}),g(\abs{w})} = O(f(\abs{w}) + g(\abs{w}))$. 
Hence $f + g$, is a proper function.

\frmrule



\highlightdef{$\text{TIME}(f(n)) \subseteq \text{SPACE}(f(n))$ }






\frmrule

\highlightdef{$\text{NTIME}(f(n)) \subseteq \text{SPACE}(f(n))$ }

\frmrule

The proof involves simulating all the possible nondeterministic choices 
within space $f(n)$ on a deterministic input-output turing machine
without worrying about the time taken.\\
To show: if $L \in \text{NTIME}(f(n))$ then $L \in \text{SPACE}(f(n))$


\begin{itemize}   
\renewcommand{\labelitemi}{$\Box$}
\item ($\Rightarrow$) Assume $L \in \text{NTIME}(f(n))$ \\
Then there is some non-deterministic turing machine $N$ that 
decides $L$ in polynomial time. 
We will construct a deterministic input-output turing machine $M$ that simulates $N$, 
which we can always do, but here, we will only use space $f(n)$. 

Let $d$ be the degree of nondeterminism of $N$. Recall that this is 
the maximal number of possible choices that can ever be outputted by $N$'s 
transition relation for any state-symbol inputted. 

Any computation of $N$ on input $x$ is a sequene of at most $f(|x|)$ 
nondeterministic choices that we can represent by integers from 0 to $d-1$. 
The input-output turing machine $M$ considers sequences of 
choices $(c_1, c_2, \cdots, c_{f(|x|)})$ ranging from $(0,0, \cdots, 0)$ 
to $(d-1,d-1, \cdots, d-1)$ and simulates $N$ on each. 

Some choices $(c_1, c_2, \cdots, c_{f(|x|)})$ won't make sense but 
we can simply stop trying that sequence and move onto the next one. 
There are an exponential number of simulations to try out, certainly 
at most $(d)^{f(|x|)}$, but the trick is that we only need to 
\textit{store the current one}. The time may be bad, 
but in terms of \textit{space} (what we care about for this proof), 
we only need to store an amount proportional $f(|x|)$.

The input output turing machine describe uses space $\ceil{\log(d)} \times f(|x|)$, 
but since $d$ is independent of $|x|$, we can apply the space linearity theorem 
with $\epsilon = \ceil{\log(d)}$ to get a space usage reduced to $f(|x|)$. 
As $f(n)$ is proper, the first sequence of 
$(0,0, \cdots, 0)$ can generated in space $f(|x|)$. 

If a simulates leads the $N$ to the \textsc{yes} state, then $M$ 
will also halt on the \textsc{yes} state. It is only when all sequences 
are exhausted without finding a sequence where we reject $x$ 
by halting on the \textsc{no} state. 
\end{itemize}

\frmrule 

Now take the previous theorem and replace TIME by SPACE. 

\highlightdef{$\text{NSPACE}(f(n)) \subseteq \text{TIME}(k^{\log(n)+f(n)})$ }

This proof uses the \textit{reachability method}. For 
any language $L \in \text{NSPACE}(f(n))$, we take the non-deterministic 
turing machine that decides $L$ and take all possible configurations that
$N$ can be in and form a directed graph that we call the \textit{configuration graph}.
We will then check to see if we can reach a \textsc{yes} using \textsc{rch}.\\
To show: if $L \in \text{NSPACE}(f(n))$ then $L \in \text{TIME}(k^{\log(n)+f(n)})$

\begin{itemize}   
\renewcommand{\labelitemi}{$\Box$}
\item ($\Rightarrow$) Assume $L \in \text{NSPACE}(f(n))$ \\
We take all possible configurations that
$N$ can be in and form a graph that we call the \textit{configuration graph}.
The configurations will be the nodes and an edge will connect to 
one configurations to another only if the machine in the first configuration 
can go to the second via a non deterministic choice. 

An important point is there are a finite number of configurations 
of a input-output turing machine (as well as all turing machines). 
How many are there \textit{exactly}? Well, each configuration can 
be written in the form $(q, i, w_2\cdot u, )$
There are $n+1$ possible positions for the input head, 

But we are constructing a machine for \textit{deciding} $L$ with bound $f(n)$. 
We are not concerned with the contents of the output tape so we can ignore its 
contents and the head's position. We can also ignore the input tape contents. 
There are no more than $|Q|(n+1)(|\Sigma|^{f(n)} \times f(n))^{k-1}$ configurations.

We form a deterministic turing machine $M$ that performs \textsc{rch} using 
an input of size $n = c^{\log(n) + f(n)}$ nodes. 
This takes time $f'(n) = (c^{\log(n)+f(n)})^{2} = k^{\log(n)+f(n)}$ 
where $k = c^2$ is a constant independent of the input size. 
Hence $M$ decides $L$ in time $k^{\log(n)+f(n)}$, 
so $L$ is in $\text{TIME}(k^{\log(n)+f(n)})$.
\end{itemize}



\section{Closure Under Complement}



\highlightdef{\textbf{Szelepscenyi-Immerman Theorem}: The number of 
nodes reachable \\from $x$ in $G$ can be computed by a NDIOTM within 
space $\log(\text{no. nodes})$}
