\chapter{TCP/IP: Advanced}


\section{Network Adapters}


CSMA/CD

\highlightdef{\textbf{MAC}: Network Adapters perform \textit{Medium Access Control} (MAC) }



\frmrule

\textit{The difference between a hub and a switch}




\section{Bijection: IP and MAC Addresses}

ARP and Ping

\frmrule 

\textit{The format of a ARP protocol data unit}

The address resolution protocol, when used for 
MAC-to-IP address resolution has PDUs of the following high-level format: 
\highlightdef{\textbf{ARP}: $[c | op | M | N]$}

Where:
\begin{itemize}[nosep]
\item $c$ is a constant fixed for MAC-to-IP address resolution
\item $op \in \{\textsf{Request}, \textsf{Reply} \}$ denotes 
the operation.
\item $sN$ is the source IP address 
\item $dN$ is the destination IP address
\end{itemize}

\frmrule 

\textit{Encapsulation of a ARP protocol data unit}


$D = [c | op | sN | dN]$ in ethernal pdu: $[p|sM|dM|t|e|D]$


Network Layer: ARP, IP
Datalink layer: Ethernet 

\frmrule 

\textit{ARP Request and ARP Reply}



$k(i,j)$ - machine $i$ knows machine $j$'s MAC address
This relation is reflexive, we assume that machine every machine knows its own MAC address.

if($v_B$ outside network) \\
$\mu_{R.0}$ = localAsk($\mathcal{R}.0$) \\
send ($\mu_{A}$, $\mu_{R.0}$, $\nu_{A}$, $\nu_{B}$, \textsf{request})\\
($\mu_{A}$, $\mu_{R.0}$, $\nu_{A}$, $\nu_{B}$, \textsf{request}) = \\
return $\mu_{B}$\\
else \\
$\mu_{B}$ = localAsk($\mathcal{B}$) \\
return $\mu_{B}$\\

\frmrule 

\begin{example}
Suppose we have two subnets $\mathcal{N}_1$: 146.0.4.x and $\mathcal{N}_2$: 146.0.1.x. 
We have a host on $\mathcal{N}_1$ who wants to find the MAC address of a host 
on $\mathcal{N}_2$. 

Complete the table.
\begin{table}[h]
    \centering
    \begin{tabular}{ll|llllll}
    & &    & $\textsf{sM}$ & $\textsf{dM}$  & $\textsf{sN}$ & $\textsf{dN}$ & $\textsf{op}$ \\ \hline
    A. Find $\mu_B$& A.1 Ask $\mu_{R.1}$ & $\mathcal{A} \rightarrow \mathcal{R}.1$  & $\mu_{A}$ & $ \mu_{*} $ & $\nu_{A}$ & $\nu_{R.1}$ & \textsf{request}  \\
    & & $\mathcal{A} \leftarrow \mathcal{R}.1$   & $\mu_{R.1}$ & $\mu_{A}$ & $\nu_{R.1}$ & $\nu_{A}$ & \textsf{reply} \\ 
    & A.2 Ask $\mu_B$ & $\mathcal{A} \rightarrow \mathcal{B}$   & $\mu_{A}$ & $\mu_{R.1}$ & $\nu_{A}$ & $\nu_{B}$ & \textsf{request}  \\ \hline
    R. Find $\mu_B$ & R.1 Ask $\mu_B$ & $\mathcal{R}.2 \rightarrow \mathcal{B}$   & $\mu_{R.2}$ & $\mu_{*}$ & $\nu_{R.2}$ & $\nu_{B}$ & \textsf{request}  \\
    & & $\mathcal{R}.2 \leftarrow \mathcal{B}$   & $\mu_{B}$ & $\mu_{R.2}$ & $\nu_{B}$ & $\nu_{R.2}$ & \textsf{reply}  \\ \hline
    &  & $\mathcal{A} \rightarrow \mathcal{B}$   & $\mu_{R.2}$ & $\mu_{B}$ & $\nu_{A}$ & $\nu_{B}$ & \textsf{request}  \\ 
    &  & $\mathcal{A} \leftarrow \mathcal{B}$   & $\mu_{B}$ & $\mu_{R.2}$ & $\nu_{B}$ & $\nu_{A}$ & \textsf{reply}  \\
    &  & $\mathcal{A} \leftarrow \mathcal{B}$   & $\mu_{R.1}$ & $\mu_{A}$ & $\nu_{B}$ & $\nu_{A}$ & \textsf{reply}  \\
    \end{tabular}
\end{table}



\end{example}




\frmrule 

\textit{ARP Caches}


\begin{sidenote}{ARP Cache}
/sbin/arp -n lists the contents of the ARP cache\\
\end{sidenote}


\frmrule 

\textit{ARP Format}

\begin{sidenote}{ARP Format}
ARP has the following format.
\end{sidenote}





\section{Token Rings}


\section{Binary Exponential Backoffs}


\section{WiFi}



\section{Errors I}


\frmrule

\section{Errors III }

