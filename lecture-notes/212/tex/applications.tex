\chapter{Applications}


\section{DNS I}


\frmrule 

\textit{Motivation for a directory service}

\frmrule 

\textit{Problems with a flat file}

Until 1984 there was one central file 
called hosts.txt that had a flat structure. 
It was simply a list of pairs $[(n,\nu)]$ 
that everybody could download it and update locally. 
To update it globally, an email would be 
sent to a central file administrator. 

This flat file approach has problems with 
with coordination, updating, access. The names used
lacked a tree structure whereas IP addresses 
do have a tree structure.
After 1984 a new system was implemented called 
the \textit{Domain Name Service} (DNS) that would 
solve the issues of using the flat file structure. 

\frmrule 

\textit{Namespace tree.}


\frmrule 

\textit{Using URNs to uniquely name tree nodes.}


\frmrule 

\textit{Domains}

A domain is a subtree in the namspace tree.
Recall that a subtree is uniquely identified by 
a single namespace node or by its URN. 

\highlightdef{\textbf{Domain}: a \textit{domain} is a \textit{subtree} $T'$ of the namespace tree $T$}

\frmrule 


\highlightdef{
\textbf{Domain name}: the name $n$ that uniquely identifies a domain $T'$:
$\langle label_n \rangle \cdot ... \cdot \langle label_2 \rangle \cdot \langle label_1 \rangle$ 
}

To uniquely identify a domain, $T'$ we use there is a unique path from the 
root of $T'$ to the root of the namespace tree, $v_1 \rightarrow ... \rightarrow v_n$, where 
$v_1 = root(T')$ and $v_n = root(T)$. 
Recall that for each node, we associate a label, $n$, unique for that level: $L(v) = n$. 
Hence the path is uniquely 
identified by the string: $l(v_1) \rightarrow ... \rightarrow l(v_n)$ 
which equals $label_1 \rightarrow ... \rightarrow label_n$. 
We often \textit{replace the arrows with dots}. 
The result is string of labels that uniquely identifies the path (and thus the domain). 
It is this string that is our \textit{domain name}: 
$\langle label_n \rangle \cdot ... \cdot \langle label_2 \rangle \cdot \langle label_1 \rangle$.

There are two uses of the term \textit{domain}. \\
\textbf{Use 1} The term \textit{domain} is used 
to refer to all the nodes $v \in T$ or all the names of 
the nodes in $T$, $L(v), v \in T$. \\
\textbf{Use 2} Sometimes, the term \textit{domain} is used 
to refer to a \textit{single name}. This name may refer to the 
label assigned to the root of the tree $l(root(T))$ or the 
name associated with the root of the tree $L(root(T))$.

By convention, we $l(root(T)) = \epsilon$, the \textit{empty string}.
As a result, all domain names will be of the form: 
$\langle label_n \rangle \cdot ... \cdot \langle label_2 \rangle \cdot$, 
and will \textit{end with a dot}.  

\frmrule

\textit{Domains vs Subdomains}

Technically   speaking,  every  node  in  the  DNS  tree  is  a
``domain'', even if it is ``terminal'', that is,  has  no
``subdomains''.  Technically speaking, every subdomain is
a domain and every domain except the root is also a sub-
domain.   
   
The  terminology is not intuitive and you would do well to read 
RFC's 1033, 1034, and 1035 to gain a complete understanding of 
this difficult and subtle topic.


\frmrule 

\begin{example}
\end{example}

\frmrule 


\textit{Partially Qualified vs Fully Qualified}

Sometimes we would like to identify where a subtree $T''$ is 
that is nested inside some subtree $T' \neq root(T)$. 
To do this, we can use a \textit{partially qualified} domain name. 

Relative to $T'$, there is a unique path to $T''$, 
$v' \rightarrow ... \rightarrow v''$. Recall that
for each level of the namespace tree, 
we associate a label, $n$, unique for each node in that level: $l(v) = n$. 
Hence the path is uniquely 
identified by the string: $l(v') \rightarrow ... \rightarrow l(v'')$ 
which equals $label_1 \rightarrow ... \rightarrow label_n$. 
Again we \textit{replace the arrows with dots}. 
This string that is a \textit{partially qualified} domain name. 

Partially qualified domain names identify unique paths from $v'' = root(T'')$ up to $v' = root(T') \neq root(T)$, 
where as \textit{fully qualified} domain names (or just \textit{domain names}) 
identify unique paths from $v' = root(T')$ up to $v = root(T)$. 

\frmrule 


\frmrule 

Notice that with the convention that $L(root(T)) = \epsilon$, 
it is quite easy to see that 


\highlightdef{
Partially qualified \textit{never} end in a dot, \\
Fully qualified \textit{always} end in a dot
}



\begin{example}
\end{example}

\frmrule 


\textit{Zones}

We partition the nodes of the namespace tree into \textit{zones}. 
if two nodes are in the same zone, we require that there is a unique 
path connecting them - \textit{and} that every node along 
such a path is in the zone. Notice that this constaint tells 
us that \textit{every zone is a tree}. More formally:



\highlightdef{
\textbf{Zone Partition}: a partition $\mathcal{Z}$ of a namespace nodes $N$ \\
where each block, or \textit{zone}, $Z \in \mathcal{Z}$ satisfies:\\
(i) $n,n' \in Z$ implies unique path from $n$ to $n'$\\
(ii) every node along this unique path is in $Z$
}

A \textit{zone partition} is a normal set partition, except that we add a further 
constraint. Here, by parititioning a set of nodes, 
we have that block is a set of nodes. Our constraint is that 
each block is a \textit{tree}. We split the namespace tree into blocks, and require 
that each block is, by itself, a tree.
A \textit{zone} is simple a \textit{block} in such a partition. 
\highlightdef{
\textbf{Zone}: a block of a namespace node set \textit{zone partition}
}
The point is that taking just any old partition of the namespace tree may not be acceptable. 
We need to make sure that each block in the partition is itself a \textit{tree}. 
We will later see that this constraint means that delegating names 
to servers becomes much easier.

\begin{sidenote}{Set Partitions} 
Recall from Discrete Maths that, a \textut{partition} $\mathcal{P}$ of a finite set $S$ is a set of sets 
that satisfies the following:\\
(i) $X \in \mathcal{P}$ implies $X \neq \emptyset$ (\textit{no block is empty})\\
(ii) $X,Y \in \mathcal{P}$ and $X \neq Y$ implies $X \cap Y = \emptyset$ (\textit{the blocks don't overlap})\\
(ii) $\cup_{X \in \mathcal{P}} = S$ (\textit{the blocks join to make $S$})
\end{sidenote}

\frmrule 

\begin{example}
For the following namespace tree, explain whether the following partitions 
are \textit{zone partitions}. \\
(a) $\mathcal{Z} = \{\{a,b,c\},\{d,e\},\{f\}\}$
\end{example}


\frmrule

\textit{More about zones}

Notice that the space occupied by zones is \textit{contiguous} in some sense. 
We cannot have a zone that occupies bits dotted around the namespace 
tree. Also note that if a zone has two nodes, then the 
least common ancestor must also be in the zone. 
Note that for each namespace node, there is exactly 
one zone it belongs to. 

\begin{sidenote}{Equivalence Relations} 
Recall from Discrete Maths that for any partition, 
there is a corresponding \textit{equivalence relation}.
\end{sidenote}

Using the idea of an equivalence relation, we define the root zone.
Here our equivalence relation is the one that corresponds to the 
zone partition.

\highlightdef{\textbf{Root Zone}: the \textit{root zone} is the zone $Z$ such that $Z = [root(T)]$}
where $[v]$ denotes the \textit{equivalence class} of $v$. 



\frmrule 

\textit{Domain name of a zone}

Because each zone is a tree, there is a root node of the zone $root(Z)$. 
But this is just a namespace node that belongs to the 
original namespace tree. This means that $root(Z)$ actually 
identifies a \textit{subtree} of the namespace tree.
And so, it can be uniquely 
identified by a fully qualified domain name. As a result, 
\textit{we can uniquely identify zones by domain names}. 

Assume that the namespace tree has been partitioned by some zone partition $\mathcal{Z}$, then
we define a \textit{zone domain name} by the following.
\highlightdef{
\textbf{Zone Domain Name}: 
the string that uniquely identifies a zone $Z$:
$\langle label_n \rangle \cdot ... \cdot \langle label_2 \rangle \cdot \langle label_1 \rangle$ \\
where $L(root(Z)) = label_n$
}
we write $dn(Z)$ to denote this string

\frmrule 

\begin{example}
\end{example}

\frmrule

\textit{Domains vs zones}

Now that we have introduced the idea of \textit{domains} and \textit{zones}, 
let's think about the relationship between the two.

\textbf{Fact 1}: Every node of a zone has a subtree associated. 
Thus every node of a zone corresponds to a domain. 
So in a way, because a zone has many nodes, 
it has many domains associated. 

\textbf{Fact 2}: A given domain may have nodes that belong to one 
or more zones. So a domain may have several zones associated.


\frmrule

\textit{Delegates}

Because the zones form a partition, then we know from Discrete Maths that 
there is a corresponding equivalence relation. 
But for zone paritions, we like to define \textit{another} relation 
called the \textit{delegation relation}.
The \textit{delegation relation} is is a 
relation $D \subseteq Z \times Z$ chosen by 
us, the namespace manager.  
This relation has a constraint. If 
we have $(Z_1,Z_2) \in D$ then we require that
all the namespace nodes in $Z_2$ are children of namespace nodes in $Z_1$. 
That is for each $n \in Z_1$, there is a unique path to every $n' \in Z_2$. 

\highlightdef{\textbf{Delegate Relation}: $(Z_1,Z_2) \in D$ iff \\
for each $n \in Z_1$, there is a unique path to every $n' \in Z_2$ }

This constraint forces the delegations relation to itself \textit{form a tree}. 
Thus can call the delegation relation, a \textit{delegation tree}, with nodes $Z$ and 
edges $D$. The zone at the root of the delegation tree $(Z,D)$ is called the \textit{root zone}.  


\frmrule 


\begin{example}
For the following namespace tree\\
(a) show that the given following partitions is a \textit{zone partitions}. \\
(b) find the underlying delegation relation\\
(c) state which zone is the root zone\\
\end{example}


\frmrule 

\textit{More about delegations}


\highlightdef{For any given zone partition, there is \textit{exactly one} delegate relation }


\highlightdef{\textbf{Delegation Edge}: Each $(Z_1,Z_2) \in D$ corresponds to edge an $e = (v_1,v_2) \in E$ 
of the namespace tree: $(V,E)$.
These edges, $(v_1,v_2)$, are called \textit{delegation edges}.  }

We can prove this via the following argument:\\

Recall the definition of a well-defined function. 
There is a function from $(Z_1,Z_2)$ to $(root(Z_1),root(Z_2))$. \\
There is a function from $(root(Z_1),root(Z_2))$ to $([root(Z_1)],[root(Z_2)])$. \\
Hence there is a function from $(Z_1,Z_2)$ to $([root(Z_1)],[root(Z_2)]) = (v_1,v_2)$. \\
And so we have shown that each pair in the delegate relation $(Z_1,Z_2) \in D$ 
there is a unique output edge in the namespace tree $e = (v_1,v_2) \in E$. 


\frmrule 

\highlightdef{A set of authoritative name servers has to be assigned for every DNS zone. }

We add NS (name server) resource records. 
We add one NS record for each name server authoritative for our zone.


% \textit{Introducing Lookups}
%
% \frmrule
%
% \textit{Primary servers and secondary servers}
%
% Each zone $Z$ has a non-empty set of nameservers $S = servers(Z)$.
% The servers are responsible for maintaining the resource record as well as
% handling requests to read the record.
%
% One server $s_P \in S$ is chosen to be the \textit{primary server}.
% All the remaining servers are in $S - \{S_P\}$ and are called \textit{secondary servers}.
%
% \highlightdef{Each zone $Z$ has exactly one server chosen to be the
% primany server, $server$($Z$)}
%
% Note that two distinct zones $Z_1$ and $Z_2$ may have overlapping server sets:
% $servers(Z_1) \cap servers(Z_2) \neq \emptyset$. Furthermore,
% they may have identical primary servers $server(Z_1) = server(Z_2)$. It may also be the case
% that
% a single server can be a primary
% server for $Z_1$ and and a secondary server
% for another zone $Z_2$.
%
% \highlightdef{For each zone $Z$, $server$($Z$) stores the IP address
% of each server $S$ such that (i) $(Z,Z') \in D$ (ii) $server(Z') = S$}
%
%
% \frmrule
%
% \textit{Iterative Lookups}
%
% $Ask(dn_1,S_1) \rightarrow Ask(dn_2, S_2)$ \\
% when $dn(v) = dn_1$ and $[v] = Z_1$  \\
% and $(Z_1,Z_2) \in D$ and $dn(root(Z_2)) = dn_2$ \\
% and $S_2 = server(Z_2)$ \\
%
% \frmrule
%
% \begin{example}
% Given the following namespace tree,  \\
% \\
% with zones and delegations:
% $Z = \{\{a,b,c\},\ \{d,e,f,g\},\ \{h,i,j\}, \{j,k,l,m,n\}\}$\\
% $D = \{(Z_1, Z_2),(Z_1, Z_3), (Z_3, Z_4)}$\\
% Show how foo.bar.gat.mic is resolved using an iterative lookup.
% \end{example}
%
% \frmrule
%
% \highlightdef{
% $S_1 = server(Z_1)$
% delegates a DNS query to $S_2 = server(Z_2)$ when
% $(Z_1, Z_2) \in D$
% }
% we send back to the DNS client the domain name of $Z_2$
% The DNS will return a list.
% For each $(Z_1, Z_2)$ there will be a server
% $server(Z_2)$ in the returned list.

%
% \highlightdef{We assume we know $\nu_R$, the IP address of a root server }
%
% The idea is that:\\
% (i) we follow the delegation edges $root(Z_{i}) \rightarrow root(Z_{i+1})$ \\
% (ii) $root(Z_{n}) \rightarrow n$
%
%
% \highlightdef{
% Base case: $n = N(v',v)\cdot N(v,root(T))$ where $v' \in Z$\\
% Recursive case: $n = s \cdot N(v',v)\cdot N(v,root(T))$ where $v' \in leaves(Z)$
% }
% where $s$ is a nonempty string and $\cdot$ here denotes concatenation
% (not to be confused with the dots in the domain name).
%
% % let $U$ be the set $\{n' \;|\; (n,n') \in E, n \in Z\}$.
% % These are the nodes at the end of the delegate edges for the zone.
%
% Each server $S_i$ stores its own IP address, $\nu_i$ as well as three partial functions. \\
% ip addresses for nodes inside zone: $Z_{\nu} : Z \rightarrow IP_z$\\
% ip addresses of delegates: $D_{\nu} : leaves(Z_i) \rightarrow IP_U$\\
% the domain names of delegates: $D_i : leaves(Z_i) \rightarrow U$\\
%
% DNS uses the services of the transport layer.
% For each query, we use a UDP application-to-application connection.
%
% $$\frac{\textsf{receiveDatagram}([n,...])}{\textsf{receiveQuery}(n)}$$
%
% $$\frac{\textsf{sendQuery}(\nu,n)}{\textsf{sendDatagram}([\nu,n,...])}$$
%
%
%
% Each server follows:\\
% \textsf{receiveQuery}($n$)\\
% ~~~case $n=\mathcal{N}(v',v)\cdot \mathcal{N}(v,root(T))$, $v' \in Z$:  $= Z_{\nu} (v')$   \\
% ~~~case $n = s \cdot \mathcal{N}(v',v)\cdot \mathcal{N}(v,root(T))$, $v' \in leaves(Z)$:  \textsf{sendQuery}($D_{\nu}(v')$,$D_{i}(v')$)  \\
% ~~~otherwise fail
%
%
% \frmrule
%
% \begin{example}
% Show how to resolve the domain name \textsf{a.b1.c.} to an IP address. \\
% You are given the IP address $\nu_R$ of a server that manages the root zone.
%
% \begin{itemize}[nosep]
% \renewcommand{\labelitemi}{$\Box$}
% \item We use $\textsf{sendQuery}(\nu_R,\textsf{a.b1.c.})$.
% \item $S_R$: does not know the answer, but knows delegate $v_2$ with FQDN $n_2$ and ip $\nu_2$.
% $\textsf{sendQuery}(\nu_2,n_2)$
% \item $S_2$: does not know the answer, but knows delegate $v_3$ with FQDN $n_3$ and ip $\nu_3$.
% $\textsf{sendQuery}(\nu_3,n_3)$
% \end{itemize}
%
% \end{example}
%
% \frmrule
%
% \begin{example}
%
% \end{example}
%
% \frmrule
%
%
% \textit{A Records}
%
% $(z_1,z_2) \in D$ means that $z_2$ was created
% by $z_1$ to take the delegated responsibility.
%
% Each zone is responsible for mapping every domain names in $n \in Z$
% an IP address, $\nu$. To do this, each zone $z_i$ stores \textsc{A} \textit{records}.
% An \textsc{A}-record is simply a mapping of a namespace string to an IP address: $n \mapsto \nu$
%
%
% Zones are sometimes called \textit{administrative authorities} following
% this idea of delegation.
%
% \frmrule
%
% \textit{Recursive Lookup vs Iterative Lookup}
%
%
% Recursive lookups should remind you of making a recursive function call
% in a programming language.
%
% $ask(ln \cdot...\cdot l1)=$\\
% $l$
% ~$\quad\quad | ln is root = ask(server(l_{n-1})$\\
% ~$\quad\quad | otherwise = ask(parent)$
%
% Note that this recursive procedure is actually \textit{tail recursive}. And
% so we can implement this function using an accumulative parameter.
%
% \frmrule
%
% \textit{Zone Records}
%
% \highlightdef{\textbf{Zone Record}: A list of resource records $[[\textsf{zdn},\textsf{t},\textsf{ttl}||\textsf{v}]]$\\
% For each namespace node $n \in Z$ there is a record in the list. }
%
% where:
% \begin{itemize}[nosep]
% \renewcommand{\labelitemi}{$\Box$}
% \item \textsf{zdn}: the \textit{zone domain name}. That is, the domain name of the namespace node at root of the zone.
% \item \textsf{t}: the \textit{type} - various types of records can be stored. $\mathsf{t} \in \{\textsc{A},\textsc{NS}\}$.
% \item \textsf{ttl}: the \textit{time to live} - the number of seconds for which the RR may be cached. After this time
% expires, the record is considered to be \textit{out-of-date}. We will look later at how to handle out-of-date records.
% \item \textsf{v}: - the \textit{value} - the actual data to be stored in the record
% \end{itemize}
%
%
%
%

%
%
%
%
% \frmrule
%
%
% \highlightdef{
% A DNS server can only match \textit{fully-qualified} domain names to IP addresses
% }
%
% \frmrule
%
% \begin{example}
% For the following namespace tree with the zones partition shown,
% give the DNS resource records for each zone.
% \end{example}
%
% \frmrule



\section{DNS II}


\frmrule 

\textit{DNS messages}

\highlightdef{
    \textbf{DNS}:\\
    \textit{query} $[H||Q]$\\
    \textit{response} $[H||Q,A,...]$\\
}

\frmrule 

\textit{Using dig to inspect DNS messages}


\frmrule 

\textit{Generic domains and country domains}

The namespace tree has the subtrees
given by the immediate child nodes 
split into two disjoint sets $\mathcal{G}$ and $\mathcal{C}$. 
The subtrees in $\mathcal{G}$ are called generic domains 
and the subtrees in $\mathcal{C}$ are called country domains. 


\frmrule 

\textit{ARP Ping and DNS}


\frmrule 

\textit{DNS uses UDP and not TCP}

What is means is that if a DNS packet is lost, there is no automatic recover. 
At first it may seem like this causes a problem. 


\frmrule

In addition to being subject to loss, DNS packets have a maximum length,
potentially as low as 576 bytes. What happens when a DNS name to be looked
up exceeds this length? Can it be sent in two packets?

\frmrule

Can a machine with a single DNS name have multiple IP addresses? How
could this occur?

\frmrule

Can a computer have two DNS names that fall in different top-level domains?
If so, give a plausible example. If not, explain why not.


\section{Websites: HTTP I}


\highlightdef{The main purpose of HTTP is to provide access to Web objects}
It is extremely popular because of \textit{web browsing}. 
But strictly speaking, HTTP has more 
functions besides \textit{reading} (downloading) web objects. 
HTTP can perform uploading. And HTTP doesn't necessarily have 
to be used for webpages. 

\frmrule 

\highlightdef{HTTP has nothing to do with how a webpage is rendered}

\frmrule 

\begin{sidenote}{Browser Rendering}
Every browser \textit{behaves the same} at the initial stage of 
using HTTP to request a webpage.
But there are issues when it comes to the next stage: \textit{rendering} webpages.  
There are separate specifications that define how a webpage should 
be rendered given its HTML, however issues occur when a browser's 
implementaton doesn't follow specification.
We end up with \textit{different behaviour} at the browser rendering stage.
This has caused a struggle between website content makers and the browsers available. 

For those interested, see: HTML spec CSS spec. 
\end{sidenote}

\frmrule 

\textit{HTTP uses uniquely identifying+locating URIs to \textit{access web objects}}


\frmrule 

\textit{HTTP messages}

We define $G = (\mathsf{connection} | \mathsf{date} | \mathsf{cachectrl} | ... )$, 
as fields that can appear anywhere, in either a request or a reply. 
These fields are called \textit{general fields}. All other fields must appear in a specific message type. 
We also define:\\
$E = (\mathsf{content.type} | \mathsf{content.enc} | \mathsf{content.len} 
| \mathsf{content.lang} | \mathsf{expdate} | \mathsf{moddate} | ...)$ \\
These fields are called \textit{entity fields} 
and can appear in either post requests, put requests or in 
replies. Entity fields are allow the owner of the web content (whether it be 
the client or the server) to describe the data that is about to be sent.

We finally define request fields $Q$ and reply fields $P$, 
fields that can appear only in re\textit{q}uest messages,
and in re\textit{p}ly messages respectively. These are defined here as: \\
$Q = (\mathsf{host} | \mathsf{userag} | \mathsf{ref} | \mathsf{userag} | \mathsf{acc.charset} 
| \mathsf{acc.enc} | \mathsf{acc.lang}  | ... | G)$\\
$P = (\mathsf{server} | ...)$

Then HTTP requests have the format:
\highlightdef{
\textbf{HTTP}: \\
\textit{get requests}: $[\textsc{get}, \textsf{uri}, Q^{*}]$\\
\textit{post,put request}: $[(\textsc{post}|\textsc{put}), \textsf{uri}, (E|Q)^{*},\textsf{content}]$\\
\textit{reply}: $[\textsf{statuscode},\textsf{statusdesc},(E|P)^{*},\textsf{content}]$
}


where
\begin{itemize}[nosep]
\renewcommand{\labelitemi}{$\Box$}
\item $\mathsf{uri}$: the URI of the web object
\item $\mathsf{statuscode}$: the \textit{status code}. Describes the result of the reply. 
$\mathsf{statuscode} \in \{200,404, .. \} $. 200 indicates that the reply was 
successful. 404 indicates that the web object cannot be found.
\item $\mathsf{statusdesc}$: a short description for the status code, $\mathsf{statuscode}$.
\item $\mathsf{content}$: the \textit{content} of the web object. 
The $\mathsf{content}$ field is often called the \textit{body}.
\end{itemize}

\frmrule 

\begin{example}
Below shows a browser requesting a webpage using HTTP. \\
(a) State which fields are in general $G$, entity $E$, request $Q$, reply $P$\\
(b) Explain how the client knows the request was successful. 
\end{example}

\frmrule 

\textit{Steps involved in visiting a URL}

The question is popular as an interview question 
to test candidates understanding of the TCP/IP stack. 

The browser acts as the client application. \\
A program running on the web server acts as the server application. 
\begin{itemize}[nosep]
\renewcommand{\labelitemi}{$\Box$}
\item The \textit{server} runs and listens on port $\rho_S = 80 \in \mathcal{W}$
\item The \textit{client} chooses some ephemeral port $\rho_C \in \mathcal{D}$
\item The \textit{client} attempts to initiate a TCP connection with the server. \\
The client connects to the server using the socket $(\nu_S,\rho_S) = (DNS(uri),80)$ 
\item The \textit{server} accepts the TCP connection from incoming socket $(\nu_C,\rho_C)$. 
\item An application-to-application connection (or \textit{flow}) is setup: $(\nu_C, \rho_C, \nu_S, \rho_S)$.
\item The \textit{client} sends a HTTP request, $[Q | \mathsf{t},\mathsf{u}, ...]$
with $\mathsf{t} = \textsc{get}$ (a \textit{get-request}), 
with $\mathsf{u} = uri$ and $Q$ containing additional request parameters. 
\item The \textit{server} sends a HTTP reply, $[P | \mathsf{t},\mathsf{u}, ...]$
with $\mathsf{t} = \textsc{get}$ (a \textit{get-request}), 
with $\mathsf{u} = uri$
\item The TCP connection is teared down.
\end{itemize}

Notice that we use the services of DNS to translate our URI to an IP address, 
$\nu_S = DNS(uri)$. Everytime we visit a webpage, we rely on a DNS translation 
to find $\nu_S$, the IP address of the server. Humans find web URIs easier 
work with and memorise. Quite clearly, web browsing 
would be a less enjoyable experience if we browsed pages by typing IP addresses 
in the address bar.


\frmrule 

\textit{Rendering a webpage can lead to more HTTP requests}

\frmrule 


\textit{Persistent connections}

The first version of HTTP used one TCP connection per web object. 
As we have seen, downloading webpage can causes a burst 
of other HTTP requests to be made. Each time, we need 
to open a TCP connection, transfer, then tear down. 
This causes an inefficient use of the network and operating system resources. 

HTTP/1.1 introduces \textit{persistent connections}. 
This allowed one TCP connection to be established for all 
the web objects to be transferred for the web page.
It allows the client to send multiple requests, and for the server to 
send multiple replies, returning \textit{multiple web objects}, 
all in \textit{one} TCP session. 

\highlightdef{\textbf{Persistent Connection}:
Using \textit{one TCP connection} for \textit{multiple web objects}
}

The HTTP/1.1 specification states that persistent connections 
are to be used by default. In order to turn it off, 
both the client and server must agree to this. 
Both client and server add the general field
$\mathsf{connection} = \textsc{close}$ to their messages.
After one exchange, of HTTP reply and request, both are 
of mutual understanding that no further transfers should 
occur and so the TCP connection will be teared down. 

\frmrule 

\textit{Caching data to improve performance}

\frmrule 

\textit{HTTP referrers}


\begin{sidenote}{Referrer vs Referer}
(Wikipedia)
The misspelling referer (one-\textit{r}, not two \textit{r}'s) originated in 
the original proposal by computer scientist Philip Hallam-Baker to incorporate 
the field into the HTTP specification. 
The misspelling was set in stone when it was incorporated into RFC 1945.  Since 
then, \textit{Referer} has since become a widely used spelling in the industry when discussing 
the \textsf{ref} field. 
In PHP, for example, we can access \textsf{ref} via \lstinline{\$_SERVER['HTTP_REFERER']}. (notice 
the spelling: one-\textit{r})
\end{sidenote}

\frmrule 

\textit{Using telnet to look at HTTP messages}


\frmrule 

\textit{Relevant RFCs}

\begin{sidenote}{Relevant RFCs}

Pretty much all that we covered about HTTP is specified in HTTP/1.1 specification: RFC 2068 (February 1997)

We looked at he ideas of stateful sessions, cookies, set-cookie, and cookie 
headers. These are specified in: \textit{HTTP State Management Mechanism}: RFC 2109.
\end{sidenote}




\section{Websites: HTTP II}


\frmrule 

\textit{Javascript}

\frmrule 

\textit{Common Gateway Interfaces}

\frmrule 

\textit{HTTP is stateless}

the behavior (semantics) of an HTTP request does not depend on
any previous request

\frmrule 

\textit{Implementing stateful sessions using cookies}

the behavior (semantics) of an HTTP request does not depend on
any previous request


\begin{sidenote}{Cookies and Javascript}
Most websites use Javascript to read and write cookies.
This is done using the \lstinline{document.cookie} object.
\end{sidenote}




\section{Emails: SMTP I}

Mail User Agents (MUAs) are programs like Microsoft Outlook, 

\highlightdef{\textbf{Mail User Agents}: Email clients are formally called 
\textit{Mail User Agents} }

Mail User Agents are sometimes called \textit{User Agents}. 
But not all user agents are mail clients. 
The term \textit{user agent} can refer to any 
piece of software that provides a nice interface to operate 
over protocols like HTTP, FTP and SMTP. 

\section{Emails: POP3}