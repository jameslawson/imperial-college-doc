
\chapter{Communication Models}



\section{Open Systems and Standards}




\section{URIs, URLs and URNs}



\section{OSI Model: Introducing}


\frmrule 

\textit{Introducing the idea of converting a blackbox into layers}

\frmrule 

\textit{Introducing the OSI model}

\highlightdef{OSI = Model. ISO = standard}



\frmrule 

\textit{Motivation for having a layered architecture}

Modularity:
Each layer can be built independently of other layers.\\
May replace the implementation of just one layer without affecting 
the others. 

Abstraction: 


\frmrule 

\textit{Main disadvantage of a layered architecture}

A given subtask repeated in more than one layer. 
This causes confusing. There may be a higher communication 
cost. This decreases efficiency. 

A layered architecture therefore makes a compromise between:\\
more layers to increase the modularity\\
fewer layers so as to improves the efficiency\\


\frmrule 

\textit{Service Oriented Architectures}

Each layer may provide one or more \textit{service elements}.\\
A service element has one or more \textit{service primitives}.

\highlightdef{\textbf{Syntax}: $\langle layer \rangle.\langle service element \rangle.\langle service primitive \rangle$}


\frmrule 



\begin{example}
A simple example of a layer implementing a \textit{connectionless transport} layer, \textsf{T1}. \\
This layer provides the service element \textsf{DATA} which has 
service primitives \textsf{request}, \textsf{indication},
\textsf{response} and \textsf{confirm}.
\end{example}

\frmrule 

\begin{example}
Here we give an example a layer is a \textit{connection-oriented transport} layer, \textsf{T2}. \\
This layer provides the service element \textsf{CONNECT}, \textsf{DATA} and \textsf{DISCONNECT}, each 
with primitives \textsf{request}, \textsf{indication},
\textsf{response} and \textsf{confirm}.
\end{example}

\frmrule 

Layers are ordered according to some partial ordering. \\

\frmrule 

\textit{The difference between a layer and a layer entity}

Each layer has several \textit{entities}. 

\highlightdef{\textbf{Layer Entity}:}

\highlightdef{\textbf{Peer}: Two layers entities of the \textit{same layer} that reside \\
on \textit{different machines} are called \textit{peers}}

\frmrule 

\textit{Services}

A higher layer may provide a service to a lower layer. \\
A lower layer requests the service of a higher layer. 
We therefore call the higher layer a \textit{service provider}, 
and a lower layer a \textit{service user}. 

Each layer uses the services provided by the layer below 
to provide services to the layer above.
For any layer number $n$, we define:

\highlightdef{$n$-\textbf{Service}: An \textit{$n$-service}, $s_n$ is a capability provided 
by $L_n$ using the services of $L_{n-1}$ to provide 
a service to entities of $L_{n+1}$  }

Each layer is defined by the specific responsibilities assigned to it. 
A layer fulfils its responsibilities by providing
services to the layer above it using the services of the layer below. 
Each layer should be modular: it should have a single well-defined 
responsibility, and be able to change with affecting other layers. 

\begin{sidenote}{Single Responsibility, Open-Closed Principle}
Hopefully you can see a connection between the design of \textit{layers} in the OSI model 
and the design of \textit{classes} in object-oriented programs. Notice how the
the \textit{single-responsibility} and \textit{open-closed} principles 
of software engineering design apply here to the design of layers. 
\end{sidenote}

\frmrule 

\textit{Transitive access to services below}

By definition, a given layer $L_n$ has access to the service immediately below $L_{n-1}$. 
However, $L_{n-1}$ has access to the service immedaitely below, $L_{n-2}$. 
By transitivity, we have that a given layer $L_n$ has access to 
the services from \textit{all the layers below}, $L_n$, $L_{n-1}$, ..., $L_1$. 
It is not necessarily limited to just the services \textit{immediately} below.



\frmrule 

\highlightdef{\textbf{Timing Diagrams}:}
\highlightdef{\textbf{State Machines}:}


% In order to provide a layer to provide service/request a service, 
% a layer must go through an \textit{interface}. 

% \highlightdef{\textbf{Layer Interfaces}: Allowing layer $\textsf{T1}$ to send data to layer $\textsf{T2}$}

% For layer $\textsf{T1}$ to interface with $\textsf{T2}$. \\



\section{OSI Model: Service Access Points}



Each layer has one or more service access points (SAPs) to 
a lower layer. 
For an interaction of entities between layers to occur, an entities are connected via a 
\textit{service access point}. 

For a given point in time, an entity $e \in E_{n+1}$, may attached to one or more 
service access points $s \in S_{n}$. Furthermore, for a given point in time, 
an entity $e' \in E_{n+1}$ may be concurrently attached to one or more 
service access point $s \in S_{n}$. Any given service access point $s \in S_{n}$
is attached to exactly one $e \in E_{n+1}$ and exactly one $e \in E_{n}$, 
for any point in time. This is a classic example of a \textit{many-to-many relationship} that 
we saw in Databases. Here, we have 
a many-to-many relationship between $n+1$ entities and $n$ entities. 
The notion of a service access point acts as a relation (diamond) between them on 
an ER-diagram. 


\begin{figure}[h]
\begin{tikzpicture}[
  title/.style={},
  entity/.style={rectangle, draw, text centered}
]
%
\node[entity] (n1layer) {$n+1$-layer};
\node[entity, right = 1.5cm of n1layer] (n1entity) {$n+1$-entity};
\node[entity, below = 1cm of n1entity] (sap) {$n$-access point};
\node[entity, below = 1cm of sap] (nentity) {$n$-entity};
\node[entity, left = 2cm of nentity] (nlayer) {$n$-layer};
%
\draw(n1layer) -- (n1entity);
\draw(n1entity) -- (sap);
\draw(sap) -- (nentity);
\draw(nentity) -- (nlayer);
% draw crowsfeet
\node[left = 0.2cm of n1entity] (crow1) {};
\draw(crow1.east) -- (n1entity.170);
\draw(crow1.east) -- (n1entity.180);
\draw(crow1.east) -- (n1entity.190);
%
\node[above = 0.2cm of sap] (crow2) {};
\draw(crow2.south) -- (sap.50);
\draw(crow2.south) -- (sap.90);
\draw(crow2.south) -- (sap.130);
%
\node[below = 0.2cm of sap] (crow3) {};
\draw(crow3.north) -- (sap.310);
\draw(crow3.north) -- (sap.270);
\draw(crow3.north) -- (sap.230);
%
\node[below = 0.2cm of sap] (crow4) {};
\draw(crow4.north) -- (sap.310);
\draw(crow4.north) -- (sap.270);
\draw(crow4.north) -- (sap.230);
\end{tikzpicture}
\end{figure} 

A SAP is \textit{uniquely identified} by a URI. That is, 
each $n+1$ entity can uniquely identify the service access points 
it is connected to via some naming scheme. 
Similarly, each $n$ entity can uniquely identify the service access points 
it is connected to via some naming scheme. 




\frmrule 

\textit{Services are exchanged via the message exchange}

\highlightdef{\textbf{Exchange of Messages}: to provide services we need \textit{communication} }

We treat providing services as the sending and receiving of messages. 
According to this definition, we can, perhaps unintuitively, 
having messages send between entities \textit{on the same machine}. 

We handle this difference by making the distinction between a \textit{Interface Data Unit}, 
and a \textit{Protocol Data Unit}.  

\highlightdef{
\textbf{IDU}: unit of data exchanged between \textit{adjacent-layer} entities on \textit{the same machine}\\
\textbf{PDU}: unit of data exchanged between \textit{same-layer} entities on \textit{different machines}
}

Recall that by \textit{adjacent layers entities} we mean that the entities 
belong to layers whose layer numbers \textit{differ by one}.
By \textit{same layer} entities, the layer numbers are the \textit{same}.


\frmrule 

\textit{Case-by-case}


Allowed cases:\\
different machines, same layer\\
same layer, different machines\\
Forbidden cases:\\
same machines, same layer\\
same layer, same machines\\

\highlightdef{
\textbf{interface send/receive}: same machines, different layer\\
\textbf{protocol send/receive}: same layer, different machines
}


And in all allowed cases, if we send data across 
a layer (interface-send), we use \textbf{I}DUs,
and if we send data across the same layer (protocol-send), we use 
\textbf{P}DUs. 




\frmrule 





\section{OSI Model: Efficiency}


These two types of data exchange have a particular format. 
They are made up of two parts. 
One part is the actual information or \textit{payload}. This is 
the data that helps provide the service. 
We call this the \textit{Service Data Unit} or SDU.
The second part is called \textit{Control Information}. This contains 
additional information that is needed to send the payload. The Control 
informaton may contain for example, addresses, lenghs, checksums. 

\highlightdef{
\textbf{I}DU = [\textbf{I}CI $|$ SDU]\\
\textbf{P}DU = [\textbf{P}CI $|$ SDU] 
}



\highlightdef{
\textbf{SDU}: data exchange to provide the service\\
\textbf{CI}: control information added to a data unit
}

\frmrule 

\textit{Revisiting Interface send/receives and Machine send/receives, 
showing how control information is added on and stripped off SPUs}


\textbf{interface send}: $n+1$ sends $IDU$. $n$ strips ICI and keeps SDU \\
\textbf{interface receive}: $n$ sends $IDU$. $n+1$ strips ICI and keeps SDU\\
\textbf{protocol send}: $M_1$ sends $PDU$. $M_2$ strips PCI and keeps SDU \\
\textbf{protocol receive}: $M_1$ sends $PDU$. $M_2$ strips PCI and keeps SDU \\




\frmrule 

\textit{Efficiency of a single transfer}


\highlightdef{\textbf{Efficiency}: $\rho=\frac{|SDU|}{|CI|+|SDU|}$}


\frmrule 





\textit{Total Efficiency}


\highlightdef{\textbf{Transfer Efficiency}: $\rho_T = \frac{|SDU|}{N|PCI|+|SDU|}$}
For sending a service message across $N$ layers. 

\frmrule

\begin{example}
A particular OSI implementation with $N=7$ layers uses a $|PCI|$ of 4bytes. \\
Calculate total transfer efficiency for $|SDU|$ equal to:
(a) 4bytes, (b) 1000bytes.

(a) $\rho_T = \frac{|SDU|}{N|PCI|+|SDU|} = \frac{4}{7 \times 4 + 4} = 0.13$ \\
(b) $\rho_T = \frac{|SDU|}{N|PCI|+|SDU|} = \frac{1000}{7 \times 4 + 1000} = 0.97$ \\

\end{example}

\frmrule 

\textit{Giving each layer a unique formats and sizes}


\highlightdef{\textbf{Unique Sizes}: $|SDU_n|,|PDU_n|,|ICI_n|,|PCI_n|$}

\frmrule 


\textit{Segmentation}

We need to perform segmentation when we send data over an interface
(interface-send), we may find that our service data unit is \textit{too big}. 
That is, we may have the case where $|SDU_{n+1}| > |SDU_{n}|$. 
When this is the case, we break down the SDU into chunks of size $|SDU_{n}|$ 
along with a remainder with size $< |SDU_{n}|$. 


Each chunk and the remainder then become several service data unit for layer $n$.
The total number of data units is given by the following expression.
\highlightdef{\textbf{Segmentation}: no. SDUs $= \left\lceil |SDU_{n+1}|/|SDU_{n}|\right\rceil$}

Notice that this segmentation problem cannot happen when we perform a protocol 
send/receive. Recall that for a protocol send/receive, both entities involved are 
using the same network layer. We will never have the case where 
$|SDU_{n+1}| > |SDU_{n+1}|$.

\frmrule

\textit{Blocking and Concatenation}

When we send over an interface
(interface-send), we saw that we may need to perform segmentation. 
when service data unit is \textit{too big}. But what if the serice data 
is smaller. In fact it is \textit{very small}. In such a case, we may want to consider 
\textit{blocking} or \textit{concatentation}. Doing so could \textit{improve efficiency}.

Both Blocking and Concatentation are defined as the combination of 
several level $n+1$ interface data units to form a single 
service data unit for layer $n$. This difference between 
blocking and concatentation is that blocking discards the ICIs of 
the interface data units that we stuck together, where as 
concatentation doesn't. 

This is only possible when
$|SDU_{n+1}| < |SDU_{n}|$.
The number of interface data units of layer $n+1$ that can be packed together into a single SDU of layer $n$
for is given the following expression. 
We give the number for the case where blocking is used, and 
for the case where concatentation is used. 

\highlightdef{
\textbf{Blocking}:       no. IDUs $= \left\lfloor |SDU_{n}|/|SDU_{n+1}|\right\rfloor$\\
\textbf{Concatentation}: no. IDUs $= \left\lfloor |SDU_{n}|/|IDU_{n+1}|\right\rfloor$
}


Concatentation is more wasteful in that we have control 
information in the payload that is serves no use. Concatentation 
will therfore, in general, have a lower efficiency. However the processing cost is lower. 
It is easier for a processor to perform concatentation that it is 
for it to perform blocking. The tradeoff between blocking and concatenation 
is the tradeoff between network efficiency and processor load. 


\frmrule

\begin{example}
Suppose we have layers 3,2,1 with SDU sizes 64000, 1000, 3000. \\
Determine:\\
(a) the number of 3PDUs to send a 2PDU \\
(b) the number of 1PDUs to send a 3PDU

(a) Because $|SDU_2| < |SDU_3|$, we use \textit{fragmentation}. \\
So we have no. SDUs $= \left\lceil |SDU_{n+1}|/|SDU_{n}|\right\rceil = \left\lceil 64000/1000\right\rceil = 64$. \\
Hence 64 $n$-2PDUs are required to send a single 3SDU. \\
(b) Because $|SDU_2| > |SDU_1|$, we can use \textit{blocking}. \\
So we have no. of 2-SDUs per 1-SDU $= \left\lfloor |SDU_{n}|/|SDU_{n+1}|\right\rfloor = \left\lceil 3000/1000\right\rceil = 3$. \\
So no. of 1-SDUs per 3-SDUs $=$ (2-SDU/3-SDUS)$/$(2-SDUS/1-SDU) $=\ceil{64/3} = 22$
\end{example}


\frmrule

\textit{Connection Endpoints}

Recall that each SAP is uniquely identified by a URI. 
A SAP can contain one or more connection endpoint (CEP), each identified by a
channel number.

\frmrule

\begin{example}
In reference to the OSI model,\\
(a) What is meant by a \textit{protocol send/receive} and a \textit{machine send/receive} \\
(b) Explain what is meant by \textit{segmentation}\\
(c) Explain why segmentation cannot happen on a machine send/receive
\end{example}