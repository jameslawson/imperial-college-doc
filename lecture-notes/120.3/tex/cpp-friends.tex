
\chapter{Friends}

\section{Friend Functions}


% credit: yet another insignificant programming notes... %

A \textit{friend function} of a class, 
marked by the \lstinline{friend} keyword.
A friend function is a function defined outside the class, 
yet it has unrestricted access to all the class members.


\begin{lstlisting}
#include <iostream>
using namespace std;

// Set function is defined outside the class
void set(Point & point, int x, int y) {
  point.x = x;  // can access private data x and y
  point.y = y;
}

class Point {
   // The set function is defined outside this class, but it can access all
   // class members via parameter names matching member variables names 
   friend void set(Point & point, int x, int y);  // prototype
   private:
      int x, y;
   public:
      Point(int x = 0, int y = 0) : x(x), y(y) { }
      void print() const { cout << "(" << x << "," << y << ")" << endl; }
};
 
int main() {
   Point p1;
   p1.print();   // (0, 0)
   set(p1, 5, 6);
   p1.print();   // (5, 6)
}
\end{lstlisting}



\begin{itemize}   
\renewcommand{\labelitemi}{$\Box$}
\item A friend function acts as a regular function, \textit{not} a 
member function of the class. It is invoked \textit{without the dot operator} 
in the form of set(p1, 5, 6), instead of p1.set(5, 6) for a member function.
\item The above example is meant for illustration. 
This operation is better served by a member function void set(int x, int y), instead of friend function.
\item The friend function prototype is provided inside the class declaration. 
You do not need to provide another prototype outside the class declaration, 
but merely provide its implementation.
\item Friend functions can enhance the performance by directly accessing the 
private data members, eliminating the overhead of going thru the public member functions.
\item Friend functions are neither public nor private, and it can be declared anywhere 
inside the class. As friends are part of the extended interface of the class, 
you may group them together with the public functions.
Friend functions will not be inherited by the subclass. 
\item Friends can't be virtual, as friends are not class member.
\end{itemize}


\section{Friend Classes}

To declare all member functions of a class (says Class1) friend functions of 
another class (says Class2), declared \lstinline{friend class Class1;} in Class2. 
Friends are not \textit{symmetric}. That is, if Class1 is a friend of Class2, 
it does not imply that Class2 is a friend of Class1. 
Friends are also not \textit{transitive}. That is, if Class1 is a friend of Class2, and Class2 is a friend of Class3, it does not imply that Class1 is a friend of Class3.
Use friend with care. Incorrect use of friends may corrupt the concept 
of \textit{information hiding and encapsulation}.