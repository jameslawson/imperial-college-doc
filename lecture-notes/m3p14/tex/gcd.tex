
\chapter{Greatest Common Divisors}





\section{Gear problems}

The greatest common divisor arises most naturally in problems relating to \textit{gears}. 

In a gear mechanism, the driving cogwheel shall move two other free cogwheels.
The first free cogwheel has $a$ cogs and the second has $b$ cogs. 
A full turn of each of the free wheels shall correspond to several full 
turns of the driving wheel. \\
What are the possible cogs numbers can that driving wheel can have? \\
What is the largest number of cogs the driving wheel can have?\\

\highlightdef{Number of teeth = Number of dips}

Although this result is simple, reading the formal reasoning will be useful.  
We can perform off-by-one error analysis. We can \textit{unwrap} the 
circlular shape of a gear and put it on a line. Then it is clear 
that if it starts with a tooth, it must end with a dip. And if 
it starts with a dip, it must end with a tooth. 

\frmrule

We can view gears instead by simple gears. 


\highlightdef{\textbf{Division (Gears)}: $a$ \textit{divides} $b$, $a | b$ iff when $B$ has finished one complete revolution,
\\ $A$ has made $k$ \textit{complete} revolutions }

\frmrule

\begin{example}
Explain, using gears, why $1 | b$.
\end{example}

\frmrule

\begin{example}
Explain, using gears, why $a | b$ implies $a | kb$
\end{example}

\frmrule

\begin{example}
Explain, using gears, why $a | b$ and $b | c$ implies $a | c$
\end{example}


\frmrule

We can also combine several gears together to form a larger gear. 

\frmrule

\begin{example}
Explain, using gears, why $a | b$ and $a | c$ implies $a | (k_1b + k_2c)$

Assume $a | b$ and $b | c$. \\
Then, when gear $B$ makes a complete revolution, gear $A$ has made some complete number of revolutions say $k$.
And, when gear $C$ makes a complete revolution, gear $A$ makes a complete number of revolutions say $k'$.

Now consider gear $D$, with $k_1b + k_2c$ teeth. Let's turn $A$. \\
Well, when $A$ has made $k$ complete revolutions, $D$ has turned $b$ teeth. \\
So when $A$ has made $k_1k$ complete revolutions, $D$ has turned $k_1b$ teeth.  \\
And when, $A$ has made $k_1k + k'$ complete revolutions, $D$ has turned $k_1b + c$ teeth. \\
And when, $A$ has made $k_1k + k_2k'$ complete revolutions, $D$ has turned $k_1b + k_2c$ teeth.

So $D$ has made a complete revolution, $A$ has made some number of complete revolutions. 
Hence $a | (k_1b + k_2c)$.
\end{example}

\frmrule

It shouldn't surprise you that we can apply the previous result for more than two cogs. 
In general we have: $a | b_1$ and $a | b_2$ and ... and  $a | b_n$ implies $a | (k_1b_1 + k_2b_2 + ... + k_nb_n)$





\section{Greatest Common Divisor}


\frmrule 

\begin{example}
Explain, using gears, why $\text{gcd}(a,b) = g$ implies $\text{gcd}(a/g,b/g) = 1$

Let's assume $a \neq b$.\\ 
Without loss of generality, we can take $a < b$.\\
Assume $\text{gcd}(a,b) = g$. \\
Then gear $G$ is the largest gear such that once $A$ has made one complete revolution $B$ have made several revolutions, 
$G$ has made some number of complete revolutions, say $k$. \\

Suppose $G$ makes one revolution, then $A$ has turned $a/g$ teeth and $B$ has turned $k'b/g$ \\
Suppose we form two new gears, $A'$ with $a/g$ teeth and $B'$ with $b/g$ teeth.

When $g$ made $k$ revolutions, $A$ turned $a$ teeth, so if $G$ makes $1$ revolution, 
then $A$ turns $a/g$ teeth. And when $g$ made $k$ revolutions, $B$ made several revolutions, and so turned 
by $k'b$ teeth, for some $k'$. So if G makes one revolution, then $B$ makes $k'/g b$



\end{example}



% Detecting cycles in a linked list


\section{Euclidean Algorithm I}

\section{Linear Diophantine Equations I}