
\chapter{Modulo Ring Arithmetic}


\section{Modulo Inverses I}



\section{Modulo Inverses II}



\section{Chinese Remainder Theorem I}


\section{Chinese Remainder Theorem II}

Suppose that $m_1, m_2, \cdots, m_n$ are pairwise relatively prime 
integers where each $m_i \geqslant 2$. From the Chinese remainder 
theorem, we have shown any integer $a$ in $Z_M$ can be uniquely 
represented by $(a \bmod m_1, a \bmod m_2, \cdots, a \bmod m_n)$. 

\highlightdef{\textbf{Chinese Remainder Representation}: $(a \bmod m_1, a \bmod m_2, \cdots, a \bmod m_n)$}

\frmrule

\begin{example}
Let $m_1 = 3, m_2 = 4$. Find all pairs $(a \bmod m_1, a \bmod m_2)$ 
for all $a = 0 \cdots Z_M - 1$. Verify that they are unique.
\end{example}

\begin{example}
Consider what will happen when we continue $a$ past $Z_M - 1$. 
Will the Chinese Remainder representation still remain unique?
\end{example}