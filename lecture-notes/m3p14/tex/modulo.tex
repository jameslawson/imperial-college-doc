
\chapter{Modulo Arithmetic}


\section{Introducing Modulo Arithmetic}


\highlightdef{\textbf{Weaken mod}: $x mod N = y$ implies $x mod N \equiv y$}


\highlightdef{\textbf{Strengthen mod}: ($x mod N \equiv y $ and $0 \leqslant y < N$) implies $x mod N = y$  }




\highlightdef{\textbf{Idempotent}: $(x mod N) mod N = x mod N$}


\highlightdef{$(x + y) mod N \equiv (x mod N + y mod N)$}


\highlightdef{$(xy) mod N \equiv x(y mod N)$}


\[ 
\begin{array}{ll >{\scriptstyle}l}
xy mod N &= \underbrace{(y + y + ... + y)}_{x \text{ times}} mod N & \text{def of multiplication}  \\
         &\equiv (y + y + ... + y) mod N & \text{weaken mod}  \\
         &\equiv y mod N + y mod N + ... + y mod N & \text{weak mod distributes}\\
         &\equiv x(y mod N) & \text{def of multiplication}\\
\end{array}
\]

\frmrule

\begin{example}
\[ 
\begin{array}{ll}
(2 \times 13) mod 5 &\equiv 2 \times (13 mod 5)  \\
                    &\equiv 2 \times 3  \\
                    &\equiv 6  \\
                    &\equiv 1  \\
\end{array}
\]
\end{example}


\frmrule




\highlightdef{$x^n mod N \equiv (x mod N)^n$}

\frmrule


\section{Parity}

\begin{example}
Show that even mod even is even. That is, $a mod b$ is even when $a > b$ and $a$,$b$ are even. 

The remainder is $r = a - qb = \text{even} - q \times \text{even}$. Now, $q \times \text{even}$ is an even.
And an even added/subtracted to/from another even is \textit{even}. So $r$ is even. Hence $a mod b$ is even.
\end{example}



\section{Problems}


\textbf{Linked List Cycle Detection - Tortoise.Hare}

Suppose we have a linked list that has been corrupted in the sense 
that there is a back-edge that creates a loop. We can detect 
the loop \textit{and} find where the start of the loop is using a 
clever trick. 

\highlightdef{$T$ always moves \textit{one} step forward. $H$ always moves \textit{two} steps forward.}

Suppose we had just the loop. Number the nodes of the 
loop 0...$n-1$. We can reason about how the 
tortoise and hare move around this loop using modular 
arithmetic. Idea: moving $i$ steps forward from node 
$x$ puts you at node number $[x + i]_n$. 

Now suppose $T$ started at $[0]_n$, and $H$ started 
at some $[k]_n$. Now consider what happens after $n-k$ time steps. 
Clearly $T$ will be at $[n-k]_n$ (as it moves forward one node per time step). 
But where will $H$ be. Recall that $H$ moves forwarsd two nodes per time step. 
Since $H$ started at $k$ we have that $H$ finishes at node:
$[k+2(n-k)]_n = [k]_n + [2n]_n - [k]_n = [-k]_n = [n-k]_n$. 
So after $n-k$ steps, $T$ and $H$ land on the same node. And this works 
for any choice of $k$. 

Suppose $T$ starts at 0. 
\highlightdef{For any $k$, if $H$ starts at $k$ then $T$ and $H$ will meet at $[n-k]_n$.}
%Tortoise vs Hare problems. 
%The greatest common divisor is related to problems that we shall call \textit{Tortoise vs Hare} problem. 

We can now solving the original problem of finding the start of the loop. I.e. 
find exactly where $[0]_n$ is. Suppose we prepend nodes $-r, ..., -2, -1$ to the loop. 
If we let $k = r$. Then $T$ and $H$ will both start at the ROOT/HEAD, and 
after $n-r$ time steps meet at $n-r$. We can then have helper tortoises hares $T_1$
and $T_2$. One at the head $n-r$, one at $-r$ and then advance time forward 
until $T_1$ and $T_2$ meet. This is where the start is. 

\frmrule

\begin{example}
Modify the algorithm to find the \textit{end} of the loop.
\end{example}