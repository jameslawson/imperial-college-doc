
\chapter{Quantum Algorithms I}


\section{Parallel Hadamard Gates}






\highlightdef{\textbf{Parity Inner Product}: 
$[\textsf{x}, \; \textsf{y}] = \bigoplus^n_{i=1} \textsf{x}_i \wedge \textsf{y}_i$}

In other words, we component-wise, take the \textsc{and} operation, and then \textsc{xor} these together.
$$[\textsf{x}, \; \textsf{y}] = (x_1 \wedge y_1) \oplus (x_2 \wedge y_2) \oplus \dotsm \oplus (x_n \wedge y_n)$$

We call this an \textit{inner product} because we are taking the inner product 
of two vectors and outputting a number. Here our vectors our strings of binary digits
and the number outputting is $\{0,1\}$. In fact, we will show soon that this 
inner product behaves quite similarly with the inner product used for Euclidean spaces (dot product) 
and the inner product used for Hilbert spaces (Hermitian inner product) .

We call this the \textit{parity} inner product because we will soon that this product 
outputs one if and only if the \textit{product string} has an odd number of ones.
By \textit{product string}, we mean the the string that results 
from doing a component-wise \textsc{and} operation.

\frmrule

\begin{example}
Determine $[\textsf{0011}, \; \textsf{0101}]$ 
\end{example}

\frameans{}{$0$}


\frmrule

This may seem like a useless way to write the Hadamard operation but it was done in preparation 
for understanding how \textit{parallel} Hadamard gates operate. 

\highlightdef{$H\ket{x} = \frac{1}{\sqrt{2}} [(-1)^{[x,0]} \ket{0} + (-1)^{[x,1]} \ket{1}]$  }



\frmrule

\begin{example}
Show that $(-1)^{[u,v]}\ket{i} \otimes (-1)^{[x,y]}\ket{j} = (-1)^{[ux,vy]}\ket{ij}$ 
where $u,v,x,y,i,j$ are bits.
\end{example}

\frmrule

$(-1)^{u \wedge v} (-1)^{x \wedge y} \ket{i} \ket{j} = $\\
$(-1)^{u \wedge v + x \wedge y} \ket{ij} = $\\
$(-1)^{[u \wedge v + x \wedge y]\text{mod } 2} \ket{ij} = $\\
$(-1)^{u \wedge v \oplus x \wedge y} \ket{ij} = $
$(-1)^{[ux, vy]} \ket{ij} $


\frmrule

\begin{example}
Show that $(-1)^{[ux,vy]}\ket{ij} \otimes (-1)^{[s, t]}\ket{k} = (-1)^{[uxs,vyt]}\ket{ijk}$ 
\end{example}

\frmrule

After doing the previous examples, it's now not too hard to what happens 
in the general case. 

$(-1)^{x_1 \wedge y_1}\ket{i_1} \otimes
(-1)^{x_2 \wedge y_2}\ket{i_2} \otimes
\dotsm \otimes
(-1)^{x_n \wedge y_n}\ket{i_n} = $ 

$(-1)^{x_1 \wedge y_1 + x_2 \wedge y_2 + \dotsm + x_n \wedge y_n}\ket{i_1 i_2 \dotsm i_n} = $

$(-1)^{[x_1 \wedge y_1 + x_2 \wedge y_2 + \dotsm + x_n \wedge y_n] \text{mod } 2}\ket{i_1 i_2 \dotsm i_n}$

$(-1)^{(x_1 \wedge y_1) \oplus (x_2 \wedge y_2) \oplus \dotsm \oplus (x_n \wedge y_n)}
\ket{i_1 i_2 \dotsm i_n} =$

$(-1)^{[x_1 x_2 \dotsm x_n, \; y_1 y_2 \dotsm y_n]}
\ket{i_1 i_2 \dotsm i_n}$

We will use this next.

\frmrule


\begin{example}
Determine $H^{\otimes n}\ket{x}$.
\end{example}


$H^{\otimes n}\ket{x}$

$= 
\frac{1}{\sqrt{2}} \left[(-1)^{x_1 \wedge 0} \ket{0} + (-1)^{x_1 \wedge 1} \ket{1} \right] \otimes
\frac{1}{\sqrt{2}} \left[(-1)^{x_2 \wedge 0} \ket{0} + (-1)^{x_2 \wedge 1} \ket{1} \right] \otimes
\dotsm 
\\ \otimes \frac{1}{\sqrt{2}} \left[(-1)^{x_n \wedge 0} \ket{0} + (-1)^{x_n \wedge 1} \ket{1} \right]
$

$= \sum_{y_1 y_2 \dotsm y_n \in \{0,1\}^n}
\frac{1}{\sqrt{2}} [(-1)^{x_1 \wedge y_1}\ket{y_1}] \otimes \frac{1}{\sqrt{2}} [(-1)^{x_2 \wedge y_2}\ket{y_2}] 
\otimes \dotsm \otimes \frac{1}{\sqrt{2}} [(-1)^{x_n \wedge y_n}\ket{y_n}]
$

$= 
\frac{1}{\sqrt{2}^n} \sum_{y_1 y_2 \dotsm y_n \in \{0,1\}^n}
(-1)^{x_1 \wedge y_1 + x_2 \wedge y_2 + \dotsm + x_n \wedge y_n} \ket{y_1 y_2 \dotsm y_n}
$

$= 
\frac{1}{\sqrt{2}^n} \sum_{y_1 y_2 \dotsm y_n \in \{0,1\}^n}
(-1)^{[x_1 \wedge y_1 + x_2 \wedge y_2 + \dotsm + x_n \wedge y_n] \text{mod } 2 } \ket{y_1 y_2 \dotsm y_n}
$

$= 
\frac{1}{\sqrt{2}^n} \sum_{y_1 y_2 \dotsm y_n \in \{0,1\}^n}
(-1)^{(x_1 \wedge y_1) \oplus (x_2 \wedge y_2) \oplus \dotsm \oplus (x_n \wedge y_n)} \ket{y_1 y_2 \dotsm y_n}
$

$= 
\frac{1}{\sqrt{2}^n} \sum_{y_1 y_2 \dotsm y_n \in \{0,1\}^n}
(-1)^{[x_1 x_2 \dotsm x_n, \; y_1 y_2 \dotsm y_n]} \ket{y_1 y_2 \dotsm y_n}
$



\highlightdef{
\textbf{Parallel Hadamard}: 
$H^{\otimes n}\ket{\textsf{x}} = 
\frac{1}{\sqrt{2}^n} \sum_{\textsf{y} \in \{0,1\}^n}
(-1)^{[\textsf{x}, \; \textsf{y}]} \ket{\textsf{y}}$  }

This is known as \textit{Fourier Samping}. 

\frmrule

\highlightdef{ $h_{(i,j)} = \frac{1}{\sqrt{N}}(-1)^{[i,j]}$ }

where $h_{(i,j)}$ denotes the element of $H^{\otimes n}$ at row $i$ and column $j$, 
where we number row/columns from 0 to $N-1$. 


\frmrule

\begin{example}
Write out explicitly $H^{\otimes 3}$ 
\end{example}

\frmrule

\begin{example}
Write out explicitly the 7th column of $H^{\otimes 4}$ 
\end{example}

\frmrule

\begin{example}
Prove by induction over $N$ that $h_{(i,j)} = \frac{1}{\sqrt{N}}(-1)^{[i,j]}$
\end{example}

\frmrule

\begin{example}
Using the previous result, what is $H^{\otimes n}\ket{\textsf{00..0}}$?
\end{example}

We insert $\textsf{x} = \textsf{00..0}$ into the Parallel Hadamard equation to get:\\
$H^{\otimes n}\ket{\textsf{00..0}} = 
\frac{1}{\sqrt{2}^n} \sum_{\textsf{y} \in \{0,1\}^n}
(-1)^{[\textsf{00..0}, \; \textsf{y}]} \ket{\textsf{y}}$
$ = \frac{1}{\sqrt{2}^n} \sum_{\textsf{y} \in \{0,1\}^n} \ket{\textsf{y}}$

\frmrule

This is a special case of Fourier Sampling where here, instead of 
using a general $\ket{\textsf{x}}$, we are using $\textsf{x} = \textsf{00..0}$. 
Notice how this gives a superposition of all $2^n$ computational basis vectors 
for an $n$ qubit system.

\highlightdef{An $n$-qubit \textbf{Uniform Superposition}
is given by $H^{\otimes n}\ket{\textsf{00..0}}$
}
In other words, by connecting $n$ Hadamards in parallel each with input $\ket{0}$. 

What happens if we give $H^{\otimes n}$ a string that is not all zeros. 
Well, we actually still get an $n$-qubit superposition, but there will 
be some \textit{minus signs}.

\frmrule

\begin{example}
Given that $H$ is its own inverse, prove that $H^{\otimes n}$ is its own inverse.
\end{example}

\frmrule

Because $H^{\otimes n}$ is it's own inverse, we have the following result. 

\highlightdef{
\textbf{Undo Superposition}:
$H^{\otimes n}
(\frac{1}{\sqrt{2}^n} \sum_{\textsf{y} \in \{0,1\}^n}
(-1)^{[\textsf{x}, \; \textsf{y}]} \ket{\textsf{y}}) 
= \ket{\textsf{x}}$ 
}

\frmrule

This is a special case where the superposition is undone and $\ket{\textsf{x}}$ is recoverd. 
It is not always the case that taking the Hadamard of a superposition performs this undoing action. 
It just so happened to work in this case because the superposition had a 
special structure having $(-1)^{[\textsf{x}, \; \textsf{y}]}$ and using 
all the computational basis vectors, $\textsf{y} \in \{0,1\}^n$.

We can consider a more general case of taking $H^{\otimes n}$ 
for a \textit{general} superposition of $\sum_x \alpha_x H^{\otimes n} \ket{x}$. 
The coefficients of this superposition are a general $\alpha_x$ and we do not have to 
use the $n$ computational basis vectors. 

\highlightdef{
\textbf{Hadamard of Superposition}:
$H^{\otimes n} (\sum_x \alpha_x \ket{x}) 
= \sum_x \sum_{y \in \{0,1\}^n} \alpha_x (-1)^{[x,y]} \ket{y}$
}

\frmrule

\begin{example}
By substituting the superposition 
$\sum_x \alpha_x \ket{x} = \sum_{\textsf{x} \in \{0,1\}^n} (-1)^{[\textsf{x}, \; \textsf{z}]}\ket{\textsf{x}}$
into the formula above, show that the result is $\ket{\textsf{z}}$
\end{example}


\frmrule

\begin{tikzpicture}
    %
    \matrix[row sep=0.1cm, column sep=0.8cm] (circuit) {
    % row
    \node (q1) {$H$};  & \coordinate (uf1); & &;
    \coordinate (end1); \\
    % next row:
    \node (q2) {\ket{\textsf{x}_2}}; & & &
    \coordinate (end2);\\
    % next row:
    \node (qx) {...};  & & &
    \coordinate (endx); \\
    % next row:
    \node (q3) {\ket{\textsf{x}_n}}; & & &;
    \coordinate (end3); \\
     % next row:
    \node (q4) {\ket{-}}; & & \coordinate (uf2); &;
    \coordinate (end4); \\
    };

    \begin{pgfonlayer}{background}
        \draw[thick] (q1) -- (end1)  (q2) -- (end2) (q3) -- (end3) (q4) -- (end4);
    \end{pgfonlayer}

    \begin{pgfonlayer}{foreground}
        \node [draw=black, fill=black!5, drop shadow, inner sep=5pt, outer sep=5pt, fit={(uf1) (uf2)}] {$U_f$};
    \end{pgfonlayer}
    %
\end{tikzpicture}



\section{Bernstein-Vazirani Algorithm}

The \textit{parity problem}.

\highlightdef{
\textbf{Parity Problem}: Input a function $f: \{0,1\}^n \rightarrow \{0,1\}$
where $f(x) = [\textsf{u},\; \textsf{x}]$ for some hidden $\textsf{u} \in \{0,1\}^n$, 
output $\textsf{u}$ 
}

\highlightdef{The \textbf{Bernstein-Vazirani Algorithm} solves the \textit{Parity Problem}}

The definition of $f$ is that it acts as a \textit{parity mask} (similar to a bit mask except that 
we are outputting parities rather than bits). 
$f(x) = 0$ only when $\textsf{x}$ 
We want to use as few queries to $f$ as possible. 

\begin{tikzpicture}
    %
    \matrix[row sep=0.1cm, column sep=0.8cm] (circuit) {
    % row
    \node (q1) {\ket{\textsf{x}_1}};  & \coordinate (uf1); & &;
    \coordinate (end1); \\
    % next row:
    \node (q2) {\ket{\textsf{x}_2}}; & & &
    \coordinate (end2);\\
    % next row:
    \node (qx) {...};  & & &
    \coordinate (endx); \\
    % next row:
    \node (q3) {\ket{\textsf{x}_n}}; & & &;
    \coordinate (end3); \\
     % next row:
    \node (q4) {\ket{-}}; & & \coordinate (uf2); &;
    \coordinate (end4); \\
    };

    \begin{pgfonlayer}{background}
        \draw[thick] (q1) -- (end1)  (q2) -- (end2) (q3) -- (end3) (q4) -- (end4);
    \end{pgfonlayer}

    \begin{pgfonlayer}{foreground}
        \node [draw=black, fill=black!5, drop shadow, inner sep=5pt, outer sep=5pt, fit={(uf1) (uf2)}] {$U_f$};
    \end{pgfonlayer}
    %
\end{tikzpicture}



\highlightdef{
$U_f\ket{\textsf{x}}\ket{-} = (-1)^{f(\textsf{x})} \ket{\textsf{x}}\ket{-}$
}

The output is the same except that whenever $f(\textsf{x}) = 1$, 
the output picks up a relative phase of minus one.

Now suppose we add a parallel Hadamard before $U_f$.
Instead of inputting $\ket{x}$, we input $H^{\otimes n}\ket{\textsf{00..0}}$. 
Recall from our previous work on parallel Hadamard gates that:
$H^{\otimes n}\ket{\textsf{00..0}} 
= \frac{1}{\sqrt{2}^n} \sum_{\textsf{y} \in \{0,1\}^n} \ket{\textsf{y}}$



\begin{tikzpicture}[thick]
    %
    \matrix[row sep=0.1cm, column sep=0.8cm] (circuit) {
    % row
    \node (q1) {\ket{0}}; &
    \coordinate (hn1); & & \coordinate (uf1); & &
    \coordinate (end1); \\
    % next row:
    \node (q2) {\ket{0}}; & & & & &
    \coordinate (end2);\\
    % next row:
    \node (qx) {...}; & & & & &
    \coordinate (endx); \\
    % next row:
    \node (q3) {\ket{0}}; & &
    \coordinate (hn2); & & &
    \coordinate (end3); \\
    % next row:
    \node (q4) {\ket{-}}; & & & & \coordinate (uf2); &
    \coordinate (end4); \\
    };
    % Draw bracket on right with resultant state.
    \draw[decorate,decoration={brace},thick]
        ($(end1.east)+(0.1cm,0.3cm)$)
        to node[midway,right] (bracket) {$\ket{\phi'}$}
        ($(end4.east)+(0.1cm,-0.3cm)$);
    \begin{pgfonlayer}{background}
        % Draw circuit parallel lines.
        \draw[thick] (q1) -- (end1)  (q2) -- (end2) (q3) -- (end3) (q4) -- (end4);
    \end{pgfonlayer}

    \begin{pgfonlayer}{foreground}
        \node [draw=black, fill=black!5,drop shadow, inner sep=5pt, outer sep=5pt, fit={(hn1) (hn2)}] {$H^{\otimes n}$};
        \node [draw=black, fill=black!5,drop shadow, inner sep=5pt, outer sep=5pt, fit={(uf1) (uf2)}] {$U_f$};
    \end{pgfonlayer}
    %
\end{tikzpicture}

The output is now the following.

\[ 
\begin{array}{lll}
 \ket{\phi'} &= U_f [(H^{\otimes n} \otimes I)(\ket{\textsf{0}^n}) \otimes \ket{-})]  & \text{from circuit} \\
  &= U_f [(H^{\otimes n}\ket{\textsf{0}^n}) \otimes (I\ket{-})]  & \text{multiplication property} \\
  &= U_f [H^{\otimes n}\ket{\textsf{0}^n}\ket{-}]  & \text{identity matrix} \\
  &= U_f [(\frac{1}{\sqrt{2}^n} \sum_{\textsf{x} \in \{0,1\}^n} \ket{\textsf{x}}) \ket{-}] & \text{Hadamard superposition} \\
  &= U_f [\frac{1}{\sqrt{2}^n} \sum_{\textsf{x} \in \{0,1\}^n} \ket{\textsf{x}} \ket{-}]  
  & \text{distributivity of } \otimes  \text{over } + \\  
  &= \frac{1}{\sqrt{2}^n} \sum_{\textsf{x} \in \{0,1\}^n} U_f \ket{\textsf{x}} \ket{-}  & \text{linearity of } U_f \\
  &= \frac{1}{\sqrt{2}^n} \sum_{\textsf{x} \in \{0,1\}^n} (-1)^{f(x)} \ket{\textsf{x}} \ket{-}  
  & U_f \text{ with } \ket{\textsf{x}}\ket{-} \text{ result} \\
  &= \frac{1}{\sqrt{2}^n} \sum_{\textsf{x} \in \{0,1\}^n} (-1)^{[u,\;x]} \ket{\textsf{x}} \ket{-}  
  & \text{def of } f(x)  \\
  &= (\frac{1}{\sqrt{2}^n} \sum_{\textsf{x} \in \{0,1\}^n} (-1)^{[u,\;x]} \ket{\textsf{x}}) \ket{-}  
  & \text{distributivity of } \otimes  \text{over } +   
\end{array}
\] 

Recall that $H^{\otimes n}$ is its own inverse. 
Now if we feed the first $n$ qubits through a parallel Hadamard 
then we undo the superposition and get $u$. We simply measure this and we have our output.


\begin{tikzpicture}[thick]
    %
    \matrix[row sep=0.1cm, column sep=0.8cm] (circuit) {
    % row
    \node (q1) {\ket{0}}; &
    \coordinate (hn1); & & \coordinate (uf1); & & \coordinate (hn3); & &;
    \coordinate (end1); \\
    % next row:
    \node (q2) {\ket{0}}; & & & & & & &
    \coordinate (end2);\\
    % next row:
    \node (qx) {...}; & & & & & & &
    \coordinate (endx); \\
    % next row:
    \node (q3) {\ket{0}}; & &
    \coordinate (hn2); & & & & \coordinate (hn4); &;
    \coordinate (end3); \\
    % next row:
    \node (q4) {\ket{-}}; & & & & \coordinate (uf2); & & &
    \coordinate (end4); \\
    };
    % Draw bracket on right with resultant state.
    \draw[decorate,decoration={brace},thick]
        ($(end1.east)+(0.1cm,0.3cm)$)
        to node[midway,right] (bracket) {$\ket{\phi''}$}
        ($(end4.east)+(0.1cm,-0.3cm)$);
    \begin{pgfonlayer}{background}
        % Draw circuit parallel lines.
        \draw[thick] (q1) -- (end1)  (q2) -- (end2) (q3) -- (end3) (q4) -- (end4);
    \end{pgfonlayer}

    \begin{pgfonlayer}{foreground}
        \node [draw=black, fill=black!5, drop shadow, inner sep=5pt, outer sep=5pt, fit={(hn1) (hn2)}] {$H^{\otimes n}$};
        \node [draw=black, fill=black!5, drop shadow, inner sep=5pt, outer sep=5pt, fit={(uf1) (uf2)}] {$U_f$};
        \node [draw=black, fill=black!5, drop shadow, inner sep=5pt, outer sep=5pt, fit={(hn3) (hn4)}] {$H^{\otimes n}$};
    \end{pgfonlayer}
    %
\end{tikzpicture}


\[ 
\begin{array}{lll}
 \ket{\phi''} 
  &= (H^{\otimes n} \otimes I)([\frac{1}{\sqrt{2}^n} \sum_{\textsf{x} \in \{0,1\}^n} (-1)^{[u,\;x]} \ket{\textsf{x}}] \otimes \ket{-})
  & \text{from circuit } \\
  &= (H^{\otimes n} [\frac{1}{\sqrt{2}^n} \sum_{\textsf{x} \in \{0,1\}^n} (-1)^{[u,\;x]} \ket{\textsf{x}}])(I\ket{-})
  & \text{multiplication property } \\
  &= \ket{\textsf{u}}(I\ket{-})
  & \text{undo superposition }    

\end{array}
\] 

We can add some final touches by using $H\ket{1} = \ket{-}$ and adding the measurement to the circuit. 
Hence our final circuit for the Bernstein-Vazirani Algorithm is the following. 

% TODO: add measurement to diagram
\begin{tikzpicture}
    %
    \matrix[row sep=0.1cm, column sep=0.8cm] (circuit) {
    % row
    \node (q1) {\ket{0}}; &
    \coordinate (hn1); & & \coordinate (uf1); & & \coordinate (hn3); & &;
    \coordinate (end1); \\
    % next row:
    \node (q2) {\ket{0}}; & & & & & & &
    \coordinate (end2);\\
    % next row:
    \node (qx) {...}; & & & & & & &
    \coordinate (endx); \\
    % next row:
    \node (q3) {\ket{0}}; & &
    \coordinate (hn2); & & & & \coordinate (hn4); &;
    \coordinate (end3); \\
    % next row:
    \node (q4) {\ket{1}}; & \node[draw=black, fill=black!5, drop shadow, inner sep=5pt, xshift=5pt] {$H$}; 
    & &  & \coordinate (uf2); & & &
    \coordinate (end4); \\
    };

    \begin{pgfonlayer}{background}
        \draw[thick] (q1) -- (end1)  (q2) -- (end2) (q3) -- (end3) (q4) -- (end4);
    \end{pgfonlayer}

    \begin{pgfonlayer}{foreground}
        \node [draw=black, fill=black!5, drop shadow, inner sep=5pt, outer sep=5pt, fit={(hn1) (hn2)}] {$H^{\otimes n}$};
        \node [draw=black, fill=black!5, drop shadow, inner sep=5pt, outer sep=5pt, fit={(uf1) (uf2)}] {$U_f$};
        \node [draw=black, fill=black!5, drop shadow, inner sep=5pt, outer sep=5pt, fit={(hn3) (hn4)}] {$H^{\otimes n}$};
    \end{pgfonlayer}
    %
\end{tikzpicture}









\section{Deutsch-Jorza Algorithm}



\highlightdef{
\textbf{Balanced Problem}: 
Can we determine whether a function $f: \{0,1\}^n \rightarrow \{0,1\}$ 
is \textit{balanced or not} using $U_f$ only once in a quantum circuit
}

\highlightdef{The \textbf{Deutsch-Jorza Algorithm} solves the \textit{Balanced Problem}}

In turns out that the Deutsch-Jorza Algorithm uses the \textit{same circuit} as the 
Bernstein-Vazirani Algorithm. But notice that now, we don't have $f(x) = [u, \; x]$ 
We instead assuming that $f(x)$ is either \textit{balanced} or \textit{constant}.
We no longer assume it's a parity function as was the case in the parity problem. 
There is no longer a $u$ to output, so the \textit{last part} of the Bernstein-Vazirani Algorithm 
will be slightly different. 
We will deviate from the last part of Bernstein-Vazirani Algorithm derivation to, instead,
create a circuit that solves the Balanced Problem.

\begin{tikzpicture}[thick]
    %
    \matrix[row sep=0.1cm, column sep=0.8cm] (circuit) {
    % row
    \node (q1) {\ket{0}}; &
    \coordinate (hn1); & & \coordinate (uf1); & &
    \coordinate (end1); \\
    % next row:
    \node (q2) {\ket{0}}; & & & & &
    \coordinate (end2);\\
    % next row:
    \node (qx) {...}; & & & & &
    \coordinate (endx); \\
    % next row:
    \node (q3) {\ket{0}}; & &
    \coordinate (hn2); & & &
    \coordinate (end3); \\
    % next row:
    \node (q4) {\ket{-}}; & & & & \coordinate (uf2); &
    \coordinate (end4); \\
    };
    % Draw bracket on right with resultant state.
    \draw[decorate,decoration={brace},thick]
        ($(end1.east)+(0.1cm,0.3cm)$)
        to node[midway,right] (bracket) {$\ket{\phi'}$}
        ($(end4.east)+(0.1cm,-0.3cm)$);
    \begin{pgfonlayer}{background}
        % Draw circuit parallel lines.
        \draw[thick] (q1) -- (end1)  (q2) -- (end2) (q3) -- (end3) (q4) -- (end4);
    \end{pgfonlayer}

    \begin{pgfonlayer}{foreground}
        \node [draw=black, fill=black!5,drop shadow, inner sep=5pt, outer sep=5pt, fit={(hn1) (hn2)}] {$H^{\otimes n}$};
        \node [draw=black, fill=black!5,drop shadow, inner sep=5pt, outer sep=5pt, fit={(uf1) (uf2)}] {$U_f$};
    \end{pgfonlayer}
    %
\end{tikzpicture}

The output is now the following.

\[ 
\begin{array}{lll}
 \ket{\phi'} &= U_f [(H^{\otimes n} \otimes I)(\ket{\textsf{0}^n}) \otimes \ket{-})]  & \text{from circuit} \\
  &= U_f [(H^{\otimes n}\ket{\textsf{0}^n}) \otimes (I\ket{-})]  & \text{multiplication property} \\
  &= U_f [H^{\otimes n}\ket{\textsf{0}^n}\ket{-}]  & \text{identity matrix} \\
  &= U_f [(\frac{1}{\sqrt{2}^n} \sum_{\textsf{x} \in \{0,1\}^n} \ket{\textsf{x}}) \ket{-}] & \text{Hadamard superposition} \\
  &= U_f [\frac{1}{\sqrt{2}^n} \sum_{\textsf{x} \in \{0,1\}^n} \ket{\textsf{x}} \ket{-}]  
  & \text{distributivity of } \otimes  \text{over } + \\  
  &= \frac{1}{\sqrt{2}^n} \sum_{\textsf{x} \in \{0,1\}^n} U_f \ket{\textsf{x}} \ket{-}  & \text{linearity of } U_f \\
  &= \frac{1}{\sqrt{2}^n} \sum_{\textsf{x} \in \{0,1\}^n} (-1)^{f(x)} \ket{\textsf{x}} \ket{-}  
  & U_f \text{ with } \ket{\textsf{x}}\ket{-} \text{ result} 
\end{array}
\] 

We now feed the first $n$ bits through an parallel Hadamard.
From our previous work on parallel Hadamards, recall the formula for 
the Hadamard of a general superposition: $H^{\otimes n} (\sum_x \alpha_x \ket{x}) 
= \sum_x \sum_{y \in \{0,1\}^n} \alpha_x (-1)^{[x,y]} \ket{y}$. We will use this 
result here, substituting $\sum_x \alpha_x \ket{x}$ for the superposition we 
current have in the circuit of  $\sum_{\textsf{x} \in \{0,1\}^n} (-1)^{f(x)} \ket{\textsf{x}}$. 

\begin{tikzpicture}
    %
    \matrix[row sep=0.1cm, column sep=0.8cm] (circuit) {
    % row
    \node (q1) {\ket{0}}; &
    \coordinate (hn1); & & \coordinate (uf1); & & \coordinate (hn3); & &;
    \coordinate (end1); \\
    % next row:
    \node (q2) {\ket{0}}; & & & & & & &
    \coordinate (end2);\\
    % next row:
    \node (qx) {...}; & & & & & & &
    \coordinate (endx); \\
    % next row:
    \node (q3) {\ket{0}}; & &
    \coordinate (hn2); & & & & \coordinate (hn4); &;
    \coordinate (end3); \\
    % next row:
    \node (q4) {\ket{-}}; & & & & \coordinate (uf2); & & &
    \coordinate (end4); \\
    };
    % Draw bracket on right with resultant state.
    \draw[decorate,decoration={brace},thick]
        ($(end1.east)+(0.1cm,0.3cm)$)
        to node[midway,right] (bracket) {$\ket{\phi''}$}
        ($(end4.east)+(0.1cm,-0.3cm)$);
    \begin{pgfonlayer}{background}
        % Draw circuit parallel lines.
        \draw[thick] (q1) -- (end1)  (q2) -- (end2) (q3) -- (end3) (q4) -- (end4);
    \end{pgfonlayer}

    \begin{pgfonlayer}{foreground}
        \node [draw=black, fill=black!5, drop shadow, inner sep=5pt, outer sep=5pt, fit={(hn1) (hn2)}] {$H^{\otimes n}$};
        \node [draw=black, fill=black!5, drop shadow, inner sep=5pt, outer sep=5pt, fit={(uf1) (uf2)}] {$U_f$};
        \node [draw=black, fill=black!5, drop shadow, inner sep=5pt, outer sep=5pt, fit={(hn3) (hn4)}] {$H^{\otimes n}$};
    \end{pgfonlayer}
    %
\end{tikzpicture}

\[ 
\begin{array}{lll}
  \ket{\phi''} 
  &= (H^{\otimes n} \otimes I)([\frac{1}{\sqrt{2}^n} \sum_{\textsf{x} \in \{0,1\}^n} (-1)^{f(x)} 
  \ket{\textsf{x}}] \otimes \ket{-}) & \text{from circuit} \\
  &= (\frac{1}{\sqrt{2}^n} H^{\otimes n}[\sum_{\textsf{x} \in \{0,1\}^n} (-1)^{f(x)}\ket{\textsf{x}}])
  \otimes (I \ket{-}) & \text{mult. property} \\
  &= (\frac{1}{\sqrt{2}^n} H^{\otimes n}[\sum_{\textsf{x} \in \{0,1\}^n} (-1)^{f(x)}\ket{\textsf{x}}])
  \otimes \ket{-} & \text{identity matrix} \\
  &= \frac{1}{\sqrt{2}^n} \frac{1}{\sqrt{2}^n} 
  [\sum_{\textsf{x} \in \{0,1\}^n} \sum_{\textsf{y} \in \{0,1\}^n}
  (-1)^{f(x)} (-1)^{[x,y]} \ket{\textsf{y}}]
  \otimes \ket{-} & H^{\otimes n} \text{ of superpos.} \\
  &= \frac{1}{2^n}
  [\sum_{\textsf{x} \in \{0,1\}^n} \sum_{\textsf{y} \in \{0,1\}^n}
  (-1)^{f(x)+[x,y]} \ket{\textsf{y}}]
  \otimes \ket{-} & \text{simplify}
\end{array}
\] 

We have a sum of y-qubits $\sum_{\textsf{y}}$. 
Notice that there are no more x-qubits being summed up.
The next step is to \textit{measure}. 
It we measure the first $n$ qubits, there are $2^n$ states that it can collapse to. 
There are $2^n$ bit strings that can be output. 

It turns out that whether or not we see $\ket{0^n}$ alone, is enough to
tell us whether $f(x)$ is constant or balanced
(recall our assumption is that the $f(x)$ inputted \textit{must be} either 
constant or balanced)


\frmrule

\begin{example}
What is the probability that $\ket{\phi''}$ collapses $\ket{0^n}$?

We find the amplitude (coefficient) of $\ket{0^n}$ and square it. 
The amplitude of $\ket{0^n}$ in $\ket{\phi''}$ is found by removing the summation and 
fixing $y = 0^n$. This gives us the amplitude ...........

\frameans{}{$\frac{1}{2^n} \sum_{\textsf{x} \in \{0,1\}^n} (-1)^{f(x)}$}

Because the amplitude of a general 
$y$ is $\frac{1}{2^n} \sum_{\textsf{x} \in \{0,1\}^n} (-1)^{f(x)+[x,\;y]}$.
Setting $y = 0^n$ gives $\frac{1}{2^n} \sum_{\textsf{x} \in \{0,1\}^n} (-1)^{f(x)+[x,\;0^n]}$
which simplifies using $[x,\;0^n] = 0$.

\end{example}


\begin{itemize}   
\renewcommand{\labelitemi}{$\Box$}
\item \textbf{Balanced}
If $f$ is balanced, $\sum_{\textsf{x} \in \{0,1\}^n} (-1)^{f(x)} = 0$. This is because 
half the outputs are 0, half the outputs are 1 (by def. of \textit{balanced}). 
So half the summation terms are -1, half the terms are +1. 
They cancel out in the sum to give zero. Hence the amplitude of $\ket{0^n}$ is $\frac{1}{2^n}(0) = 0$.
So the probability of seeing $\ket{0^n}$ after measurement is $0^2 = 0$.
\item \textbf{Constant with Zero} 
If $f$ outputs constantly zero, $f(x) = 0$ for all $x$, and so the summation simplifies to
$\sum_{\textsf{x} \in \{0,1\}^n} (-1)^{f(x)} = \sum_{\textsf{x} \in \{0,1\}^n} 1 = 2^n$. 
Hence the amplitude of $\ket{0^n}$ is $\frac{1}{2^n}(2^n) = 1$.
So the probability of seeing $\ket{0^n}$ after measurement is $1^2 = 1$.
\item \textbf{Constant with One}
If $f$ is constantly one, $f(x) = 1$ for all $x$, and so the summation 
$\sum_{\textsf{x} \in \{0,1\}^n} (-1)^{f(x)} = \sum_{\textsf{x} \in \{0,1\}^n} (-1) = -2^n$.
Hence the amplitude of $\ket{0^n}$ is $\frac{1}{2^n}(-2^n) = -1$.
So the probability of seeing $\ket{0^n}$ after measurement is $(-1)^2 = 1$.
\end{itemize}

\highlightdef{After measuring the first $n$ quibits, if we observe $\ket{0^n}$, $f$ is \textit{definitely} constant, 
if we don't observe $\ket{0^n}$, $f$ is \textit{definitely} balanced}


The natural question is what happens is what happens when we input a function that 
isn't balanced or constant into this circuit. If we do this, then the 
amplitude of $\ket{0^n}$ will be between zero and one. So the probability of 
seeing $\ket{0^n}$ is $0 < p < 1$, and so sometimes the circuit outputs 
\textit{constant} (with probability $p$), and sometimes the circuit will 
output \textit{balanced} (with probability $1-p$).

However constant and balanced functions are special because they
conveniently cause the amplitude for $\ket{0^n}$ to be 1 and 0 respectively.
By using a single measurement, we can see whether $f$ is
constant or balanced. This gives an exponential reduction in
the complexity of the classical algorithm.

\section{Simon's Algorithm}


\section{Grover's Algorithm}


\highlightdef{\textbf{Averaging Operator}: finds the mean amplitude of $phi$}


\begin{example}
Show that the averaging operator is \textit{unitary}.
\end{example}



\highlightdef{\textbf{About-the-Mean Inversion}: $v_i'= \frac{2}{N}\sum_j v_j - v_i$}



\begin{example}
Prove that \textit{about-the-mean inversion} does not change the mean.
\end{example}





\highlightdef{\textbf{Increase Amplitude}: combine \textit{phase inversion} with 
\textit{about-the-mean inversion}}