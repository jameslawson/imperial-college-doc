
\chapter{Qubits}



\section{Bloch Sphere}


\highlightdef{A circle has one degree of freedom, a sphere has \textit{two degrees of freedom}}

The \textit{area} of a circle has two degrees of freedom. 
We can therefore find a mapping between points on the \textit{surface} of the Bloch Sphere and points 
within an \textit{area} of two dimensional unit circle. 

Every point on 



\highlightdef{\textbf{Sphere Equation}: 
$\ket{\psi} = \text{cos}(\frac{\theta}{2})\ket{0} 
+ e^{i \phi} \text{sin}(\frac{\theta}{2})\ket{1}$ }


The $\frac{\theta}{2}$ converts the \textit{angle associcated with perpendicularity}. 
$\ket{0}$ and $\ket{1}$ are always orthogonal. Geometrically, we are used to using 90 degrees
but, \textit{on the block sphere}, we use 180 degrees to express perpendicularity. 
It turns out that in four dimensional space, 180 degrees is gives a more natural 
angle for perpendicularity. The same idea comes up when dealing with \textit{Quaternions}. 
Quaternions are a useful mathematical abstraction for expressing rotations 
using a four dimensional space.


\begin{example}
Plot the following qubit on the Bloch Sphere $\frac{1}{\sqrt{2}}(\ket{0} + i\ket{1})$

We have  $\text{cos}(\frac{\theta}{2}) = \text{sin}(\frac{\theta}{2})$
which only occurs when $\frac{\theta}{2} = 45$. Hence $\theta = 90$.
And solving  $e^{i\phi} = i$ gives $\phi = 90$.
\end{example}

\begin{example}
Plot the following qubit on the Bloch Sphere $\frac{1}{\sqrt{2}}(\ket{0} - i\ket{1})$

We have  $\text{cos}(\frac{\theta}{2}) = \text{sin}(\frac{\theta}{2})$
which only occurs when $\frac{\theta}{2} = 45$. Hence $\theta = 90$.
And solving  $e^{i\phi} = -i$ gives $\phi = 270$.
\end{example}

\section{Density Matrices}

Previously, in order to find a point on the Bloch sphere, 
we had to solve simulatenous equations to find angles $\theta$ and 
$\phi$. We look at a new way to find points on a Bloch sphere 
by introducing the \textit{Density Matrix} of a state, 
and the concept of a \textit{Bloch vector}. 


\frmrule 

\highlightdef{$\rho = \frac{1}{2}(I + r_xX + r_yY + r_zZ)$}
where $X,Y,Z$ are the Pauli matrices. 

The vector $(r_x,r_y,r_z)$ is called the Bloch vector. 
The elements are real numbers and locate a qubit on the Bloch sphere. 
They can be determined from the density matrix of a state using:
$r_x = 2Re[\rho_{10}]$
$r_y = 2Im[\rho_{10}]$
$r_z = \rho_{00} - \rho_{11}$

\frmrule 

\begin{example}
Locate on the Bloch sphere, the location 
of the qubit $\frac{1}{2}(1+i, 1-i)$ by determining its density matrix.
\end{example}

Rather than finding  $\rho$, we can just directly use 
the following equations 

$r_x = 2Re[\alpha^{*}\beta]$
$r_y = 2Im[\alpha^{*}\beta]$
$r_z = |\alpha|^2 - |\beta|^2$	

\section{Measurement}

Albert Einstein once said:
\begin{quote}
I think that matter must have a 
separate reality indepenent of the measurements. 
That is an electron has spin, location and so forth even when 
it is not being measured, I like to think that the moon is there even if 
I am not looking at it.
\end{quote}

A quantum state is in a \textit{superposition state} when 
one is not looking at it. And when we look at the state 
or \textit{measure} does a different system form with 
physical quantities of interest. 


Erwin Schr\"{o}dinger's cat: 
$\ket{\text{cat}} = \frac{1}{\sqrt{2}} (\ket{\text{dead}} + \ket{\text{alive}})$