
\chapter{Further Topics}





\section{Bell States}


Bell states are \textit{rotationally invariant}.
Given $\ket{\psi} = \frac{1}{\sqrt{2}} [\ket{uu} + \ket{u^{\perp}u^{\perp}}]$ 
and a rotation of the bit basis:
$\ket{u} = a\ket{0} + b\ket{1}$ with $\ket{u^{\perp}} = -b\ket{0} + a\ket{1}$


\[ \begin{array}{ll}
\ket{\psi}
& =
\frac{1}{\sqrt{2}} [\ket{uu} + \ket{u^{\perp}u^{\perp}}] \\
& =
\frac{1}{\sqrt{2}} [(a\ket{0} + b\ket{1})(a\ket{0} + b\ket{1}) 
+ (-b\ket{0} + a\ket{1})(-b\ket{0} + a\ket{1})]  \\ 
& =
\frac{1}{\sqrt{2}} [(a^2 + b^2)\ket{00} + (a^2 + b^2)\ket{11} \\ 
& =
\frac{1}{\sqrt{2}} [\ket{00} + \ket{11}] \\ 
\end{array}\] 

\frmrule

The previous bell state is only rotationlly invariant for real valued rotations. 
It is not invariant for complex rotations. However the bell state, 
psi-minus, $\ket{\psi^{-}} = \frac{1}{\sqrt{2}} [\ket{01} - \ket{10}]$
is invariant under all complex rotations.

\frmrule

\begin{example}
Prove that the bell state $\ket{\psi^{-}}$ is invariant under all complex rotations.
\end{example}

\frmrule

\highlightdef{All four Bell States are \textit{entangled}}



\section{Quantum Teleportation}

\begin{example}
In order for quantum teleportation to work, Alice and Bob must share .........
\end{example}



\section{Super-dense Coding}

Super dense coding is the less popular sibling of teleportation. It can actually be viewed as the
process in reverse. The idea is to send two classical bits of information by only sending one
quantum bit. The process starts out with an EPR pair that is shared between the receiver and
sender (the sender has one half and the receiver has the other).




\section{Qutrits and Ququads}

\textit{Qutrits} are systems with 3 degrees of freedom. 
These are represented by vectors in $\mathbb{C}^3$ 
using base vectors $\ket{0}$, $\ket{1}$, and $\ket{2}$. 

\textit{Qutrits} are systems with 4 degrees of freedom. 
These are represented by vectors in $\mathbb{C}^4$ 
using base vectors $\ket{0}$, $\ket{1}$, $\ket{2}$ and $\ket{2}$.  

\frmrule

\begin{example}
Give the coordinates of the base vectors $\ket{0}$, $\ket{1}$, $\ket{2}$ in $\mathbb{C}^2$ 
\end{example} 

\frmrule

\begin{example}
Give the coordinates of the 2-qutrit states: $\ket{00}$, $\ket{11}$, $\ket{22}$. 
\end{example} 

\frmrule

\begin{example}
Show that $\ket{\phi} = \frac{1}{\sqrt{3}}[\ket{000}$, $\ket{111}$, $\ket{222}]$ is entangled.
\end{example}

\frmrule 

\begin{example}
Describe teleportation for a ququad. 
Sketch the appropriate quantum circuit and calculate the 
state before the measurement (the corrections needed can be ignored).
\end{example} 



\section{Quantum Cryptography}


\section{Quantum Error Correction}



\highlightdef{\textbf{3-Qubit Bit-Flip Code}: }

We need to keep in mind that we have to measure to determine if there was an error.






 \begin{tikzpicture}[thick]
    % `operator' will only be used by Hadamard (H) gates here.
    % `phase' is used for controlled phase gates (dots).
    % `surround' is used for the background box.
    \tikzstyle{operator} = [draw,fill=white,minimum size=1.5em] 
    \tikzstyle{phase} = [draw,fill,shape=circle,minimum size=5pt,inner sep=0pt]
    \tikzstyle{control} = [draw,shape=circle,minimum size=5pt,inner sep=0pt]
    \tikzstyle{surround} = [fill=blue!10,thick,draw=black,rounded corners=2mm]
    %
    \matrix[row sep=0.4cm, column sep=0.8cm] (circuit) {
    % First row.
    \node (q1) {\ket{\phi}}; &
    \node[phase] (p1) {}; &
    \node[phase] (p2) {}; &
    \coordinate (end1); \\
    % Second row.
    %
    \node (q2) {\ket{0}}; &
    \node[control] (c1) {}; &
    &
    \coordinate (end2);\\
    % Third row.
    %
    \node (q3) {\ket{0}}; &
    &
    \node[control] (c2) {}; &
    \coordinate (end3); \\
    };
    % Draw bracket on right with resultant state.
    \draw[decorate,decoration={brace},thick]
        ($(circuit.north east)-(0cm,0.3cm)$)
        to node[midway,right] (bracket) {$\ket{\phi'}$}
        ($(circuit.south east)+(0cm,0.3cm)$);
    \begin{pgfonlayer}{background}
        % Draw lines.
        \draw[thick] (q1) -- (end1)  (q2) -- (end2) (q3) -- (end3) (p1) -- (c1) (p2) -- (c2);
        % Draw control vertical lines.
        \draw[thick] (c1.north) -- (c1.south);
        \draw[thick] (c2.north) -- (c2.south);


    \end{pgfonlayer}
    %
\end{tikzpicture}




The initial state $\ket{\phi} = \alpha \ket{0} + \beta \ket{1}$ is converted by
the circuit into $\ket{\phi'} = \alpha \ket{000} + \beta \ket{111}$.
Then each qubit of $\ket{000}$ is sent independently through the bit flip channel.
Each qubit of $\ket{111}$ is sent independently through the bit flip channel. 
This gives us $\ket{\phi''}$.

We then apply a two-stage error correction method. 
We first make a measurement using the four measurement
operators. Each measurement operator is gives a basis for the errors. 

These operators are called \textit{error syndromes}.
Depending on which error syndrome is measured we then recover the original
state.

$M_0 = \ket{000}\bra{000} + \ket{111}\bra{111}$ \\
$M_1 = \ket{100}\bra{100} + \ket{011}\bra{011}$ \\
$M_2 = \ket{010}\bra{010} + \ket{101}\bra{101}$ \\
$M_3 = \ket{001}\bra{001} + \ket{110}\bra{110}$ \\



\begin{example}
Suppose, for example, that a bit flip occurred, on the second qubit.
Then $\ket{\phi} = \alpha \ket{0} + \beta \ket{1}$ is converted by
the circuit into $\ket{\phi'} = \alpha \ket{000} + \beta \ket{111}$.
Then $\ket{\phi'} = \alpha \ket{000} + \beta \ket{111}$ is converted to 
$\ket{\phi''} = \alpha \ket{010} + \beta \ket{101}$. 

Now we determine $p(2) = \bra{\phi''} M^{H}_{2} M_{2} \ket{\phi''} = ...$

\end{example}

\frameans{}{$1$}

So it is \textit{certain} that $j = 2$. 

The state of the system will not change $M_2 \ket{\phi''} = \ket{\phi''}$
Therefore, we finally flip the second qubit two to recover $\ket{\phi'}$.

\frmrule

\begin{example}
Show that these four measurement operators 
satisfy the \textit{General Measurement Principle}: $\sum^{3}_{j=0} M^{H}_{j} M_{j} = I$
\end{example}

\frmrule

\begin{example}
Classical error correction uses repeated copies of the input bit(s); 
the no-cloning theorem forbids copying of qubits. 
Explain why the circuit for the 3-qubit bit-flip code does 
not violate the no-cloning theorem.
\end{example}


\frmrule

\begin{example}
Sketch how one can, in general, use an encoding circuit $U_{enc}$ (like 
the 3-qubit bit-flip code circuit above) to recover a distorted qubit. 
\end{example}

\frmrule

\begin{example}
Sketch how one can, in general, use an encoding circuit $U_{enc}$ (like 
the 3-qubit bit-flip code circuit above) to recover a distorted qubit. 
\end{example}

\frmrule

\begin{example}
Assume that at most one 1-qubit error occurs on a codeword for the 3-qubit 
bit-flip code. How can one recover the original state? Which recovery 
operation(s) are needed? 

State all the intermediate states during an execution of the complete 
correction circuit encoding $\ket{0}$ for, (i) when no errors occur
(ii) all cases where exactly one of the qubits flips.
\end{example}

\frmrule

\begin{example}
Sketch Shor's 9-qubit code circuit and calculate its output for general qubits
\end{example}




\section{Teleportation Quantum Computation}


\section{One-way Quantum Computation}


One-way Quantum Computing (1WQC) is a alternative method of thinking about 
implementing quantum algorithms using \textit{graphs} rather than quantum circuits. 
Let us show the contrast between the circuit scheme of quantum computation 
and this new scheme that uses graphs. 

Quantum circuits have a number of qubits which are initially in a
product state and then are subjected to various one and two-qubit gates before 
being measured.


We can imagine the qubits as nodes of a graph and the graph has edges between 
two nodes iff the two qubits are entabled. 
In fact, we can have edges from many nodes to the same node (and vice-versa).
These are special \textit{multi-qubit} entangled state. 
The graph itself is a special multi-qubit entangled state. 


Then \textit{measure} qubits individually. We measure qubits one-by-one. 
Each measurement causes the state to be less entangled. 
It turns out that via a suitable choice of measurements, any 
quantum computation can be acheived using this process. 
However the result of each measurement is random, so we need to post-process or \textit{correct}, 
the state so that we always get the exact computation we desire (regardless of which state 
a measurement could collapse to). 


We treat entanglement as a limited resource (like space on a classical computer). 
We call the first state, the \textit{resource state}. 
It is called \textit{one-way} because the resource state 
is destroyed by the measurements.
 

\frmrule 

\highlightdef{Each measurement, $M_j$, is of the form: $\cos(\theta_j)\ket{0} + (-1)^{s_j}\sin(\theta_j)\ket{1}$}
where $s_j \in \{0,1\}$.

\frmrule 

\begin{example}
The following are Pauli measurements. For each state, $\theta_j$ and $s_j$. \\
\textbf{(a)} $\ket{+}, \ket{-}$\\
\textbf{(b)} $\frac{1}{\sqrt{2}}(\ket{0} + i\ket{1})$, $\frac{1}{\sqrt{2}}(i\ket{0} + \ket{1})$ \\
\textbf{(c)} $\ket{0}, \ket{1}$\\
\end{example}

\frmrule 



\section{Measurement Calculus I}


The measurement calculus is a formalisation of \textit{Measurement-Based Quantum Computing} (MBQC)
and \textit{one-way quantum computing} (1WQC). 
The measurement calculus formally describes these by using a \textit{sequence of commands}. 
Each command is either \textit{preparation}, \textit{entanglement}, \textit{measurement}, or 
\textit{correction}. Each command is denoted with a capital letter (namely $N$, $E$, $M$, $X$, $Z$).
Each command uses indicies $i,j \in \{1,...,n\}$ indicate qubits to which the operations apply.  

We informally describe the commands using in measurement calculus. 

\begin{itemize}
\item \textbf{N - Prepare $i$th Qubit} The \textit{preparation command} $N_i$ prepares 
the $i$th qubit. 
\item \textbf{E - Entangle $i$th with $j$th Qubit} The \textit{entanglement command}, $E_{i,j}$ 
entangles qubits numbered $i$ and $j$. The entangled state is $\ket{+} = C(Z_{i,j})$. 
That is, a controlled $Z_{i,j}$ gate, where $Z_{i,j}$ is the matrix 
with $z_{ii} = 1$, $z_{ii} = 1$,
\item \textbf{M - Measure $i$th Qubit} The \textit{measurement command}, $M_{i}$ 
measures the $i$th qubit. The measurement uses a specific basis that we will 
look at later. We will later extend the measurement command to specify this basis.
\item \textbf{X, Z - Correct $i$th Qubit} The \textit{correction commands}, $X_{i}$ 
and $Z_{i}$ are designed to correct the state of the system after qubits have been measured. 
The $X_i$ command corrects the $i$th qubit by multiplication by the Pauli matrix $X$. 
Similarly, $Z_i$ command corrects the $i$th qubit by multiplication by the Pauli matrix $Z$. 
These corrections are often called \textit{Pauli corrections} because of the use of Pauli 
matrices $Z$ and $X$.
\end{itemize}

\frmrule

\highlightdef{Commands are read \textit{right-to-left}}
So if we have commands $ABCD$, it is command $D$ that is done first. 

\frmrule

\begin{example}
Complete the following:\\
(a) The command(s) that act on one qubit are: ...................\\
(b) The command(s) that act on two qubits are: ...................
\end{example}

\frmrule

We now define a \textit{pattern}. 
Informally, a pattern consists of three things. Firstly, we need \textit{qubits}. 
These will act as our resouce for computation. Secondly, 
we have a \textit{command sequence}. This is a sequence of commands 
that we previously described (\textit{preparation}, \textit{entanglement}, 
\textit{measurement}, or \textit{correction}). They specify precisely how to 
perform the computation on our qubits.  Thirdly, we need \textit{rules}. 
We cannot just use any combination of commands. We need specific rules 
about what order commands can appear in sequences.


\frmrule

In general qubit $i$ has a lifetime consisting of (1)
\textit{preparation}, then (2) \textit{entanglement}, 
then (3) \textit{measurement}, and finally \textit{correction}. 
We want form rules so that our commands give qubits this 
lifetime. 

\begin{tikzpicture}
\node[align=center] (a) {prepare\\$N_i$}; 
\node[align=center, right=1cm of a] (b) {entangle\\$E_{i,j}$}; 
\node[align=center, right=1cm of b] (c) {measure\\$M_i$}; 
\node[align=center, right=1cm of c] (d) {correct\\$X_i$, $Z_i$}; 
\draw[->] (a) -- (b); \draw[->] (b) -- (c); \draw[->] (c) -- (d);
\end{tikzpicture}

Furthermore, input qubits are output qubits have slightly 
different lifetimes and so our rules need to capture these differences. 
In particular, output qubits are never measured. Input/intermediate 
qubits must always be measured. 

\begin{tikzpicture}
\node[align=center] (input) {\textbf{Input Qubits}};
\node[align=center, right=1cm of input] (inputa) {prepare\\$N_i$}; 
\node[align=center, right=1cm of inputa] (inputb) {entangle\\$E_{i,j}$}; 
\node[align=center, right=1cm of inputb] (inputc) {measure\\$M_i$}; 
\node[align=center, right=1cm of inputc] (inputd) {correct\\$X_i$, $Z_i$}; 
\draw[->] (inputa) -- (inputb); 
\draw[->] (inputb) -- (inputc); 
\draw[->] (inputc) -- (inputd);
%
\node[align=center, below = 0.5cm of input, text width=2cm] (int) {\textbf{Intermediate Qubits}};
\node[align=center, right=1cm of int] (inta) {prepare\\$N_i$}; 
\node[align=center, right=1cm of inta] (intb) {entangle\\$E_{i,j}$}; 
\node[align=center, right=1cm of intb] (intc) {measure\\$M_i$}; 
\node[align=center, right=1cm of intc] (intd) {correct\\$X_i$, $Z_i$}; 
\draw[->] (inta) -- (intb); 
\draw[->] (intb) -- (intc); 
\draw[->] (intc) -- (intd);
%
\node[align=center, below = 0.5cm of int] (out) {\textbf{Output Qubits}};
\node[align=center, right=1cm of out] (outa) {prepare\\$N_i$}; 
\node[align=center, right=1cm of outa] (outb) {entangle\\$E_{i,j}$}; 
\node[align=center, right=1cm of outb] (outc) {measure\\$M_i$}; 
\node[align=center, right=1cm of outc] (outd) {correct\\$X_i$, $Z_i$}; 
\draw[->] (outa) -- (outb); 
\draw[->] (outb) -- (outc); 
\draw[->] (outc) -- (outd);
\end{tikzpicture}

We specify the following rules for command sequences. Each command sequence 
must obey the following rules to be valid for a pattern.
\begin{itemize}
\item \textbf{D0} - no command depends on an outcome not yet measured
\item \textbf{D1} - no command acts on a qubit already measured
\item \textbf{D2} - no command acts on a qubit not yet prepared, except if is an input qubit
\item \textbf{D3} - a qubit $i$ is measured iff it is not an output qubit (input/intermediate)
\end{itemize}

\frmrule

\begin{example}
Assume we have a pattern $\mathcal{P}$ where 1 is an input qubit, and 
2 is an intermediate qubit. \\
Explain why the following command sequence is \textit{invalid}. 
$$A = X_1 M_1 E_{1,2} $$

\frameans{Look at rules D0, D1, D2, D3 to see which is violated}{D2 violated. $E_{1,2}$ acts on unprepared intermediate qubit: qubit 2}


\end{example}

\frmrule

\begin{example}
Assume we have a pattern $\mathcal{P}$ where 1 is an input qubit, and 
2 is an output qubit. \\
For the following command sequences explain why they are \textit{invalid}.\\ 
(a) $M_1 X_1 E_{1,2} N_2$ \\
(b) $M_2 M_1 N_2$ \\
(c) $N_2$ \\
\end{example}

\frameans{Look at rules D0, D1, D2, D3 to see which is violated}
{
(a) D0 violated. $X_{1}$ depends on qubit 1 but was not measured earlier in sequence \\
(b) D3 violated. output qubit is measured\\
(c) D3 violated. input qubit is never measured
}

\frmrule

We can also combine patterns using two methods, the \textit{Composite Pattern} 
and the \textit{Tensor Pattern}.  
\begin{itemize}
\item \textbf{Composite Pattern} $\mathcal{P}_1 \mathcal{P}_2$
\item \textbf{Tensor Pattern} $\mathcal{P}_1 \otimes \mathcal{P}_2$
\end{itemize}


\section{Measurement Calculus II}

If there are $m$ measurements then there are $2^m$ different 
executions. Each execution is described by a binary string $s$ that has $m$ bits. 


\highlightdef{\textbf{Domain}: The \textit{domain} of a signal is the set of qubits on which it depends}
The domain of the signals of a dependent command (measurement or correction), represents
the set of measurements which one has to do before one can determine
the actual value of the command.


\begin{example}
Describe the \textit{standard form} (or \textit{normal form}) 
of a pattern in the Measurement Calculus. Explain its parts. 
\end{example}

\frmrule

\begin{example}
Transform the pattern $X^{s_2}_3 X^{x}_2 E_{23} X^{s_1}_2 M^{-\alpha}_1 E_{12}$ into the \textit{standard form}. 
\end{example}


\section{Measurement Calculus III}

\highlightdef{$\mathbb{Z}^W_2$ denotes all total functions from $W$ to $\mathbb{Z}_2$}
