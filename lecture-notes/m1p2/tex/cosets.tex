
\chapter{Cosets}



\section{Introducing Cosets}



\highlightdef{$(a\cdot)$ is a function from $H$ to some set $Y$ }

\highlightdef{\textbf{Coset}: $aH$ is the range of $(a\cdot): H \rightarrow Y$ }

A coset is simply the range of a function. Note that it is a \textit{set}, 
not a group. Hence why we say co\textit{set}. 
By definition, it is the range of a function i.e. the \textit{codomain's set}, 
which can be shortened to \textit{coset}. 
We can think of $aH = $ map $(a\cdot) \; H$ which is why we have the notation 
$aH$. It's like taking the set $H$ and mapping $(a\cdot)$ to all the elements 
in $H$. Typically the set would shrink in the case where two inputs to $(a\cdot)$ 
give the same output. But it turns out that, for $(a\cdot)$, this \textit{never} happens.  

\highlightdef{The function $(a\cdot):H \rightarrow Y$ is a \textit{bijection}}

And so by the Pidegon-hole principle, $|Y| = |H|$, the size of the domain is the same 
as the size of the codomain. Furthermore, the range and codomain are equal, 
so $Y = aH$. To show that this function is a bijection, 
we need to show that it is surjective and injective. 

\begin{itemize}
\item \textbf{Surjective} (onto)
\item \textbf{Injective} (one-to-one)
\end{itemize}

One subtle question arrises from the proof of $(a \cdot)$ being surjective.
And that is: how did we know that $a^{-1}b \in H$? What if some other element 
was in $H$ was the input to $(a \cdot)$ that gave $b$? 


\frmrule



\section{Coset Membership Relation}

The question that naturally arises is \textit{which elements belong to which coset}?
For a group with $n$ elements, we need to think about $n^2$ possible cases. 
Each case asks: is $a$ in $bH$? In order to answer these $n^2$ questions, 
we define relation $R$ on $G$. This is known as the \textit{coset membership relation}.
Intuitively $aRb$ iff $a$ is in $b$'s coset (or more precisely $(b\cdot)$'s coset). 

\highlightdef{\textbf{Coset Membership Relation}: $aRb$ iff $a \in bH$}


\highlightdef{Every element of G is in some coset since $g \in gH$.}




\highlightdef{$aRb$ iff $b^{-1}a \in H$}


\highlightdef{$aRb$ is an equivalence relation}


\begin{itemize}
\item \textbf{Reflexive}: $e \in H$ since $H$ is a subgroup. 
And so when we perform the mapping $(a\cdot)(e) = a$. So $a$ is in the range of $(a\cdot)$, $a \in aH$. 
Hence $aRa$.
\item \textbf{Symmetric}:  Assume $aRb$
$b^{-1}a \in H$. Since $H$ is a group, the inverse element exists in $H$. 
So $(b^{-1}a)^{-1} \in H$. That is $a^{-1}b \in H$. 
In other words, $bRa$.
\item \textbf{Transitive}: Assume $aRb$ and $bRc$. 
Since $aRb$, $a$ is in the range of $(b\cdot)$. 
By undoing the left multiplication of $b$, we get $b^{-1}a \in H$. 
Similarly since $bRc$, we undo multiplication of $c$ to recover the 
corrosponding element in the domain, $c^{-1}b \in H$. 
Since $H$ is closed, $(c^{-1}b)(b^{-1}a) \in H$. That is, $c^{-1}a \in H$. In other words, 
$aRc$. 
\end{itemize}

\frmrule

\begin{example}
For the following subgroup,\\
(a) Compute the cosets \\
(b) Give the coset relation $R$
\end{example}

\frmrule

\begin{example}
For the given coset relation $R$, give the cosets $aH, bH, cH, dH$.
\end{example}

\begin{example}
Suppose we know coset $aH = \{a,b,c,d\}$. \\
What can we say about the cosets that $a$ is in?
\end{example}

\frameans{}{We know, at the very least, $a \in aH,bH,cH,dH$ }

The question gives us that $aRa, bRa, cRa, dRa$. Since $R$ is an equivalence 
relation, it is \textit{symmetric}, and so we have $aRa, aRb, aRc, aRd$. 
That is $a \in aH,bH,cH,dH$, by def. of $H$. 

\frmrule

\highlightdef{$aRb$ iff $aH = bH$}
Equality of cosets is given precisely by membership in just one direction. 
That is, an element $a$ is in coset $bH$ if and only if 
the cosets $aH$ and $bH$ are equal. 

\begin{itemize}
\item ($\Rightarrow$) Assume $aRb$. To show $aH = bH$\\
$x \in aH \Rightarrow xRa \Rightarrow xRb \Rightarrow x \in bH$\\
$x \in bH \Rightarrow xRb \Rightarrow bRx \Rightarrow aRx \Rightarrow xRa \Rightarrow x \in aH$
\item ($\Leftarrow$) Assume $aH = bH$. To show $aRb$. \\
$aRa \Rightarrow a \in aH \Rightarrow a \in bH \Rightarrow aRb$
\end{itemize}

\frmrule

\begin{example}
Suppose we know coset $aH = \{a,b,c,d\}$. \\
What can we say about cosets $bH, cH, dH$?
\end{example}

\frameans{}{$aH = bH = cH = dH$ }

The question gives us that $aRa, bRa, cRa, dRa$. 
Since $xRy$ iff $xH = yH$, we deduce that $aH = aH$, $bH = aH$, $cH = aH$ and $dH = aH$. 
In other words $aH = bH = cH = dH = \{a,b,c,d\}$. 

\frmrule

Let us consider the equivalence classes for the coset membership relation. 

$[a] = \{ x \;|\; aRx \}$ by def of an equivalence class \\
$ = \{ x \;|\; xRa \}$ reflexivity \\
$ = \{ x \;|\; xa^{-1} \in H \}$ property of coset membership relation \\
$ = \{ ya \in G \;|\; y \in H \}$  let $y = xa^{-1}$ \\
$ = Ha$ 

What should now be clear is that all the cosets form a \textit{partition} of $G$. 
This result follows from the fact that $R$ is an equivalence relation. 
Take some equivalence class $[a]$. This is all the elements that are related to 
each other under $R$. Say if $[a] = \{a,b,c,d,e\}$, then 
$aRb, bRc, cRd, eRa$ must be true. But coset equality is precisely being related 
in one direction, so $aH = bH$, $bH = cH$, $cH = dH$, $eH = aH$. 
Each equivalence class gives one unique coset. 
An element can't be in two equivalence classes, and 
every element of $G$ belongs to some equivalence class. 
So the classes partition $G$. 
Given that 
each class gives one unique coset,
we conclude that the cosets partition $G$. 

\frmrule

An intersting property of the coset membership relation is that it is 
\textit{compatible with left multiplication}. That is:

\highlightdef{\textbf{Compatible w/left multiplication}: $aLb$ iff $(ca)L(cb)$}


$aLb$  \\
$\Leftrightarrow$ $b^{-1}a \in H$. By property of coset membership relation\\
$\Leftrightarrow$ $b^{-1}c^{-1}ca \in H$ By algebra \\
$\Leftrightarrow$ $(cb)^{-1}(ca) \in H$ By algebra \\
$\Leftrightarrow$ $(ca)L(cb)$. By property of coset membership relation\\


\section{Left vs Right Cosets}


\highlightdef{
\textbf{Left}: $a L b$ iff $a \in bH$\\
\textbf{Right}: $a R b$ iff $a \in Hb$
}

Anything done on the left holds with symmetric on the right 
so long as we take inverses. 
\highlightdef{
$a L b$ iff $a^{-1} R b^{-1}$ 
}

Every element has exactly one inverse element. 
So clearly, all the results for left cosets have the same result 
for right cosets. For a given subgroup, there is the same 
number of left cosets as there are right cosets. 

\frmrule

\begin{example}
Suppose we know the left coset $aH = \{a,b,c,d\}$. \\
What can we say about the \textit{right} cosets?
\end{example}

\frmrule

For any subset $S \subseteq G$, let $S^{-1} = \{ s^{-1} \;|\; s \in S\}$. 
What happens if we invert all the elements of a coset? 
Well take some coset $aH$, then \\
$(aH)^{-1} = \{ay \;|\; y \in h\}$ by def of coset \\
$= \{(ay)^{-1} \;|\; y \in h\}$ by def of $S^{-1}$\\
$= \{y^{-1}a^{-1} \;|\; y \in h\}$ by algebra\\
$= \{y a^{-1} \;|\; y^{-1} \in h\}$ same expression\\
$= \{y a^{-1} \;|\; y \in h\}$ subgroups closed under inversion\\
$= Ha^{-1}$  by def of coset\\

Similarly, $(Ha^{-1})^{-1} = ....$

\frameans{}{$(aH)^{-1}$ }

\highlightdef{
The function $f(S) = S^{-1}$ is a bijecion from $G/H$ to $H\backslash G$
}

\highlightdef{
The function $f = aH \mapsto Ha^{-1}$ is a bijecion from $G/H$ to $H\backslash G$
}

We introduce some new notation here. 
$G/H$ is the denotes a set of cosets (a set of sets). 
In particular $G/H$ is the set of \textit{left} cosets. 
Similarly, $H\backslash G$ is the set of \textit{right} cosets. 

\highlightdef{
\textbf{Set of Left Cosets}: $G/H$ \\
\textbf{Set of Right Cosets}: $H\backslash G$
}
These sets of sets are sometimes called coset \textit{spaces}. 
The slash notation should remind you of $S/R$ when we looked at equivalence 
classes in set theory. Recall that $S/R$ denotes a paritition of $S$
induced by equivalence relation $R$. Here, instead of $S/R$, 
we use $G/H$, but the idea is the same. We have a paritition of $G$ 
induced by the coset membership relation $L_H \equiv$ [$a L b$ iff $a \in bH$]. 
But rather then writing $G/L_H$, it's clearer to write $G/H$. 
For group theory it's nicer to emphasise what subgroup is in question.
Besides, when we are given the subgroup $H$, 
we know what equivalence relation gives the partition (given $H$, 
we know that the relation is $L_H$). 

\frmrule

\begin{example}
For the following $G$, $H$:\\
(a) Find $G/H$ \\
(b) Find $H\backslash G$ \\
(c) Find a bijection from $G/H$ to $H\backslash G$
\end{example}

\frmrule


\section{Double Cosets}


\highlightdef{$(\cdot a \cdot)$ is a function from $H \times K$ to some set $Y$}

\highlightdef{\textbf{Double Coset}: $HaK$ is a the range of $(\cdot a \cdot)$}

Just as we previously defined the \textit{coset membership relation}, we define:
\highlightdef{\textbf{Double Coset Membership Relation}: $aDb$ iff $a \in HbK$}
And it turns out that this relation is an equivalence relation.



And so double cosets partition $G$. 

\frmrule

Note that ($H$, $\{e\}$)-double cosets are just the right cosets of $H$. 
And($\{e\}$, $K$)-double cosets are just the left cosets of $K$. 

\frmrule

Recall that cosets of a subgroup all have the same cardinality equal to 
the subgroups cardinality. This is not the case for double cosets. 

\highlightdef{Double cosets do not necessarily have the same cardinality}

So the sets inside $H\backslash G/K$ do not have to have the same size.
It turns out that when a subgroup is a \textit{normal subgroup}, then 
it is indeed the case that double cosets have the same cardinality. 
In fact, subgroup is a \textit{normal subgroup}, we can find a 
bijection from $H \backslash G$ to $H\backslash G/K$, and a bijection from 
to $G/K$ to $H\backslash G/K$. So not only will double cosets have the same size as left/right
cosets, but there will be the same number of left/right cosets as double cosets. 



\section{Normal Subgroups}

Let $f:G/H \rightarrow H/G\backslash H$ be a function from \textit{$H$ left cosets} to \textit{$(H,H)$ double cosets}. 
\highlightdef{
\textbf{Normal Subgroup}: $H$ is \textit{normal} when $f = aH \mapsto HaH$ is a bijection
}
where $aH$ is the codomain of $(\cdot a): H \rightarrow aH$ \\
and $HaH$ is the codomain of $(\cdot a \cdot): (H \times H) \rightarrow HaH$ 

Let us prove the following alternate definition of a normal subgroup is equivalent 
to the first.
\highlightdef{
$H$ is \textit{normal} iff [$h \in H \Rightarrow aha^{-1} \in H$]
}
\begin{itemize}
\item $(\Rightarrow)$  Assume not [$h \in H \Rightarrow aha^{-1} \in H$]. \\
Then there is some $h \in H$ where $aha^{-1} \notin H$. \\
So $ah \notin Ha$. \\
By reflexivity of coset membership, $ah \in ahH$. \\
Since an element is in $ahH$ but not in $Ha$ we have $Ha \neq ahH$. \\
Now $HaH = H(ah)H  $. \\
Hence $f(aH) = f(haH)$ but $aH \neq haH$. So $f$ cannot be an injection.\\ 
So $f$ cannot be bijective. So $H$ is not normal.

\item $(\Leftarrow)$  Assume [$h \in H \Rightarrow aha^{-1} \in H$]. \\
We will show that $f$ is a bijection.
\begin{itemize}
\item Injection: Assume $f(aH) = f(bH)$. Then $HaH = HbH$
Take any $x \in aH$. Then $a^{-1}x \in H$. 
So by assumption, $aa^{-1}xa^{-1} \in H$.  That is, $xa^{-1} \in H$
\item Surjection:
\end{itemize}
\end{itemize}

\frmrule 

\highlightdef{
If $H$ is \textit{normal} then $f = Ha \mapsto HaH$ is also a bijection
}


\frmrule 

Fix $a$. Then the following gives us a condition for equality of left and right cosets. 
\highlightdef{
$aH = Ha$ iff [$h \in H$ iff ($h \in H$ and $aha^{-1} \in H$)]
}

\begin{itemize}
\item ($\Rightarrow$) Assume $aH = Ha$. \\
$h \in H$ $\Leftrightarrow$
$a^{-1} a h a^{-1} a \in H$ $\Leftrightarrow$
$a h a^{-1} a \in aH$ $\Leftrightarrow$
$a h a^{-1} a \in Ha$ $\Leftrightarrow$
$a h a^{-1} \in H$ 


\item ($\Leftarrow$) Assume [$h \in H$ $\rightarrow$ ($h \in H$ iff $aha^{-1} \in H$]. \\
$x \in aH$ $\Leftrightarrow$
$a^{-1} x \in H$ $\Leftrightarrow$
$a a^{-1} x a^{-1} \in H$ $\Leftrightarrow$
$x a^{-1} \in H$ $\Leftrightarrow$
$x \in Ha$


\end{itemize}


\highlightdef{$aH = Ha$ iff [$a^{-1}x \in H$ iff $xa^{-1} \in H$]}
Equality of left and right cosets for $a$ means $a^{-1}$ has \textit{commutative membership} with elements in $H$


\section{Quotient Groups}