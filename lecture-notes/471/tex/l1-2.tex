
\chapter{$\mathcal{L}_1$ Language II}



%--------------------------------------------------------------------------------------------%
%--------------------------------------------------------------------------------------------%
%--------------------------------------------------------------------------------------------%

\section{Operational Semantics III}

%--------------------------------------------------------------------------------------------%
%--------------------------------------------------------------------------------------------%
%--------------------------------------------------------------------------------------------%

In the final section for operational semantics, we look at errors, stuck 
execution and exception propagation. 

Deviations (dev) indicate abnormal termination due to null-pointer de-referencing,
or an evaluation being stuck. As we shall show, well-typed expressions are
never stuck, although they may throw null pointer exceptions.

\highlightdef{\textbf{Deviations}:
$dev = \{ \text{nullPtrExc}, \text{stuckErr} \}$ }



Trying to access the fields or methods of a null objects gives null. 

\begin{prooftree}
\def\defaultHypSeparation{\hskip .01in}
\AxiomC{$(e,\phi, \chi) \leadsto (\text{null}, \chi')$}
\LeftLabel{\textsc{null.field}}
\UnaryInfC{$(\text{e.f}, \phi,\chi) \leadsto (\text{nullPtrExc}, \chi')$}
\end{prooftree}

\begin{prooftree}
\def\defaultHypSeparation{\hskip .01in}
\AxiomC{$(e,\phi, \chi) \leadsto (\text{null}, \chi')$}
\LeftLabel{\textsc{null.assign}}
\UnaryInfC{$(e.f := e', \phi,\chi) \leadsto (\text{nullPtrExc}, \chi')$}
\end{prooftree}

\begin{prooftree}
\def\defaultHypSeparation{\hskip .01in}
\AxiomC{$(e,\phi, \chi) \leadsto (\text{null}, \chi')$}
\LeftLabel{\textsc{null.method}}
\UnaryInfC{$(e.m(e1), \phi,\chi) \leadsto (\text{nullPtrExc}, \chi')$}
\end{prooftree}

We can also get stuck during execution for various reasons. 
Our address for an object on the heap might give a mapping that is undefined.


\begin{prooftree}
\def\defaultHypSeparation{\hskip .01in}
\AxiomC{$(e,\phi, \chi) \leadsto (v, \chi')$}
\LeftLabel{\textsc{objundef-field}}
\RightLabel{
$\chi'(v)$ is undefined
}
\UnaryInfC{$(e.f, \phi,\chi) \leadsto (\text{stuckErr}, \chi')$}
\end{prooftree}

\begin{prooftree}
\def\defaultHypSeparation{\hskip .01in}
\AxiomC{$(e,\phi, \chi) \leadsto (v, \chi')$}
\LeftLabel{\textsc{objundef-assign}}
\RightLabel{
$\chi'(v)$ is undefined
}
\UnaryInfC{$(e.f := e', \phi,\chi) \leadsto (\text{stuckErr}, \chi')$}
\end{prooftree}

\begin{prooftree}
\def\defaultHypSeparation{\hskip .01in}
\AxiomC{$(e,\phi, \chi) \leadsto (v, \chi')$}
\LeftLabel{\textsc{objundef-method}}
\RightLabel{
$\begin{smallmatrix}
\chi'(v) \text{ is undefined} \\
v \neq \text{ null}
\end{smallmatrix}
$ 
}
\UnaryInfC{$(e.m(e_1) := e', \phi,\chi) \leadsto (\text{stuckErr}, \chi')$}
\end{prooftree}


We might true to lookup a field or method that is not defined for that class. 

\begin{prooftree}
\def\defaultHypSeparation{\hskip .01in}
\AxiomC{$(e,\phi, \chi) \leadsto (\iota, \chi')$}
\LeftLabel{\textsc{lookupundef-field}}
\RightLabel{
$\begin{smallmatrix}
\chi'(\iota) \downarrow_1 = \text{c} \\
F(P,c,f) \text{ is undef.}
\end{smallmatrix}
$ 
}
\UnaryInfC{$(e.f, \phi,\chi) \leadsto (\text{stuckErr}, \chi')$}
\end{prooftree}

\begin{prooftree}
\def\defaultHypSeparation{\hskip .01in}
\AxiomC{$(e,\phi, \chi) \leadsto (\iota, \chi')$}
\LeftLabel{\textsc{lookupundef-assign}}
\RightLabel{
$\begin{smallmatrix}
\chi'(\iota) \downarrow_1 = \text{c} \\
F(P,c,f) \text{ is undef.}
\end{smallmatrix}
$ 
}
\UnaryInfC{$(e.f := e', \phi,\chi) \leadsto (\text{stuckErr}, \chi')$}
\end{prooftree}

\begin{prooftree}
\def\defaultHypSeparation{\hskip .01in}
\AxiomC{$(e_0,\phi, \chi) \leadsto (\iota, \chi')$}
\AxiomC{$(e_1,\phi, \chi) \leadsto (v, \chi'')$}
\LeftLabel{\textsc{lookupundef-method}}
\RightLabel{
$\begin{smallmatrix}
\chi'(\iota) \downarrow_1 = \text{c} \\
M(P,c,m) \text{ is undef.}
\end{smallmatrix}
$ 
}
\BinaryInfC{$(e_0.m(e_1) := e', \phi,\chi) \leadsto (\text{stuckErr}, \chi'')$}
\end{prooftree}



Another reason is that our if statement might not have a boolean value.

\begin{prooftree}
\def\defaultHypSeparation{\hskip .01in}
\AxiomC{$(e,\phi, \chi) \leadsto (v, \chi')$}
\LeftLabel{\textsc{if-nobool}}
\RightLabel{
$\begin{smallmatrix}
v \neq \text{true} \\
v \neq \text{false}
\end{smallmatrix}$
}
\UnaryInfC{$(\text{if } e \text{ then } e_1 \text{ then } e_2, \phi,\chi) \leadsto (\text{stuckErr}, \chi')$}
\end{prooftree}


How do deviations propagate through expressions. 
Once a subexpression evaluates to a deviation, we want this result to 
propagate up through the expression tree to the top so that the overall result is the error. 

Every rule has a dev version that propagates the error upwards. 


\begin{prooftree}
\def\defaultHypSeparation{\hskip .01in}
\AxiomC{$(e,\phi, \chi) \leadsto (dv, \chi')$}
\LeftLabel{\textsc{dev-field}}
\UnaryInfC{$(e.f, \phi,\chi) \leadsto (dv, \chi')$}
\end{prooftree}

\begin{prooftree}
\def\defaultHypSeparation{\hskip .01in}
\AxiomC{$(e,\phi, \chi) \leadsto (dv, \chi')$}
\LeftLabel{\textsc{dev-assign1}}
\UnaryInfC{$(e.f := e', \phi, \chi) \leadsto (dv, \chi')$}
\end{prooftree}

\begin{prooftree}
\def\defaultHypSeparation{\hskip .01in}
\AxiomC{$(e,\phi, \chi) \leadsto (\iota, \chi')$}
\AxiomC{$(e',\phi, \chi) \leadsto (dv, \chi')$}
\LeftLabel{\textsc{dev-assign2}}
\BinaryInfC{$(e.f := e', \phi, \chi) \leadsto (dv, \chi')$}
\end{prooftree}

For methods, an error can occur in the object, in parameter evaluation 
or for the expression that represents the body of the method. 

\begin{prooftree}
\def\defaultHypSeparation{\hskip .01in}
\AxiomC{$(e_1,\phi, \chi) \leadsto (dv, \chi')$}
\LeftLabel{\textsc{dev-method1}}
\UnaryInfC{$(e_1.m(e_2), \phi, \chi) \leadsto (dv, \chi')$}
\end{prooftree}

\begin{prooftree}
\def\defaultHypSeparation{\hskip .01in}
\AxiomC{$(e_1,\phi, \chi) \leadsto (\iota, \chi')$}
\AxiomC{$(e_2,\phi, \chi') \leadsto (dv, \chi'')$}
\LeftLabel{\textsc{dev-method2}}
\RightLabel{$\chi'(\iota) \text{ is defined}$}
\BinaryInfC{$(e_1.m(e_2), \phi, \chi) \leadsto (dv, \chi'')$}
\end{prooftree}



\begin{prooftree}
\def\defaultHypSeparation{\hskip .01in}
\AxiomC{$(e_1,\phi, \chi_1) \leadsto (\iota, \chi_2)$}
\AxiomC{$(e_2,\phi, \chi_2) \leadsto (v, \chi_3)$}
\AxiomC{$(e_3,\phi, \chi_3) \leadsto (dev, \chi_4)$}
\LeftLabel{\textsc{dev-method3}}
\RightLabel{$\begin{smallmatrix}
                    \chi_3(\iota) \downarrow_1 = \text{c} \\
                    \mathcal{M}(\text{P},\text{c}, \text{m})\, = t_1 m (t_2 x) \{ e_3 \}
                  \end{smallmatrix}$}
\TrinaryInfC{$(\text{e}_1.\text{m}(\text{e}_2),\phi, \chi) \leadsto (v, \chi_4)$}
\end{prooftree}



\begin{prooftree}
\def\defaultHypSeparation{\hskip .01in}
\AxiomC{
$
\begin{smallmatrix}
(e,\phi, \chi) \leadsto (dv, \chi') \\
\text{or }[(e_1,\phi, \chi) \leadsto (\text{true}, \chi') \text{ and } (e_2,\phi, \chi') \leadsto (dv, \chi'')] \\
\text{or }[(e_1,\phi, \chi) \leadsto (\text{false}, \chi') \text{ and } (e_2,\phi, \chi') \leadsto (dv, \chi'')]
\end{smallmatrix}
$
}
\LeftLabel{\textsc{dev-if}}

\UnaryInfC{$(\text{if } e \text{ then } e_1 \text{ then } e_2, \phi,\chi) \leadsto (dv, \chi'')$}
\end{prooftree}


\frmrule

\begin{example}
What happens if two deviations occur? 
If an expression has a nullPtrError and a stuckErr, 
what determines which error is the overall result?
\end{example}

\frmrule

\highlightdef{\textbf{Strong Normalisation}: }




%--------------------------------------------------------------------------------------------%
%--------------------------------------------------------------------------------------------%
%--------------------------------------------------------------------------------------------%

\section{Proving Determinism}


\highlightdef{\textbf{Determinacy}: }


We will look at an example that motivates the need for equivalence \textit{up to address renaming}.

\frmrule 

\begin{example}
The execution of: $(\text{new Book}, \phi_0, \chi_0)$. 
\end{example}

Recall that \textit{new} can put a newly created object at any unused 
address on the heap. 

\frmrule 

We would execution of $\text{new Book}$ to do the same thing every time. 
It is simply adding an object to the heap. However the object can 
be in any location. Each time we rerun the program, we technically 
have a different heap. 

If we don't find some way to say these runs 
are the same, then every expression with a new inside, could 
potentially be different. However we can say these runs are the same when we introduce 
the notion of equivalent stacks and heaps \textit{up to address renaming}.


\begin{prooftree}
\def\defaultHypSeparation{\hskip .01in}
\AxiomC{}
\LeftLabel{$\alpha$-\textsc{addr}}
\UnaryInfC{$\iota =_{\alpha} \alpha(\iota)$}
\end{prooftree}

\begin{prooftree}
\def\defaultHypSeparation{\hskip .01in}
\AxiomC{}
\LeftLabel{$\alpha$-\textsc{true}}
\UnaryInfC{$\text{true} =_{\alpha} \text{true}$}
\end{prooftree}

\begin{prooftree}
\def\defaultHypSeparation{\hskip .01in}
\AxiomC{}
\LeftLabel{$\alpha$-\textsc{false}}
\UnaryInfC{$\text{false} =_{\alpha} \text{false}$}
\end{prooftree}

\begin{prooftree}
\def\defaultHypSeparation{\hskip .01in}
\AxiomC{}
\LeftLabel{$\alpha$-\textsc{null}}
\UnaryInfC{$\text{null} =_{\alpha} \text{null}$}
\end{prooftree}

\begin{prooftree}
\def\defaultHypSeparation{\hskip .01in}
\AxiomC{}
\LeftLabel{$\alpha$-\textsc{obj}}
\UnaryInfC{$\text{null} =_{\alpha} \text{null}$}
\end{prooftree}




%--------------------------------------------------------------------------------------------%
%--------------------------------------------------------------------------------------------%
%--------------------------------------------------------------------------------------------%


%--------------------------------------------------------------------------------------------%
%--------------------------------------------------------------------------------------------%
%--------------------------------------------------------------------------------------------%

\section{Type System}

\begin{example}
Explain what is meant by a \textit{sound} type system.

A well-typed expression always coverges and returns either a value of the 
same type as the expression, or nullPntrExc. However, it nevers get stuck.
After execution, the types of objects on heap must be consistent 
with program and the environment.
\end{example}

We \textit{cannot} have $r = \text{stuckErr}$. 

%--------------------------------------------------------------------------------------------%
%--------------------------------------------------------------------------------------------%
%--------------------------------------------------------------------------------------------%