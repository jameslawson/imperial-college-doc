
\chapter{$\mathcal{L}_2$ Language I}


\section{Introducing $\mathcal{L}_2$}

Here extend the language $\mathcal{L}_1$ and create a new language 
$\mathcal{L}_2$. $\mathcal{L}_2$ has  inheritence.

The work in this chapter will be on giving a formal description of $\mathcal{L}_1$ 
consists of
\begin{itemize}   
\renewcommand{\labelitemi}{$\Box$}
\item \textbf{Syntax} 
\item \textbf{Operational Semantics} 
\item \textbf{Type System} We will see how to prove \textit{type soundness}, 
demonstrated through an important theorem called the \textit{subject reduction theorem}.
\end{itemize} 

The \textit{structure} of an $\mathcal{L}_1$ program is defined by the function
\highlightdef{ $Prog = ClassId \rightarrow  (ClassId \times 
(FieldId \rightarrow type) 
\times (MethodId \rightarrow meth))$ }

where $meth$ and $type$ have a syntax defined by:
\begin{flalign*}
meth &::= type\, m (type\, x) \{ x \} &\\
type &::= \text{bool} | m \\
e    &::= \text{if}\, e \text{then}\, e\, \text{else}\, e | e.f | e.f := e 
               | e.m(e) | \text{new}\, c | x | \text{this} | \text{true} 
               | \text{false} | \text{null}
\end{flalign*}
with the identifier conventions: $c \in ClassId$  for class identifiers,
$f \in FieldId$ for field identifiers and $m \in MethId$  for class 
identifiers.

\frmrule
