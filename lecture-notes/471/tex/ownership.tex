
\chapter{Ownership Types}


\section{Owners as Dominators}



Taking all paths from the root, $a_r$ to $a'$, 
$a$ \textit{dominates} $a'$ iff $a$ is always one of the objects 
on the path. In other words, we always have to go through 
$a$ to reach $a'$ from the root. 

Fix objects from $\chi$. For all paths from $a_r$ to $a'$, $a_r \ll a_1, \dotsm, a_k \gg a'$, we have:

\begin{prooftree}
\def\defaultHypSeparation{\hskip .01in}
\AxiomC{$\chi \vdash a_r \ll a_1, \dotsm, a_k \gg a'$ implies $a = a_i$ for some $i \in \{ 1 \dotsm k \}$}
\LeftLabel{\textsc{dom}}
\UnaryInfC{$\chi \vdash a \text{ dom } a'$}
\end{prooftree}

We have that $a$ is always on the path, on of the 

\frmrule


\begin{example}
Does $\chi \vdash a \text{ dom } a$ hold for all objects $a$? 
That is, is it possible for an object to dominate itself. Explain why/why not.
\end{example}

\frmrule

\begin{example}
Does $\chi \vdash a \text{ dom } a'$ imply that $\chi \vdash a \ll \dotsm \gg a'$?
That is, does a node dominating another node imply that the nominated 
node is reachable by the nominating node? 
\end{example}

\frmrule

\begin{example}
Does $\chi \vdash a \text{ dom } a'$ and $\chi \vdash a \text{ dom } a''$ imply that $a' = a''$?
That is, is it impossible for an object to dominate two distinct objects at the same time. 
\end{example}

\frmrule

\begin{example}
Does $\chi \vdash a' \text{ dom } a$ and $\chi \vdash a'' \text{ dom } a$ imply that $a' = a''$?
That is, is it impossible for an object to be dominated by two distinct objects. 
\end{example}

\frmrule

\begin{example}
Does $\chi \vdash a \text{ dom } a'$ and $\chi \vdash a' \text{ dom } a''$ imply  
$\chi \vdash a \text{ dom } a''$. That is, is the dom relation transitive?
\end{example}

\frameans{Find a counterexample or prove it to be true.}{No.}

Below shows a counterexample. 



\frmrule

\begin{example}
For the following heap, find the dom relation.
\end{example}


\frmrule

Recall the ownership mapping, $\text{own}: Addr \rightarrow Addr$. 
This mapping is defined by the type system and code written
in the programming language. 

Now that we have dominations defined, 
we can define the notion of object owners dominating the objects 
that they own. 

This is the notion of \textit{owners as dominators}. 
We say that a given ownership mapping \textit{respects} the 
dom relationship iff for all objects, their owners are their 
dominators. 

For all objects $a$, we have: 

\begin{prooftree}
\def\defaultHypSeparation{\hskip .01in}
\AxiomC{$\chi \vdash a' \text{ dom } a$}
\LeftLabel{\textsc{own-dom}}
\RightLabel{$\text{own}(a) = a'$}
\UnaryInfC{$\text{own} \vdash \chi \lozenge$}
\end{prooftree}


\highlightdef{\textbf{Ownership Respecting Domination}: $\text{own} \vdash \chi \lozenge$}


\frmrule 




\highlightdef{\textbf{Direct Owner Access}: $\chi \vdash a \ll a, a' \gg \text{ implies } \exists \text{own}^k{a} = \text{own}(a')$}

That is, a reference from $a$ to $a’$ is only legal if $a$ owns $a'$, 
or the owner of $a$ owns $a'$ or, the owner of the owner of $a$ owns $a'$, ...

\begin{example}
Which arrows are legal in the following diagram under \textit{direct owner access}
\end{example}

\frmrule 


\begin{example}
Given a heap with ownership types $\chi$, if 
$\chi$ satisfies \textit{direct owner access}, is it true that

$$\chi \vdash a \ll \dotsm \gg a'  \text{ imply } \exists \text{own}^k{a} = \text{own}(a')$$

for all paths and all objects $a, a'$? 
That is, are indirect references valid if the owner of $a'$ is an ancestral owner of $a$?
\end{example}



\frmrule 

If heap has direct owner access has then we have:
\highlightdef{\textbf{Access Control}: $a$ controls $a'$ iff $a$ owns $a'$ and there is a path from $a$ to $a'$}


The ownership boxes and the dom relationship are not part of 
definition of the heap $\chi$. Ownership types are predetermined 
by the code. The ownership types are not changed when the 
state $\chi$ is updated. 
The dom relation can easily change when the state $\chi$ is updated.

