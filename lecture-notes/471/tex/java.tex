
\chapter{Java Implementation}



\section{Introducing Java Bytecode}




\section{$\mathcal{JVM}_{00}$ Language I}

formalizes a subset of the possible instructions.
For simplicity, we fix a method and fix the class. 
We only need to know the argument types 
$t_1$, $t_2$, ... $t_n$, the result type $t$ and 
$\text{Stack} = I_1$, $\text{Locals} = I_2$, $\text{ArgsSize} = I_3$.

The syntax is defined below. Notice that there is no recursion in 
this definition. 

$$instruction ::= \text{inc } | \text{ pop } | \text{ store } x\; | \text{ load } x\; | \text{ if } L\; | \text{ halt}$$

where $x$ is the number (we define semantics, this will be for the local variable $x \leqslant \text{Locals}$)  
and $L$ is a label of some instruction.

\frmrule

\begin{itemize}   
\renewcommand{\labelitemi}{$\Box$}
\item \textbf{Program} A function $P: 0..k \rightarrow instruction$ stands for the program. 
The integer $k$, when adding one, tells us how many the bytecode instructiona there are. 
\item \textbf{Program Counter} $pc$ is the program counter. $P[pc]$, gives us the 
the current instruction. 
\item \textbf{Local Variables} A function $f: 0..j \rightarrow integer$ 
describes the contents of local variables. Here $integer$
represents both integers and addresses. The value of $j$, after adding one,
tells us the number of local variables we have ($= \text{Locals}$).
\item \textbf{Stack} The operand stack $s$ is a list of values. 
The maximal length of the list is $\text{Stack}$.
\end{itemize} 

Execution has the format $P \vdash (pc, f, s) \leadsto (pc', f', s')$. 
After execution, the program counter may have changed, the local variables may have changed, 
and the stack contents may have changed.

\begin{prooftree}
\def\defaultHypSeparation{\hskip .01in}
\AxiomC{$P[pc] = \text{inc}$}
\LeftLabel{\textsc{inc}}
\UnaryInfC{$P \vdash (pc, f, n \cdot s) \leadsto (pc+1, f, (n+1) \cdot s)$}
\end{prooftree}

\begin{prooftree}
\def\defaultHypSeparation{\hskip .01in}
\AxiomC{$P[pc] = \text{pop}$}
\LeftLabel{\textsc{pop}}
\UnaryInfC{$P \vdash (pc, f, v \cdot s) \leadsto (pc+1, f, s)$}
\end{prooftree}

\begin{prooftree}
\def\defaultHypSeparation{\hskip .01in}
\AxiomC{$P[pc] = \text{load}$}
\LeftLabel{\textsc{load}}
\UnaryInfC{$P \vdash (pc, f, s) \leadsto (pc+1, f, f(x) \cdot s)$}
\end{prooftree}

\begin{prooftree}
\def\defaultHypSeparation{\hskip .01in}
\AxiomC{$P[pc] = \text{store}$}
\LeftLabel{\textsc{store}}
\UnaryInfC{$P \vdash (pc, f, v \cdot s) \leadsto (pc+1, f[x \mapsto v], s)$}
\end{prooftree}

\begin{prooftree}
\def\defaultHypSeparation{\hskip .01in}
\AxiomC{$P[pc] = \text{if } L$}
\AxiomC{$n = 0$}
\LeftLabel{\textsc{if-false}}
\BinaryInfC{$P \vdash (pc, f, n \cdot s) \leadsto (pc+1, f, s)$}
\end{prooftree}

\begin{prooftree}
\def\defaultHypSeparation{\hskip .01in}
\AxiomC{$P[pc] = \text{if } L$}
\AxiomC{$n \neq 0$}
\LeftLabel{\textsc{if-true}}
\BinaryInfC{$P \vdash (pc, f, n \cdot s) \leadsto (L, f, s)$}
\end{prooftree}

\frmrule

\begin{example}
Which instructions increment the program counter?
\end{example}

\begin{example}
Why is there no operational semantics rule for halt?
\end{example}

\frmrule

\begin{example}
Write out the derivation tree for the execution of the following.
\end{example}



\frmrule

We now look at the type system for $\mathcal{JVM}_{00}$. 
The only types we consider for $\mathcal{JVM}_{00}$ are integers and 
classes. Hence the set of types is:

$$Type = \{ \text{ int } \} \cup \{ \text{ A } | \text{ A is the name of a class } \} $$

The type of the stack can be seen as a \textit{vector of types}. Similarly, 
for convenience, we can think of the types for local variables can 
being a \textit{vector of types}. 
For example, $\text{int} \cdot \text{A} \cdot \text{int}$ is an example 
of such a vector, where A is the name of some class.


\begin{itemize}   
\renewcommand{\labelitemi}{$\Box$}
\item \textbf{Local Variable Type Vector} We have $F_i$ which 
is a vector of types for local variables immediately before executing the $i$th instruction
For example, $F_3[2]$ is the type of the 2nd local variable before executing instruction 3.
\item \textbf{Stack Type Vector} We have $S_i$ which 
is a vector of types for local variables as indexed  immediately before executing the $i$th instruction
For example, $S_3[2]$ is the type of the 2nd operand immediately before executing instruction 3.
\end{itemize} 

\frmrule

The judgement $F, S, i \vdash P$ asserts that the i-th instruction of 
P is \textit{well-formed with respect to F and S}. 
We will now define this judgement recursively over the structure of $instruction$. 

\begin{prooftree}
\def\defaultHypSeparation{\hskip .01in}
\AxiomC{
$
\begin{matrix}
P[i] = \text{inc} \\
F_{i+1} = F_{i}\\
S_{i+1} = S_{i} = \text{int} \cdot \alpha  \\
i+1 \in dom(P)
\end{matrix}
$
}
\LeftLabel{\textsc{inc.wftype}}
\UnaryInfC{$F, S, i \vdash P$}
\end{prooftree}

Firstly, we pattern match on the $i$th instruction being the inc instruction. Recall 
that the inc instruction increments the value on top of the stack. 
We assert that the type vectors for the stack and local variables \textit{must not change}. 
So we have $F_{i+1} = F_{i}$ and $S_{i+1} = S_{i}$. Furthermore, we require the top of the stack 
to be of type $\text{int}$. We can only increment integers. Increment is undefined for other types.
Finally, after having incremented the program counter, we 
expect program $P$ to be have an instruction for the new program counter. In other words, 
$i+1$ must be in the domain of function $P$. 


\begin{prooftree}
\def\defaultHypSeparation{\hskip .01in}
\AxiomC{
$
\begin{matrix}
P[i] = \text{pop} \\
F_{i+1} = F_{i}\\
\text{t} \cdot S_{i+1} = S_{i}  \\
i+1 \in dom(P)
\end{matrix}
$
}
\LeftLabel{\textsc{pop.wftype}}
\UnaryInfC{$F, S, i \vdash P$}
\end{prooftree}

We pattern match on the $i$th instruction being the pop instruction. 
Next, we assert that local variable type vector does change, $F_{i+1} = F_{i}$. 
However the stack changes in a pop instruction, so we expect the stack type vector 
to change. In particular, the old stack type vector is the same as the old one, but with some 
type on the front. $\text{t} \cdot S_{i+1} = S_{i}$. Finally, after having incremented $pc$, we 
expect $P$ to be have an instruction for this new program counter. 

\begin{prooftree}
\def\defaultHypSeparation{\hskip .01in}
\AxiomC{
$
\begin{matrix}
P[i] = \text{store } x \\
x \in dom(F_i) \\
F_{i+1} = F_{i}[x \mapsto t]\\
\text{t} \cdot S_{i+1} = S_{i}  \\
i+1 \in dom(P)
\end{matrix}
$
}
\LeftLabel{\textsc{store.wftype}}
\UnaryInfC{$F, S, i \vdash P$}
\end{prooftree}

We pattern match on the $i$th instruction being the store instruction. 
Recall that \textit{store} pops the value from top of the stack 
and stores it onto the  $x$th local variable. 
Because we are storing into the  $x$th local variable, the type vector
must have an mapping. In other words, $x \in dom(F_i)$, where $dom(F_i)$ 
represents all the local vars (roughly thinking of $F_i$ as a function from local vars 
to types). The new local var type vector $F_{i+1}$ is the same as before but
it has the type of $x$ updated to $t$ for $x$ where 
type $t$ is the type of the value we popped top of the stack, $S_i = t \cdot S_{i+1}$. 
Finally, we increment $pc$, and assert that $P$ must have the new instruction. 

\begin{prooftree}
\def\defaultHypSeparation{\hskip .01in}
\AxiomC{
$
\begin{matrix}
P[i] = \text{load } x \\
x \in dom(F_i) \\
F_{i+1} = F_{i}\\
S_{i+1} = F_{i}[x] \cdot S_{i}  \\
i+1 \in dom(P)
\end{matrix}
$
}
\LeftLabel{\textsc{load.wftype}}
\UnaryInfC{$F, S, i \vdash P$}
\end{prooftree}

We pattern match on the $i$th instruction being the load instruction. 
Recall that \textit{load} pushes a local variable onto the stack. 
We are pushing the  $x$th local variable, so the type vector
must have an mapping for $x$. In other words, $x \in dom(F_i)$, again where $dom(F_i)$ 
represents all the local vars that correspond to the types in the type vector. 
We are only reading local variables. There are no writes so the type vector for local 
variables remains unchanged, $F_{i+1} = F_{i}$. However the stack is updated.
The new stack is the same as the old except the top has the local variable $x$. 
So the stack \textit{type vector} is the same as the old, except that 
top should have the \textit{type} of the local variable $x$, which is $F_{i}[x]$. 
Hence $S_{i+1} = F_{i}[x] \cdot S_{i}$. 
Finally, we increment $pc$, and assert that $P$ must have the new instruction. 


\begin{prooftree}
\def\defaultHypSeparation{\hskip .01in}
\AxiomC{
$
\begin{matrix}
P[i] = \text{if } x \\
F_{i+1} = F_L = F_{i}\\
t \cdot S_{i+1} = t \cdot S_L \cdot S_{i}  \\
i+1 \in dom(P), L \in dom(P)
\end{matrix}
$
}
\LeftLabel{\textsc{if.wftype}}
\UnaryInfC{$F, S, i \vdash P$}
\end{prooftree}

We pattern match on the $i$th instruction being the if instruction. 
Recall that \textit{if} pops the stack, checks it it is zero and branches 
accordingly. This is a rule for types and not for operational semantics. 
So we are unsure which branch is taken. So we will assert that \textit{both} 
paths are well formed with respect to $F$ and $S$, regardless of the 
fact that only one path is actually taken during exection.

The equation $t \cdot S_{i+1} = t \cdot S_L \cdot S_{i}$ 
tells us that $S_L$ and $S_{i+1}$ have had a value of type $t$ popped off. 
The stack has been popped no matter which path has been taken. 
If we branch to instruction $L$, the stack, $S_L$ has been popped, if we 
don't jump to $L$ and continue to instruction $i+1$, the stack $S_{i+1}$ has still been 
popped. We are only reading local variables, there are no writes.
So the type vector for local variables remains unchanged, $F_{i+1} = F_{i}$.

Because we type check both branches, we type check that 
the next instruction exists for both branches. Hence we require  
$i+1 \in dom(P)$ (as usual), but \textit{also require}, $L \in dom(P)$ to hold.


\frmrule

\begin{example}
Look at the rule \textsc{if.wftype}. 
Notice that we assert that some value $t$ has been popped from the operand stack but,
there is no assertion that $t$ is of \textit{type int}. Look at the operational semantics 
and explain what happens if an object is popped off during an if instruction. 
It is necessary to assert that $t$ must be of \textit{type int}?

\frmrule

If the value isn't an int, then, looking at the operational semantic rules 
we can be sure that the if evaluates to true. And object popped off the stack 
causes an if statement to evaluate to true. In fact, the rules \textsc{if-true} 
and \textsc{if-false} together tell us that so long as the top of the stack is 
not the integer zero, an the if-statement evaluates to true. 

It is not necessary to assert that $t$ must be of type int. 
Our operational semenatics allow us to evaluate objects for if statements and 
so we have written our type system to reflect this. 

\end{example}

\frmrule

\begin{example}
Which instructions require the next instruction to be defined in $P$?
\end{example}

\begin{example}
Which instructions leave the local variable type vector unmodified?
\end{example}


\section{$\mathcal{JVM}_{00}$ Language II}



\section{Dynamic Linking and Loading}