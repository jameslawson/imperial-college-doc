
\chapter{Devices and Drivers}


\section{Device Drivers Revision}

We revise the device drivers from 113. 

What is perhaps misleading about the names \textit{top half} and \textit{bottom half} 
is that one might think that half the work is done by the \textit{top half} and the other half 
of processing is done by the \textit{bottom half}. This is wrong. In fact, 
the very purpose for having a top half/bottom half divide is to allow the halves to differ 
greatly in the amount of processing they do. 

A given driver can spend $L$ amount of time in a top half. If it spends any longer then 
it is putting a bottleneck in the system. Interrupts are turned off in the top 
half and the top half has to be \textit{quick}. If not everything can be done in the top 
half, we resort to having to schedule that work to be done later in a thread called 
the \textit{bottom half}.
We require that the time for the top half work to be completed does not take too long 
$0 \leq \text{time}(w_T) < L$.
The bottom half has no bound on how long it takes: $0 \leq \text{time}(w_B)$. 


\begin{figure}[h]
\begin{tikzpicture}[
  n/.style={draw=none, fill=none, text width=1.6cm, align=center,inner sep=6pt,font=\small},
  s/.style={draw=black, fill=white, drop shadow, inner sep=2pt},
]
  %   
  \node[n] (t1) {Top Half};
  \node[n, below=0cm of t1, minimum height=3cm](b1) {Bottom Half};
  \draw[dashed] (b1.north west) -- (b1.north east);
  %
  \node[n, right=1cm of t1] (t2) {Top Half};
  \node[n, below=0cm of t2, minimum height=3cm](b2) {Bottom Half};
  \draw[dashed] (b2.north west) -- (b2.north east);
  %
  \node[n, right=3cm of t2] (t3) {Top Half};
  \node[n, below=0cm of t3, minimum height=3cm](b3) {Bottom Half};
  \draw[dashed] (b3.north west) -- (b3.north east);

  \begin{pgfonlayer}{background}
      \node [s, fit={(t1) (b1)}] (s1) {};
      \node [s, fit={(t2) (b2)}] (s2) {};
      \node [s, fit={(t3) (b3)}] (s3) {};
      \fill[draw=none,color=black!15] (t1.north west) rectangle (b1.south east);
      \fill[draw=none,color=black!15] (b2.north west) rectangle (b2.south east);
      \draw[->] ($(s1.east)+(0.1cm,0.0cm)$) -- ($(s2.west)+(-0.1cm,0.0cm)$);
      \draw[->] ($(s2.east)+(0.1cm,0.0cm)$) -- ($(s3.west)+(-0.1cm,0.0cm)$);
      \draw[decorate,decoration={brace,mirror},thick]
      ($(s1.south)+(0.1cm,-0.3cm)$) to node[midway,below,yshift=-0.1cm] (bracket) {Interrupts Disabled}
      ($(s2.south)+(-0.1cm,-0.3cm)$);
      \draw[decorate,decoration={brace,mirror},thick]
      ($(s2.south)+(0.1cm,-0.3cm)$) to node[midway,below,yshift=-0.1cm] (bracket) {Interrupts Enabled}
      ($(s3.south)+(-0.1cm,-0.3cm)$);
  \end{pgfonlayer}
\end{tikzpicture}
\end{figure} 


\begin{itemize}   
\renewcommand{\labelitemi}{$\Box$}
\item \textbf{Immediate Handler} (Top Half)
Also called the \textit{interrupt handler}. 
This services the interrupt. It checks for errors and copies data to/from 
the memory area it shares with the bottom half.
The processing done here is for \textit{time sensitive data} but at the same time, 
the processing done must be \textit{minimal}. Any non time-critical processing related
to the event that caused the interrupt is \textit{deferred} for the bottom half.
\item \textbf{Kernel Thread} (Bottom Half)
The bottom half runs as a schedulable thread within the os and performs
the remainder of the work that was too heavy to do in the immediate handler. 
\end{itemize}

In most systems (including Linux), it is \textit{not mandatory} to perform this top-half/bottom-half 
split. However quite often processing the entirety of an interrupt is indeed 
too heavy to do while interrupts are disabled in the top half. 
It's up the device driver developer to find the right balance between 
top half and bottom half. It's a \textit{design issue}. Also the developer has to design the top half 
so that that time sensitive data is handled but without causing too much latency. 

\frmrule

\begin{example}
Why not put all the work in the top half?

Interrupts are disabled during the top half. Performing too much work in the top half 
will mean that the interrupts will remain disabled for longer. In other words, 
access is denied for other devices 
for a longer time and may make the system less responsive.
\end{example}

\frmrule

\begin{example}
Why not put all the work in the bottom half?

Interrupts often have to respond to an interrupt event in a timely fasion. 
Yet, work that is done in the bottom 
half is delayed by putting it in a schedulable kernel thread. 
Any time critical processing mistakenly put 
in the bottom half will be preempted by the processing of other threads. 
This will make the device appear less responsive to the user.
\end{example}

\frmrule

\begin{example}
Network interrupt causes packet to be copied from 
network card to kernel space (kernel interrupt handler)
Other processing on the packet should be deferred until 
its time comes (kernel deferrable function)
\end{example}


\begin{example}
How is data passed between the top and bottom halves of a device driver?

The top half copied data to a shared memory area.
\end{example}

\frmrule

The bottom half can’t block. 
Interrupts are enabled during bottom half, 
there are not disabled like they are during top half.

If an interrupt occurs during the bottom half, 
a new top half is introduced. 
The bottom half preemptable by top half of a different driver.

This introduces the idea of \textit{interrupt nesting}.






\section{Introducing Linux Drivers}





\highlightdef{\textbf{LKM}: Linux drivers are written 
inside a \textit{Loadable Kernel Module}. These are binary 
files can be use to extend kernel code on-demand  }



\textsc{linux} puts io devices, mostly, into two categories.


\begin{figure}[h]
\begin{tikzpicture} [
   every node/.style={node_sty, align=center},
   edge from parent/.style={edge_sty},
   edge from parent path={(\tikzparentnode.south) 
   |- ($(\tikzparentnode.south)!0.5!(\tikzchildnode.north)$) -| (\tikzchildnode.north)},
   level 1/.style={sibling distance=4cm, level distance = 0.9cm}
]
   \node 
   (root) {\textsc{linux} \textsc{io} Device}
      child {node (pio) {Character}}
      child {node(intio) {Block}};
   
\end{tikzpicture}
\end{figure}

We a familiar with \textsc{io} devices being hardware.
However \textsc{unix} allows an \textsc{io} device to be purely software. 


\begin{itemize}   
\renewcommand{\labelitemi}{$\Box$}
\item \textbf{Character} 
A \textit{character io device} is one that works with a sequential streams of bytes. 
The driver for character \textsc{io} devices are written to be able to work with streams.
They need to implement at least the \textit{open}, \textit{close}, \textit{read}, and 
\textit{write} unix system calls. 
\item \textbf{Block} 
A \textit{block io device} is the like a character io device. 
It works with streams of bytes. It also must implement at least 
\textit{open}, \textit{close}, \textit{read}, and \textit{write}. 
However, \textit{more operations} are provided to handle streams of bytes as 
a \textit{flow of blocks}. Block \textsc{io} devices are 
able to process blocks whose data arrives in a non-sequential order.
\end{itemize}

Note that these two categories are \textit{not absolute}. 
The distinction is often \textit{blurred}. For example, is a tape a 
block device or a character device? Others devices don't fit 
into these characters.
For example clocks and network devices.
One important distinction is that block device transfers 
are buffered through a \textit{block buffer cache}.


\highlightdef{Block devices have access to the \textit{block buffer cache}. 
Character devices however do not. }


\frmrule


Recall from the first chapter that \textsc{unix} makes a distinction between \textit{normal files}
and \textit{special files}. Each device has a \textsc{unix} special file.
We will call this the \textit{device special file}. 


\begin{figure}[h]
\begin{tikzpicture} [
   every node/.style={node_sty, align=center},
   edge from parent/.style={edge_sty},
   edge from parent path={(\tikzparentnode.south) 
   |- ($(\tikzparentnode.south)!0.5!(\tikzchildnode.north)$) -| (\tikzchildnode.north)},
   level 1/.style={sibling distance=6cm, level distance = 0.9cm},
   level 2/.style={sibling distance=1cm},
   ch/.style={text width=0.5cm}
]
   \node 
   (root) {\textsc{unix} Files}
      child {
        node (pio) {Normal Files}
        child {
            node(n)[ch] {\lstinline{-}}
        }
      }
      child {
        node(intio) {Special Files}
        child {node(d)[ch] {\lstinline{d}}}
        child {node(l)[ch] {\lstinline{l}}}
        child {node(p)[ch] {\lstinline{p}}}
        child {node(c)[ch, fill=black!15] {\lstinline{c}}}
        child {node(b)[ch, fill=black!15] {\lstinline{b}}}
        child {node(s)[ch] {\lstinline{s}}}
      };
   
\end{tikzpicture}
\end{figure}

Character and block devices have their special files located in the \lstinline{/dev} directory.


\frmrule

Every device is uniquely identified up to a pair of numbers $(j,n)$,
the first, $j$ is called the \textit{major identification number} 
and the second, $n$ is called the \textit{minor identification number}. 
Having major and minor identification numbers allows the operating system 
to organise devices and group devices that behave similarly. In particular, 
devices with \textit{same major number} controlled by \textit{same driver}.


\section{Writing a Unix Driver}

Suppose we create our own driver called \textsc{doc} Driver or \lstinline{docdrv}. 
This will be stored in \lstinline{/dev/docdrv}. 

These are event functions that give us a chance 
to write code that executes when the driver is loaded and also write code 
for when it is stopped. 

\begin{itemize}   
\renewcommand{\labelitemi}{$\Box$}
\item \textbf{Driver Open Function} When the user opens \lstinline{/dev/docdrv}, 
the kernel calls a function in \lstinline{docdrv} called the
\textit{driver open function}. This is used for initialisation code.
\item \textbf{Driver Close Function} Similarly, when the user closes \lstinline{/dev/docdrv}, 
the kernel calls a function in \lstinline{docdrv} called the 
\textit{driver close function}. This is used for deallocating memory and to ensure 
the driver closes safely.
\end{itemize}

\frmrule

Io modules can be loaded and removed using \textit{insmod} and \textit{rmmod}.

\begin{lstlisting}
doc > gcc -c hello.c
doc > insmod hello.o
Hello doc!
doc > rmmod hello.o
See you soon doc!
doc >
\end{lstlisting}


\section{Devices in Pintos}