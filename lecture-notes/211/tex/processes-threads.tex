
\chapter{Processes and Threads \textsc{i}}





\section{Introducing Processes}

%Q: \textit{What is a process exactly?} \\
%Q: \textit{Is it possible to have more than one process? How?}  \\
%Q: \textit{How does the operating system' code get executed?}

Right now, the operating system sees a program \textit{during execution} 
an ongoing sequence of machine code instructions. Before code is executed 
a program can be seen as high-level source code but during 
execution the operating system doesn't have anything to describes its execution. 
There is nothing that captures its execution state, memory usage or register contents. 

The problem is that we \textit{need} this higher level view of a program running on a processor.
The execution of a program machine code instructions will lead to it requiring io devices.
Recall that use of io devices is complex and expensive. 
The operating system as is resonsible for managing io devices as resources that 
are to be allocated to programs. It acts as a resource manager

So, as a resource manager, an operating system should know how programs in 
execution affect resources. This is done with the idea of a \textit{process}. 
A \textit{process} is an invention by an operating system. 
It is used as a placeholder for executing programs so that we can keep a precise 
account of how program execution is affecting the use of hardware and software resources.

\highlightdef{A \textbf{Process} is the os's view of a program in execution}

It's an abstraction of a running program that's useful for the operating system.
As a virtual machine, the os needs to provide the user with easy access to the 
resources. As a processor-helper it acts as software that controls the
medium between the processor and devices.

\highlightdef{A \textbf{Process Control Block} or \textsc{pcb} is a data structure 
that the \textsc{os} uses to store \textit{information about a process}. 
It allows the \textsc{os} to perform \textit{context switches}. }

We will see that we can have processes running in a somewhat simulatenous manner.
Here, the operating system will allocates 
a process control block for each process running. 

The \textsc{pcb} contains a \textit{huge amount of information} in
one \textit{large} structure. It is a heavyweight abstraction.


\begin{example}
In \textsc{unix}, a process is represented by a struct called \lstinline{task}. 
Below shows a definition of a typical Unix \lstinline{task} struct.  
\end{example}

Within the \textsc{unix} communicity this struct is known as a \textit{process descriptor}
or a \textit{task descriptor}.

\highlightdef{The operating system executes its own code using \textit{kernel processes}} 



\highlightdef{An \textbf{Execution Context Switch} is }



\section{UNIX Fork-Exec Model}


\highlightdef{The \lstinline{fork()} system call creates 
a \textit{branching copy} of the current process}

The contents of all the general purpose registers are copied across. 
In fact, the exact execution context is recreated for newly created branch process, 
but with \textit{one difference}. 
The difference between the branch copy and the original is
that the return result of \lstinline{fork()} is different.

\begin{figure}[h]
\begin{tikzpicture}
  % circles
  \node[circle,draw,fill=white] (c1) {};
  \node[circle,draw,fill=black, right=1.5cm of c1] (c2) {};
  \node[circle,draw,fill=black, below=1cm of c2] (c3) {};
  % endpoints
  \coordinate (x1) at ($(c1)+(-3.5cm, 0.0cm)$);
  \coordinate (x2) at ($(c2)+(2.5cm, 0.0cm)$);
  \coordinate (x3) at ($(c3)+(2.5cm, 0.0cm)$);
  % lines
  \draw[dashed,->] (c1) -- (c2);
  \draw[dashed,->] (c1) -- (c3);
  \draw[->] (x1) -- (c1);
  \draw[->] (c2) -- (x2);
  \draw[->] (c3) -- (x3);
  % labels
  \node[above=0.1cm of c1] {\lstinline{r = fork()}};
  \node[above=0.1cm of c2] {$\text{\lstinline{r} } \neq 0$};
  \node[above=0.1cm of c3] {$\text{\lstinline{r} } = 0$};
  \node[right=0.1cm of x2] {original process};
  \node[right=0.1cm of x3] {branch process};
	
\end{tikzpicture}
\end{figure}

The newly created process has a return value of \textit{zero} given back 
from calling fork(). On the other hand, the original process has a \textit{non-zero} 
value given back. When the \textsc{os} is recreating the execution context
for the new process, it ensures that this difference is put in.

We can use this difference to write conditional statements 
that will only execute code for a single branch (i.e. for a single process). 
Below shows an example.

\frmrule





\begin{example}
\begin{lstlisting}
void example() {
    pid_t pid = fork();
    if (pid != 0) printf("P");
    else          printf("C");
}
\end{lstlisting}
\end{example}

\frmrule

We don't have to use the return value of \lstinline{fork()}. 
Below shows an example where we don't do any 
conditional logic using the result from \lstinline{fork()}.

\begin{example}
\begin{lstlisting}
void example() {
    printf("E");
    fork();
    printf("F");
}
\end{lstlisting}
\end{example}


\frmrule


\begin{example}
Explain what the following 
means: \textit{A single call to function \lstinline{fork()} returns twice}.
In particular explain why we can think of this function as returning 
twice. 
\end{example}

\frmrule

\begin{example}
What process-related data structure will the \textsc{os} need to read from 
to recreate the execution context for the new branch process?
\end{example}

\frmrule


\begin{example}
For the following code, state what is printed, and precisely in what order.
Explain your answer. State how many processes are running 
after \lstinline{example()} returns.
\begin{lstlisting}
void example() {
    pid_t pid = fork();
    if (pid != 0) printf("P");
    else          printf("C");
    printf("X");
}
\end{lstlisting}
\end{example}

\frmrule


\begin{example}
For the following code, state what is printed, and precisely in what order.
Explain your answer. State how many processes are running 
after \lstinline{example()} returns.
\begin{lstlisting}
void example() {
    pid_t pid1 = fork();
    if (pid1 != 0) {
        printf("A")
    } else {
        pid_t pid2 = fork();
        if (pid2 != 0) {
            printf("B");
        } else {
            printf("C");
        }
    } 
}
\end{lstlisting}
\end{example}


\frmrule

\begin{example}
For the following code, state what is printed, and precisely in what order.
Explain your answer. State how many processes are running 
after \lstinline{example()} returns.
\begin{lstlisting}
void example() {
    pid_t pid1 = fork();
    if (pid1 != 0) {
        printf("A")
    } else {
        pid_t pid2 = fork();
        if (pid2 != 0) {
            printf("B");
        } else {
            printf("C");
        }
        printf("Y");
    } 
    printf("X");
}
\end{lstlisting}
\end{example}

\frmrule

\begin{example}
For the following code, state what is printed, and precisely in what order.
Explain your answer. State how many processes are running 
after \lstinline{example()} returns.
\begin{lstlisting}
void example() {
    pid_t pid1 = fork();
    if (pid1 != 0) {
        printf("X")
    }
    pid_t pid2 = fork();
    if (pid2 != 0) {
        printf("Y")
    }
}
\end{lstlisting}
\end{example}

\frmrule

\begin{example}
For the following code, state what is printed, and precisely in what order.
Explain your answer. State how many processes are running 
after \lstinline{example()} returns.
\begin{lstlisting}
void example() {
    pid_t pid1 = fork();
    if (pid1 != 0) {
        printf("X");
    }
    printf("S");
    pid_t pid2 = fork();
    if (pid2 != 0) {
        printf("Y");
    }
    printf("T");
}
\end{lstlisting}
\end{example}

\frmrule

\begin{example}
Suppose the first process has a pid of 0. What problem arrises 
when conditional logic with \lstinline{fork()} is used?
\end{example}

The initial process is therefore given an id of 1.
This initial process is handcrafted by the \textsc{os}. 
There is no \lstinline{fork()} call. 


\frmrule

Quite often, the child process we desire will have \textit{little resemblence} 
to the parent process. We need a mechanism to making it easier to change our child 
process's execution context. This can be achieved with the \lstinline{exec} 
system call.

\highlightdef{The \lstinline{exec} system calls
 replaces the current process with a new process}

By this, we mean that the memory allocated for a process is replaced with new 
instructions and new data. For a child process, this effectively 
replaces the recreated execution context - so the child has no similarity 
to the parent at all. 

There is a large family of \lstinline{exec} calls. These are defined in  
\lstinline{<unistd.h>} \textsc{posix} header.


First we will look at \lstinline{execv}.

\frmrule

\begin{example}
For the following code, state what is printed, and precisely in what order.
Explain your answer. State how many processes are running 
after \lstinline{example()} returns.

\begin{lstlisting}
void example() {
    pid_t pid = fork();
    if (pid != 0) {
        printf("P");
    } else {
        printf("C");
        execv('./prog', 5, 'hello');
    }       
    printf("X");
}
\end{lstlisting}
\end{example}

\frmrule

Another \lstinline{exec} system call is \lstinline{execl}.

\begin{example}
For the following code, state what is printed, in all possible orders.
Explain your answer. State how many processes are running 
after \lstinline{example()} returns.

\begin{lstlisting}
void example() {
    pid_t pid = fork();
    if (pid == 0) {
        execl("/bin/ps", "ps", "-l", NULL);
        printf("A");
    } else {
        printf("B");
    }
    printf("C");
}
\end{lstlisting}
\end{example}

\lstinline{execl} first takes in the \textit{path of a program} to run,
then we assign a name \textit{name of the command} (usually a shortened form), 
and then a  \textit{list of parameters} (just like on 
the command line), and then finishes with \textsc{null}.




% execl(“/bin/cat”, “cat”, doc.txt”, NULL);



\section{Execution-Schedule Management}

The job of a \textit{scheduler} is to arbitrate access to the 
current (\textsc{cpu} between multiple processes.
The header file include/linux/sched.h is included (either explicitly or indirectly) by
virtually every kernel source file.

The Linux kernel maintains a global variable called \lstinline{jiffies}. 
This represents the number of timer ticks since the machine started. 
This variable is initialized to zero and increments each timer interrupt.

Below shows an example that uses \lstinline{get_jiffies_64}. This reads the jiffies 
variable. This value can be converted to a millisecond value using 
\lstinline{jiffies_to_msecs}.


\highlightdef{The execution-schedule is \textbf{non-deterministic} 
because of the use of a \textit{timer interrupt}}

\frmrule

If we have five processes that use the processor twenty percent of the time (spending eighty percent 
doing \textsc{io}) then we should be able to achieve one hundred percent CPU utilisation. Of course, in 
reality, this will not happen as there may be times when all five processes are waiting for \textsc{io}. 
However, it seems reasonable that we will achieve better than twenty percent utilisation than we 
would achieve with monoprogramming. 
 
We can build a model from a probabilistic viewpoint. Assume that that a process spends 
$p$ percent of its time waiting for \textsc{io}. With $n$ processes in memory 
the probability that all n processes are waiting for \textsc{io} (meaning the CPU is idle) is $p^n$


\highlightdef{$P[\text{cpu free}] = p^n$} 


\highlightdef{As we introduce more processes, the \textsc{cpu} utilisation increases} 

The model is a little contrived as it assumes that all the processes are independent in that processes 
could be running at the same time. This (on a single processor machine) is obviously not possible. 
More complex models could be built using queuing theory but we can still use this simplistic model 
to make approximate predictions. 

\frmrule

\begin{example}

\end{example}

\frmrule


\section{Interprocess Communication I}


\highlightdef{\textbf{Interprocess Communication} (\textsc{ipc})
concerns \textit{data communication} and \textit{synchronisation} between processes} 



In \textsc{unix} most \textsc{ipc} obhects are \textsc{unix} \textit{special files} by 
are identified by \textit{file descriptors}. 
File descriptors 0, 1, and 2 are reserved by convention 
for standard input, standard output, and standard error.

Recall that \textsc{unix} is written in \textsc{c}. From a source-code perspective, 
a program has a standard output stream and a standard input stream. 

\highlightdef{A \textbf{UNIX Pipe} connects a 
\textit{write file descriptor} of $A$ to a \textit{read file descriptor} of $B$}

\textsc{unix} pipes therefore provide a mechanism
for \textit{one-way communication} betweeen $A$ and $B$ where 
$A$ is sending data to $B$. A common example would be 
connecting the \textit{standard output} of $A$ 
to the \textit{standard input} of $B$.

We can create a pipe using the \lstinline{pipe()} system call.
The \lstinline{pipe()} has just one parameter, \lstinline{pipefds} 
which is an out array for two file descriptors. 
The system call creates the pipe and two descriptors, 
one for writing to the pipe (\lstinline{pipefds[1]}) 
and one for reading from the pipe (\lstinline{pipefds[0]}). 

\frmrule

\begin{example}
Below shows an example of how to create a pipe 
using the \lstinline{pipe()} system call.

\begin{lstlisting}
int pipefds[2] = {0,0};
pipe(pipefds);
printf("pipe read fds: %d\n", pipefds[0]);
printf("pipe write fds: %d\n", pipefds[1]);
\end{lstlisting}
\end{example}

\frmrule

In the previous example, after the call to \lstinline{pipe()} 
we file descriptor table for the process looks like the following.


\frmrule


\begin{example}
Below shows an example of parent child communication 
using \lstinline{fork()} and \lstinline{pipe()}.


\begin{lstlisting}
int main(void) {
    int pipefds[2] = {0,0};
    pipe(pipefds);
    pid_t pid = fork();
    if (pid != 0) { 
        write(pipefds[1], "test", 5);
        printf("send: %s\n", buf);
        exit(0);
    } else { 
        char message[30];
        read(pipefds[0], buf, 5);
        printf("received: %s\n", buf);
        wait(NULL);
    }
    return 0;
}
\end{lstlisting}
\end{example}






\frmrule

\begin{example}
Below shows an example of a \textsc{unix} pipe is used to 
implement a \textsc{unix} \textit{command line pipe}  is used 
in shell command \lstinline{ls | wc -l}.

\begin{figure}[h]
\begin{tikzpicture}[
  process/.style={rectangle, draw, minimum height=2em},
  pipe/.style={cylinder, draw, minimum height=3cm, minimum width=0.5cm}
]
  \node [process] (ls) {\lstinline{ls}};
  \node [pipe, right=0.2cm of ls] (pipe1) {};
  \node [process, right=0.2cm of pipe1] (wc) {\lstinline{wc}};
  \draw [draw, dashed, ->] (ls) -- (wc);
\end{tikzpicture}
\end{figure} 
\end{example}



\frmrule

\begin{example}
Below shows an example of a \textsc{unix} \textit{command line pipe} 
used in shell command \lstinline{cat hello.txt | wc -w}.

\begin{lstlisting}
int example() {
    int pipefds[2] = {0,0};
    pipe(pipefds);
    pid_t pid = fork();
    if (pid != 0) { 
        close(pipefds[1]);
        dup2(pipefds[0], STDIN_FILENO);
        close(pipefds[0]);
        execl("/usr/bin/wc", "wc", "-w", NULL);
    } else { 
        close(pipefds[0]);
        dup2(pipefds[1], STDOUT_FILENO);
        close(pipefds[1]);
        execl("/bin/cat", "cat", "hello.txt", NULL);
    }
}
\end{lstlisting}
\end{example}



\frmrule

\begin{example}
The program below is run with the input 10 20 30 40 50 and piped to a file, as follows: 
\lstinline{a.out >output 10 20 30 40 50}. \\
\textbf{(a)} What is the sum of all the numbers written to the file \lstinline{output}? \\
\textbf{(b)} What numbers are printed on standard error?

\begin{lstlisting}
int main(int argc, char **argv) {
    int i, a=0, n=1;
    int fd[2];
    pipe(fd);
    dup2(fd[0],0);
    dup2(fd[1],1);
    for (i=0;i<5;i++) {
        printf("%d\n",n);
        fflush(stdout);
        n = n+a;
        fprintf(stderr, "%d\n",n);
        scanf("%d", &a);
    }
}
\end{lstlisting}
\end{example}

\frameans{}{
\textbf{(a)} 0.  \\
\textbf{(b)} 1 2 3 5 8
}

The program redirects stdout to a pipe. So no data is written to file output.

\frmrule

So far we have been using \textit{anonymous pipes}. This is fine if both 
processes are living in the same parent process space, 
started by the same user. 
There are however circumstances where the communicating processes
must use \textit{named pipes}. One such circumstance is that the processes have to be executed under different user names and permissions.


(Anonymous) pipes can only be used by related 
processes
FIFOs == pipe with name in file system
Any process can open and use FIFO
I/O is same as for pipes

A named pipe works much like a regular pipe, but does have
some noticeable differences.
* Named pipes exist as a device special file in the file
system.
* Processes of different ancestry can share data through a
named pipe.
* When all I/O is done by sharing processes, the named 
pipe remains in the file system for later use.

• Persistent pipes can outlive the process which 
created them
• Stored on the file system
• Any process can open it like a regular file
 Why ever use named pipes instead of files?


\begin{example}
Here is another shell command \lstinline{ls -l | tee file.txt | less}.
Technically tee is a process but semantically it is a nice to
visualise this as a T. 
\end{example}



\section{Interprocess Communication II}


\highlightdef{A \textbf{UNIX Signal} provide notifications for internal or external events} 

These can \textit{roughly} be compared to exceptions as seen in \textsc{c}++, Java.
The biggest difference is that signals are implemented at the \textit{kernel level}. 


Unix domain sockets (IPC socket) \\
Solaris doors \\
Windows COM \\


\section{Interprocess Communication III}

Mach Ports and Messages 



\section{Introducing Threads}





\highlightdef{\textbf{User-Level Threads}: } 


\highlightdef{\textbf{Kernel-Level Threads}: } 

How do user and kernel threads map into each other?

Many-to-One
One-to-One
Many-to-Many



\highlightdef{\textbf{POSIX} was developed to 
\textit{standardise} user-level threads and thread management 
}
as well as to standardise \textsc{ipc} mechanisms and commands 
in \textsc{unix} operating systems.

Before \textsc{posix}, the responsibility of writing the
implementation of threads and their interaction was placed 
solely on the developers of the given \textsc{os}. 
As a result, there were many varying implementations floating around, 
and writing portable multithreaded applications was difficult.

\textsc{ieee} came up with the \textsc{posix} standard in 1995. 
\textsc{posix} is an abbreviation for 
\textit{portable \textsc{os} interface standards}.
Threads that follow the standard are called 
\textit{\textsc{posix} threads} or \textit{pthreads}.

\begin{example}
What does the \textsc{p} in \textsc{posix} stand for? \\
For what reason was it seen neccessary to standardise threads.
\end{example}


User-level threads: 
POSIX \textit{P-threads}, Mach \textit{C-threads}, Solaris \textit{threads}



