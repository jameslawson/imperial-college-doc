
\chapter{Heaps}


\section{Introducing Binary Heaps}


\begin{example}

\textbf{Pairsumonious Numbers} \\
Any set of n integers form n(n−1)/2 sums by adding every possible pair. Your task is to find the
n integers given the set of sums. Define a function that, given n(n−1)/2 integers, 
returns n integers in non-descending order such that the input numbers are pairwise sums 
of the n numbers. If there is more than one solution, 
any one will do. If there is no solution, return the empty list.


\textbf{Two Sum Problem} 


\begin{example}
Google Jam, Africa 2010, Qualification Round - \textit{Store Credit}
\end{example}


\end{example}
