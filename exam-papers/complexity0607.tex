\documentclass[a4paper, draft, 12pt]{article}

% BEGIN PACKAGE IMPORT
% =============================================================================
\usepackage{examsty}
\usepackage{complexitysty}
% =============================================================================

% BEGIN SOLUTIONS
% =============================================================================



\begin{document}
\textbf{Complexity, 2006-2007}\\
\textit{James Lawson}

\begin{enumerate} 
\item % Question 1
\begin{enumerate}
\item % (a)
  \begin{enumerate}
  \item % (i)
  $L$ is $\textsf{NP}$ is there is a nondeterminstic turing machine that decides $L$ in p-time.
  \item % (ii)
   $L$ is $\textsf{NP}$-complete if it is in $\textsf{NP}$, and is $\textsf{NP}$-hard: if $L' \in \textsf{NP}$ then $L' \leqslant L$ .
  \item % (iii)
  Assume $L$ is $\textsf{NP}$-complete. 
  Assume for contradiction that $L$ is in \textsf{P}.
  Then take any langauge $L'$ in \textsf{NP}.
  So $L' \leqslant L$. 
  But given that $L$ is in \textsf{P}, we deduce that $L$ is in \textsf{P} by the downward closure property of reduction. 
  In other words, we have shown that $\textsf{P} \subseteq \textsf{NP}$. We already know that $\textsf{NP} \subseteq \textsf{P}$, 
  so $\textsf{P} = \textsf{NP}$ giving a contradiction. 
  \end{enumerate}
\item % (b)
  \begin{enumerate}
  \item % (i)
  $\textsc{TRI}$ is given sets $B$, $G$, $H$, each with $n$ elements
  and triples $T \subseteq B \times G \times H$, is there a subset $T' \subseteq T$ 
  such that $T'$ has $n$ elements and no two triples in $T'$ have an element in common.  

  \item % (ii)
  Consider a reduction $\textsc{tri} \leqslant \textsc{SubsetSum}$. \\
  Take $B, G, H $ of size $n$ and $T$ of size $m$.\\
  For each $t_i \in T$, associate string, of $3n$ 0/1s for a $(m+1)$-ary numbers. \\
  $f(B,G,H,T) = (W,K)$ where $K = \sum^{3n-1}_{i=0}(1+m)^{i}$ and $W = \{w_{t_i} \;|\; t_i \in T\}$. 
  \begin{itemize}
  \item $(\Rightarrow)$ Assume $(B,G, H, T)$ has a matching, then the sum will have $3n$ 1s because 
  there are no common elements so each $n$-bit grouping has unique operands in the summation. 
  \item $(\Leftarrow)$ Assume there is a subset sum in $f(B,G,H,T)$. \\
  There is some $w_{t_1} + w_{t_2} + ... + w_{t_k} = K$. $K$ is a series of 3$n$ 1s. 
  Each group of $n$ bits, is a series of 1s. Since each element has a unique $n$-string
  each bit is covered by exactly one element. So $k = n$ and each $w_{t_i}$ corresponds 
  to a different tuple, and the tuples sum to $K$.  
  \end{itemize}

  \end{enumerate}


\end{enumerate}

\end{enumerate}


\end{document}
