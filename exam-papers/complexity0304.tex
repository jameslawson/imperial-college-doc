\documentclass[a4paper, draft, 12pt]{article}

% BEGIN PACKAGE IMPORT
% =============================================================================
\usepackage{examsty}
\usepackage{complexitysty}
% =============================================================================

% BEGIN SOLUTIONS
% =============================================================================



\begin{document}
\textbf{Complexity, 2003-2004}\\
\textit{James Lawson}

\begin{enumerate} 
\item % Question 1
\begin{enumerate}
\item % (a)
  \begin{enumerate}
  \item % (i)
  There is a nondeterministic TM, $N$, that decides $L$ in polynomial time. 
  \item
  $L_1 \red L_2$ iff there is a map $f$ such that, (i) $x \in L_1$ iff $f(x) \in L_2$ 
  and (ii) $f$ is a p-time computable function.
  \item
  $L$ is \npc~iff (i) $L$ is \cl{NP}-hard: if $L'$ in \np~then $L' \red L$ (ii) $L$ in \np.
  \item
  To show \cl{HC}~is \npc~, consider a reduction $\hp \red \hc$.
  $$
  f((V,E)) = \left\{ 
  \begin{array}{ll}
    \onegraph    & \text{if } |V| = 1 \\
    (V \cup \{\textsf{p}\}, E \cup \{ (u, \textsf{p}) \;|\; u \in V \} )  & \text{otherwise} \\
    \quad \textsf{p} \text{ not in } V &\\
  \end{array} \right.
  $$
  To show: $G \in \hp$ iff $f(G) \in \hc$.
  \begin{itemize}
  \item ($\Rightarrow$) Assume $G$ has a HP. 
  If $|V| = 1$, then $f(G) = \onegraph$ which trivially has a HC.
  Otherwise, $G$ has some HP $x..y$. $f(G)$ has HC $\textsf{p}x..y\textsf{p}$. 
  \item ($\Leftarrow$) Assume $f(G) = (V',E')$ has a HC. 
  If $|V'| = 1$, only first case matches, $|V| = 1$, so $G$ trivially has a HP.
  Otherwise second case. $|V'| \geqslant 3$ and $f(G)$ has $\textsf{p}$. 
  So $f(G)$ has some HC of the form $\textsf{p}x..y\textsf{p}$, so $G$ must have HP $x..y$. 
  \end{itemize}
  We can compute $f$ in p-time (reading/writing to a adjacency matrix polynomial number of times) 
  so $f$ is a map that gives us $\hp \red \hc$. 
  So \hc~is NP-hard. But HC is in \np~so, we can guess string and check in polynomial time that 
  it is indeed a valid HC. Hence HC is \npc.
  \end{enumerate}
\item % (b)
  \begin{enumerate}
  \item % (i)
  $L$ is \textit{Cook Reducible} to $L'$ iff there is a p-time algorithm 
  which decides $L$ using an oracle for $L'$. 
  \item % (ii)
  Assume $\textsc{GIsom} \in \p$. \\
  Consider a Cook Reduction from \textsc{GAuto} to \textsc{GIsom}. 
  We show there is a p-time algorithm which decides \textsc{GAuto} using a \textsc{GIsom} oracle.
  \\\textit{Algorithm}:\\
  For each node in $G$, 
  delete node. call this graph $G'$. Ask the \textsc{GIsom} if there is an isomorphism from $G'$ to $G'$.
  \end{enumerate}
\end{enumerate}
\item % Question 2
\begin{enumerate}
\item % (a)
  \begin{enumerate}
  \item % (i)
  We can guess a path and check in logspace that is indeed a path from $x$ to $y$. 
  We use a counter to store the current node and another counter to count the 
  number of nodes we have checked so far. For the current node, check that the next node 
  in the path is adjacent to $x'$. Repeat up to $n$. If counter reaches $n$ and we haven't 
  reached $y$, halt and fail. If we reach $y$ within $n$ steps, succeed. 

  Each counter requires $\log(n)$ space and there are a fixed number of counters 
  (independent of input size). We also only \textit{read} from the input (read the path, 
  read adjacency matrix) which has no charge in terms of space. This algorithm shows \textsc{Rch} is in $\nl$. 
  \item % (ii)
  \textsc{Rch} is \nlc. If \textsc{Rch} were in \el, then this would imply $\nl = \el$ which is believed to be unlikely.
  \end{enumerate}
  \item % (b)
  A $k$-tape TM with input-output has at most $C = Q.n.(|\Sigma|^{f(n)}.f(n))^{k-2}.2$ configs. 
  We can apply the \textit{reachability method}. For any machine in $\textsf{NSPACE}$, 
  we can construct a machine $M'$ that on input $x$, builds config graph $G(M,x)$ then 
  runs \textsc{Rch} in $O(C^2)$ time, returning yes iff there is a path from init config to a yes state.
  This gives, $\textsf{NSPACE}(f(n)) \subseteq \textsf{TIME}(k^{\log(n)+f(n)})$ when $f$ is proper.

  Let $f(n) = \log(n)$. Since $\log(n)$ is a proper function, we can apply the previous result. 
  $k^{\log(n)+f(n)} = k^{\log(n)+\log(n)} = c^{\log(n)} = n^{c'}$. \\
  Hence $\textsf{NSPACE}(\log(n)) \subseteq \textsf{TIME}(n^{c'})$, i.e. $\nl \subseteq \p$.
  \item % (c)
  \textit{Algorithm}:
  Use one counter $s = 0$ (to store the input's size).
  Read input tape from left to right while incrementing $s$ until we reach the end. \\
  If $s$ is odd, then halt and fail. \\
  Otherwise. Loop using two counters $l = 0$ and $r = s/2$. \\
  Read the $l$th symbol and $r$th symbol of the input, if they are not equal, halt and fail.
  Otherwise increment $l$ and $r$ and repeat comparison. \\
  Repeat loop until $l = s/2$. If $l$ reaches $s/2$, Halt and succeed.

  \textit{Reasoning}:
  Reading from the tape has no charge. We use a fixed number of counters: 
  $s$, $l$, $r$ whose values are bounded by $\log(n)$.
  We can use a fixed number of work tapes for the comparison whose 
  values are bounded by $\log(n)$. (eg use 2 tapes of size 1). Overall 
  we use space $\leqslant m\log(n)$ where $m$ is independent of input size, so this 
  algorithm decides \textsc{Double} in logspace, hence $\textsc{Double} \in \el$. 
  \item % (d)
  \textit{Algorithm}:
  Guess some split $w_1 \cdot w_2 \cdot w_3 \cdot ... \cdot w_n$ where $w_i \in \{0,1\}^{*}$ \\
  Use one counter to store $n$. Use two counters $i$, $j$. \\
  $i$ points to beginning $w_1$ and $j$ points to beginning of $w_2$. \\
  Check $w_1$, $w_2$ in $L$ by running $M(w_1), M(w_2)$ respectively.\\
  If either $M(w_1), M(w_2)$ return no, halt and fail. \\
  Compare $w_1$ with $w_2$ and check they are equal. \\
  Now repeat, by incrementing $i$ and $j$ and checking $w_2$, $w_3$, instead. \\
  Repeat until $i$ reaches $n$ ($w_{n-1}$, $w_{n}$ checked).
  Halt and succeed.\\
  \textit{Reasoning}:
  To compare $w_{i}$ and $w_{j}$, we use can one tape with 1 symbol
  We have a fixed number of counters bounded by $\log(n)$ and 
  running $M(w_1), M(w_2)$ can be done in logspace. Hence this algorithm 
  describes a nondeterministic machine that decides $L^{*}$ in logspace. 
  Hence $L^{*} \in \nl$.
\end{enumerate}
\item % Question 3
\begin{enumerate}
\item % (a)
  \begin{enumerate}
  \item Use $n^3$ processors, and each calculates $A_{ik}B_{kj}$ in one step.
  Then use $n^2$ of processors to find sum $S_{ij} = \sum_{k} A_{ik}B_{kj}$. 
  When computing $S_{ij}$, each processor will parallelise addition 
  by using a further $n$ processors. 
  Create a knockout-tournament tree of additions of depth $O(\log(n))$.
  Overall time $=$ 1 step for mult. $+$ $O(\log(n))$ steps for additions $ = O(\log(n))$
  \item Take the adjacency matrix for $A$ and square $A$ successively $\ceil{\log(n)}$ times.
  This gives the transitive closure for $A$. By using the previous matrix mult. algorithm, 
  we can square a matrix in $O(\log(n))$ steps. Overall time is num of times we square $\times$ steps for squaring 
  $ = O(\log^2(n))$. 
  \end{enumerate}
  \item % (b)
  Create knockout tournament. Each match calculates total 0s and total 1s by performing 
  parallel addition of the result of two matches from the previous parallel round, $(x+x',y+y')$. 
  The final summation has some total $(x,y)$ and we return yes iff $x = y$.
  Total work: $2(n-1)$. Parallel time: $O(\log(n))$
  \item % (c) 
  \begin{enumerate}
  \item 
  $L$ is in $NC_j$ iff there is a logspace uniform family of circuits $\mathcal{C}$ that decide
  $L$ in parallel time $O(\log^j(n))$ with $O(n^k)$ work. 
  \item
  Because $L \in NC_j$:
  \begin{itemize}
  \item 
  from family $\mathcal{C}$, there is a circuit $C_1$ that decides if $x_1 \in L$,
  a circuit $C_2$ that decides if $x_1x_2 \in L$,
  a circuit $C_3$ that decides if $x_1x_2x_3 \in L$,
  $\ldots$
  a circuit $C_n$ that decides if $x_1x_2x_3 ... x_n \in L$.
  We perform $\wedge$ on all the outputs of $C_1$, $C_2$, ..., $C_n$ and the 
  result outputs \textsf{tt} iff for all $y \le x$, $y \in L$.
  In other words, $C'_n$ decides $L'$.   
  %
  \item $C_1$, $C_2$, ..., $C_n$ 
  have depths bounded by $O(\log^{j}(n))$.
  When we perform $\wedge$, we can connect them using a tournament-style knockout tree 
  that has a depth of $\lceil \log(n) \rceil$. 
  Overall, $C'_n$ has depth $O(\log^{j}(n)) + \lceil \log(n) \rceil = O(\log^{j}(n))$
  %
  \item $C_1$, $C_2$, ..., $C_n$  have sizes bounded by $O(n^k)$.
  We use $n-1$ $\wedge$-gates for our knockout tree. Overall, $C'_n$ has size:
  $nO(n^k) + n-1 = O(n^{k'})$ for some $k'$. \\
  \end{itemize}
  Since $L'$ is in $PT/WK(\log^{j} n, n^{k'})$, we have $L' \in NC_j$.
  \end{enumerate}
\end{enumerate}
\item % Question 4
\begin{enumerate}
\item % (a)
  \begin{enumerate}
  \item % (i)
  \textsf{RP} is the class of languages with p-time Monte Carlo TMs
  (if $x \in L$ then TM outputs yes with probability $> \frac{1}{2}$, 
  if $x \notin L$ then TM definitely outputs no).
  \item % (ii)
  Suppose that $L_1, L_2 \in \textsf{RP}$ and are recognised by Monte Carlo machines $M_1$, $M_2$ 
  respectively. We describe a machine $M$ for $L_1 \cap L_2$.  
  Run $M_1$ followed by $M_2$ and accept iff both $M_1$ and $M_1$ accepts.
  Clearly $M$ is can be represented as a precise NDTM with polybound $p_1 + p_2$.
  
  Let $\frac{1}{2} + \epsilon_1$, $\frac{1}{2} + \epsilon_2$ be probabilities 
  of success, when $x \in L_1$, $x \in L_2$, for $M_1$, $M_2$ respectively, $\epsilon_1, \epsilon_2 > 0$.
  Then probability of success when $x \in L$ is 
  $(\frac{1}{2} + \epsilon_1)2^{p_1} \times (\frac{1}{2} + \epsilon_2)2^{p_2}$
  $ = (\frac{1}{2} + \epsilon_1)(\frac{1}{2} + \epsilon_2)2^{p_1 + p_2}$ 
  $ = \epsilon 2^{p_1 + p_2}$, for some $\epsilon > 0$. 
  Suppose we needed to run $M$, $m$ times until a successful computation. Then
  $(1 - (1-\epsilon)^m) 2^{p_1 + p_2}$ of computations succeed. 
  Then let $M$ run $m$ times such that $(1 - (1-\epsilon)^m) > \frac{1}{2}$. 
  Then we have that if $x \in L$, then $M$ succeeds with prob. $> \frac{1}{2}$. 
  \end{enumerate}
\item % (b)
  \begin{enumerate}
  \item % (i)
  \textsc{Sat}: given a set of clauses, is there a satisfying assignment that 
  makes each clause evaluate to true\\
  \textsc{FSat}: given a set of clauses, find a satisfying assignment that 
  makes each clause evaluate to true
  \item % (ii)
  Let $\varphi(x_1,x_2,x_3,...,x_n)$ be our formula in CNF. \\
  We can solve \textsc{FSat} by following alg:\\
  \textit{Algorithm}:\\
  Ask oracle if $\varphi$ is sat. If no halt. If yes continue.\\
  Ask oracle if $\varphi(\textsf{tt},x_2,x_3,...,x_n)$ is sat.
  If yes, assign $h(x_1) = \textsf{tt}$. If no $h(x_1) = \textsf{ff}$.
  Ask oracle if $\varphi(h(x_1),\textsf{tt},x_3,...,x_n)$ is sat.
  If yes assign $h(x_2) = \textsf{tt}$. If no $h(x_2) = \textsf{ff}$.
  Repeat until all variables assigned.
  We only use oracle $n+1$ times (polynomial times) 
  so we have computed \textsc{FSat} in p-time.
  \item % (iii)
  \textsc{F3Col}: Given a graph $G$, find an assignment of colours to the nodes of $G$
  in such a way that no two adjacent nodes have the same colour, and no 
  more than three colours are used.

  Let $G(v_1,v_2,v_3,...,v_n)$ be our graph. \\
  Given a \textsc{F3Col} oracle, we can solve \textsc{F3Col} by following alg:\\
  Fix three colours \textsf{R}, \textsf{G}, \textsf{B}. 
  Ask oracle if $G(v_1,v_2,v_3,...,v_n)$ is 3-colourable. If no halt. If yes continue.\\

  Modify $G$ and add $e$ \textsf{RGB} $K_3$ cliques, where $e$ is the number of edges in $G$.

  Let $G(\textsf{R},v_2,v_3,...,v_n)$ denote modifying $G$ (with cliques) where,
  for each edge, $e$, that $v_1$ connects with, add two new edges to nodes \textsf{G} and \textsf{B} 
  in corresponding \textsf{RGB}-clique for $e$.
  Similarly, let $G(\textsf{R},\textsf{G},v_3,...,v_n)$ denote $G_1$ with
  but adding for each edge, $e$, that $v_2$ connects with
  add two new edges to nodes \textsf{R} and \textsf{B} 
  in corresponding \textsf{RGB}-clique for $e$. And so on, for $n$ colour assignments.

  In general let $G(...,\textsf{C}_i,...)$ where $\textsf{C}_i \in \{\textsf{R},\textsf{G},\textsf{B}\}$
  denote, taking the previous graph and modifying it 
  so that for each edge, $e$, that $v_i$ connects with, we add two new edges to nodes $\neq \textsf{C}_i$
  in corresponding \textsf{RGB}-clique for $e$.\\
  
  Let $\chi$ be our colouring. Set $\chi(v_1) = \textsf{R}$. \\ 
  Ask oracle if $G(\textsf{R},\textsf{R},v_3,...,v_n)$ is 3-colourable.\\
  If yes, then set $\chi(v_2) = \textsf{R}$.
  Else ask oracle if $G(\textsf{R},\textsf{G},v_3,...,v_n)$ is 3-colourable.\\
  If yes, then set $\chi(v_2) = \textsf{G}$, else $\chi(v_3) = \textsf{B}$. \\

  Ask oracle if $G(\textsf{R},\chi(v_2),\textsf{R},...,v_n)$ is 3-colourable.\\
  If yes, then set $\chi(v_3) = \textsf{R}$
  Else ask oracle if $G(\textsf{R},\chi(v_2),\textsf{G},...,v_n)$ is 3-colourable.\\
  If yes, then set $\chi(v_3) = \textsf{G}$, else $\chi(v_3) = \textsf{B}$. \\

  Ask oracle if $G(\textsf{R},\chi(v_2),\chi(v_3),...,\textsf{R})$ is 3-colourable.\\
  If yes, then set $\chi(v_n) = \textsf{R}$,
  Else, ask oracle if $G(\textsf{R},\chi(v_2),\chi(v_3),...,\textsf{G})$ is 3-colourable.\\
  If yes, then set $\chi(v_n) = \textsf{G}$, else $\chi(v_n) = \textsf{B}$.

  We ask the oracle $O(n)$ times and graph modifications can be done in p-time.
  Hence we have a p-time algorithm. 

  \end{enumerate}

\end{enumerate}




\end{enumerate}

\end{document}
