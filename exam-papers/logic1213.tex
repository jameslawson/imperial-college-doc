\documentclass[a4paper, draft, 12pt]{article}

% BEGIN PACKAGE IMPORT
% =============================================================================
\usepackage{examsty}
\usepackage{logicsty}
% =============================================================================

% BEGIN SOLUTIONS
% =============================================================================



\begin{document}
\textbf{Modal and Temporal Logic, 2012-2013}\\
\textit{James Lawson}

\begin{enumerate} 
\item % Question 1
\begin{enumerate}
\item
  \begin{enumerate}
  \item
  $A$ is \textit{valid in a model}, $\M$, if $A$ is satisfied at every world of $\M$.
  \item
  $A$ is \textit{valid in a frame}, $\F$, if for any model, $\M$, constructed from $\F$, $A$ is valid in $\M$.
  \item
  $A$ is \textit{valid} if $A$ is valid in all frames.
  \end{enumerate}
\item 
  \begin{enumerate}
  \item
  Reflexive frames.
  \item
  Let $A = \bx p \imp p$.
  In the $\M$ given, $A$ is valid because it is satisfied at all worlds. We can see that $1, \M \models A$ 
  (1 sees only 2 and $p$ is true at 2), and that $2, \M \models A$ (2 sees no worlds). 

  However, $A$ is not valid for the frame, $\F$. 
  We can find a witness model $\M_0$ created from $\F$ and see that $A$ is not valid in $\M_0$. 
  Take $\M_0$ as below (where $h(p) = \emptyset$). $1, \M \nvDash A$ so 
  $A$ is not valid in $\M_0$. Hence $A$ cannot be valid in $\F$.
  \end{enumerate}
\item 
  \begin{enumerate}
  \item
  Assume without proof $\M, t \sat A \imp B$ iff $\M, t \sat A \imp \M, t \sat B$. \\
  Take arbt. world $t$ from arbt. model $\M$ constructed from arbt. frame $\F$. \\
  To show: $\M, t \sat \bx(p \imp q) \imp (\dm p \imp \dm q)$.\\
  Assume $\M, t \sat \bx(p \imp q)$.
  By Kripke semantics, if $R(t,u)$ then $\M, u \sat p \imp q$ for 
  all $u$ in $\F$. Now assume $\M, t \sat \dm p$. Then $R(t,u_0)$ and $\M, u_0 \sat p$ for some world $u_0$.
  But we also have $\M, u_0 \sat p \imp q$ by first assumption. So $\M, u_0 \sat q$. 
  Hence $\M, t \sat \dm q$ So $\M, t \sat \dm p \imp \dm q$, and, $\M, t \sat \bx(p \imp q) \imp (\dm p \imp \dm q)$
  \item
  \textit{Not} valid. \\
  We can have $\M, t \sat \dm(p \imp q)$ but not $\M, t \sat \dm p \imp \dm q$. 
  Assume $\M, t \sat \dm(p \imp q)$, then assume $\M, t \sat \dm p$. 
  There is some world that $t$ can see that satisfies $p \imp q$ and some world 
  that $t$ can see that satisfies $p$. 
  However we cannot guarantee that these worlds as the \textit{same}. We cannot 
  apply modus ponens like before and say there is a a world that $t$ can see that satisfies $q$. 
  Counter-example:
  \end{enumerate}
\item 
  \begin{enumerate}
  \item
  $\neg \dm(\dm \bx p \wedge \neg p)$
  \item
  Let $A = \dm(\dm \bx p \wedge \neg p)$ \\
  We will apply Sahlqvist's algorithm to find the \textit{Sahlqvist's correspondant} of $A$, $\alpha[t]$ and 
  thus have $\F, u \sat \neg A$ iff $\F \sat \forall t \neg \alpha(t)$. 
  \textit{Boxed atoms}: $\bx p$.  \textit{Negative formulas}: $\neg p$. We want to find a lazy assignment that 
  makes boxed atoms true without concern for negative formulas. 
  Suppose arbt world $t$ sees some world $u$ and $u$ sees some world $v$, 
  then our assignment must make $p$ true at worlds $v$ can see in order to 
  make boxed atom $\bx p$ true. So \textit{lazy assignment} is $h^{o}(p) = \{x \;|\; R(v,x)\}$.

  Now we take the \textit{standard translation} of $A$:
  $$ 
  \begin{array}{ll}
    A^{t} &= \exists u(R(t, u) \wedge (\dm\bx p \wedge \neg p)^u  \\
          &= \exists u(R(t, u) \wedge ((\dm\bx p)^u \wedge (\neg p)^u) \\
          &= \exists u(R(t, u) \wedge ( (\exists v(R(u,v) \wedge (\bx p)^v) \wedge (\neg p^u)) \\
          &= \exists u(R(t, u) \wedge \exists v(R(u,v) \wedge (\bx p)^v \wedge \neg P(u)) \\
  \end{array}
  $$
  To preserve equivalence, we move $\exists v$ outside: \\
  $\exists u \exists v(R(t, u) \wedge R(u,v) \wedge (\bx p)^v \wedge \neg P(u))$. \\
  Our lazy assignment lets us replace $P(u)$ with $R(u,v)$ and $(\bx p)^v$ with $(\top)^v=\top$. 
  $\exists u \exists v(R(t, u) \wedge R(u,v) \wedge \top \wedge \neg R(u,v)) = \alpha[t]$. \\
  So $\forall t \neg \alpha(t) = \forall t \exists u \exists v(R(t, u) \wedge R(u,v) \wedge (\top)^v \wedge \neg R(u,v))$
  which is logically equivalent to $\forall t \forall u \forall v (R(t,u) \wedge R(u,v) \imp \neg R(u,v))$. 

  Since $(\F, h), t \sat A$ iff $\F \sat (\F, h^{o}), t \sat A$ iff $\F \sat \alpha[t]$. \\
  we have: $(\F, h), t \sat \neg A$ iff $\F \sat \forall t \neg \alpha[t]$ for any $\F, h, t$. 
  That is, $B$ is valid in $\F$ iff $\F$ satisfies (using first-order semantics) $\forall t \neg \alpha[t]$  \\
  \item 
  $B$ is valid in $\F$ iff $\forall t \forall u \forall v (R(t,u) \wedge R(u,v) \imp \neg R(u,v))$. 
  So we need to find a frame $\F_0$ that doesn't satisfy this first-order condition. The frame below 
  doesn't satisfy the condition because there are some worlds $x,y,z$ 
  where $x$ sees $y$, $y$ sees $z$ but we have $z$ sees $y$. 
  \end{enumerate}
\end{enumerate}

%
\item % Question 2
\begin{enumerate}
\item % (a)
  \begin{enumerate}
  \item % (i)
  Since $\F'$ is a p-morphic image of $\F$, there is a p-morphism from $\F$ to $\F'$. 
  Every p-morphism satisfies the forth property and back-property. \\
  \textit{Forth property}: $R(x,y) \imp R'(f(x),f(y))$ for all $x,y \in W$\\
  \textit{Back property}: $R'(f(x),v) \imp (R(x,y)$ and $f(y) = v$ for some $y \in W$)\\
  for all $x \in W, v \in W'$.

  Take arbt $x, y, z \in W'$. Assume $R'(x,y)$ and $R'(y,z)$ To show $R'(x,z)$.\\
  $x$ must the image of some world $t \in W$, $x = f(t)$. By the back-property, 
  given $R'(f(t),y)$, we must have $R(t, t')$ where $f(t') = y$ for some world $t' \in W$. 
  By the back-property, since $R'(f(t'),z)$ , we have $R(t', t'')$ 
  where $f(t'') = z$ for some world $t'' \in W$. 

  Since $\F$ is transitive, given that $R(t,t')$ and $R(t', t'')$, we must have $R(t,t'')$. 
  And by the forth property, given that $R(t,t'')$ we must have $R'(f(t),f(t''))$, in
  other words, $R(x,z)$.
  \item % (ii)
  No. Counter-example:
  \end{enumerate}
\item % (b)
  \begin{enumerate}
  \item % (i)
  Example:
  \item % (ii)
  Let $\F \times (\mathbb{N}, <) = (W^{\times}, R^{\times})$ 
  $(x,u)R^{\times}(y,v)$ iff $R(x,y)$ and $u < v$. \\
  Assume $\F$ is triangle-free. To show: $R^{\times}(x,y)$ is triangle free.\\ 
  Assume for contradiction, $R^{\times}(x,y)$, $R^{\times}(y,z)$, $R^{\times}(z,x)$ for some $x,y,z \in W^{\times}$.\\
  Then by def. of $R^{\times}$, we have $x < y$, we have $y < z$ and $z < x$. By transitivity of $<$, 
  we have $x < z$ but this contradicts $z < x$. \\
  (\textit{alternatively}: derive contradiction by seeing that there is a triangle in $\F$).
  So there are no $x,y,z \in W^{\times}$ such that $R^{\times}(x,y)$, $R^{\times}(y,z)$, $R^{\times}(z,x)$.
  \item % (iii)
  From lectures, there is a p-morphism from $\F \times (\mathbb{N}, <)$ to $\F$. \\
  So $\F$ is a p-morphic image of $(\mathbb{N}, <)$. From lectures, p-morphic images 
  preserve validity of modal formulas. That is, if $A$ valid in $\F \times (\mathbb{N}, <)$ then $A$ valid in $\F$. 
  \end{enumerate}
\item % (c)
  \begin{enumerate}
  \item % (i)
  Take some arbitrary time $t$, in $\M = ((\mathbb{N}, <), h)$ for some arbitrary $h$. \\
  Assume $\M, t \sat FGq$. Then there is some future time $f$, where $q$ is true for the all future times of $f$.

  Now take any time $t'$ in the future of $t$. If $t' < f$ then there is indeed a future time 
  where $q$ is true ($f$). So $t'$ must satisfy $Fq$. If $t' \geqslant f'$, $q$ is always true 
  in the future of $f'$, and so must always be true in the future of $t'$.
  So there must be a time when $q$ is true in the future $t'$. Hence $t'$ must satisfy $Fq$. \\
  Any future time of $t$, $t'$, satisfies $Fq$, so we have $GFq$.\\
  Hence $FGq \imp GFq$ is satisfied at all worlds $t$ of $\M$.

  \item % (ii)
  Not valid. We can find some model $\M_0$ constructed from $(\mathbb{N}, <)$ where 
  at some world, $t$, $GFq$ holds but $FGq$ doesn't. Let $\M_0$ have assignment
  $h(q) = \{\;t\in \mathbb{N} \;|\; t \text{ is even }\}$.\\
  $GFq$ is satisfied at $t$, as for all future times of $t$, $t'$, 
  there will be some future time of $t'$ where $q$ is true 
  (say, at the next even time larger than $t'$). 
  However $FGq$ isn't satisfied at $t$. We cannot find a 
  future time of $t$, $f$, where for all times in future of $f$, $q$ is true. 
  There will be some odd time in the future of $f$ where $q$ is false. 
  \end{enumerate}
\end{enumerate}
\item % Question 3
  \begin{enumerate}
  \item % (a)
  Lemmon filtration for $A$: 
  \begin{itemize}
  \item $W^{f} = $ set of $\sim$-equivalence classes, where $\sim$ is a relation on $W$
  where $t \sim u \Longleftrightarrow [\M, t \sat B \Leftrightarrow \M, u \sat B$, for all subformulas, $B$, of $A]$
  \item $R^{\ell}(X,Y) \Longleftrightarrow [\M,x \sat \bx B \Rightarrow \M, y \sat \bx B \wedge B$ for all $x \in X, y \in Y$,  subformulas, $\bx B$, of $A]$
  \item $h^f(p) = \{\; X \in W^{f} \;|\; \M, x \sat p \text{ for some } x \in X \;\}$
  \end{itemize}
  \item % (b)
  Lemmon filtration $\mathcal{N}^{\ell}_{\bx p}$: \\
  \begin{tikzpicture}
  \node(A)[draw, circle, label={$p, \bx p$}]{$A$};
  \node(B)[below left=1cm and 0.8cm of A][draw, circle]{$B$};
  \node(C)[below right=1cm and 0.8cm of A, label={$\bx p$}][draw, circle]{$C$};
  \draw[->] (B) -- (A); \draw[->] (B) -- (C); \draw[->] (C) -- (A); 
  \draw[->] (B) edge  [in=120,out=60,loop, looseness=7] (B);
  \draw[->] (A) edge  [in=280,out=250,loop, looseness=6] (A);
  \end{tikzpicture}
  \item % (c)
  Take any arbitrary $X \in \mathcal{F}^{\ell}_{A}$. \\
  Take any arbitrary world $x \in X$. Let $B$ be any subformula of $A$.\\
  Assume $\M, x \sat \bx B$.

  $\F$ is serial, so there 
  is some world $y \in W$ with $R(x,y)$. So by our assumption, $\M, y \sat B$.
  Take some arbt $z \in W$. Assume $R(y,z)$. By transitivity of $\F$, we have $R(x,z)$. 
  And so by our first assumption, $\M, z \sat B$.
  Hence $\forall z(R(y,z) \imp \M, z \sat B)$, so $\M, y \sat \bx B$. 
  Hence $\M, y \sat \bx B \wedge B$

  Let $Y\in \mathcal{F}^{\ell}_{A}$ be the $\sim$-equivalence class that $y$ is in.
  Take any $y'$ in $Y$. Since $y$ and $y'$ agree on the satisfiability of all subformulas of $A$, 
  they agree on the satisfisfiability of $B \wedge \bx B$.
  So must have for all $y' \in Y$, $\M, y' \sat B \wedge \bx B$. 
  So $\M, x \sat \bx B$ implies [$\M, y' \sat B \wedge \bx B$ for all $x \in X$, $y' \in Y$, all subformulas $B$.]
  Hence $R^{\ell}(X,Y)$. 

  So for any $X\in \mathcal{F}^{\ell}_{A}$, there is some $Y\in \mathcal{F}^{\ell}_{A}$ 
  such that $R^{\ell}(X,Y)$.
  \item % (d)
  $KTS = K + (\dm \top, \bx p \imp \bx \bx p)$ is sound and complete for $\mathcal{TS}$.
  We claim $\text{Thm}(KTS)$ has the strong finite model property. 

  Take some $A \notin \text{Thm}(KTS)$,
  Since $KTS$ is sound and complete over $\mathcal{TS}$, 
  $A \notin \text{Log}(\mathcal{TS})$, in other words, there is some 
  model $\M$ constructed from a frame in $\mathcal{TS}$ with $\M,t \sat \neg A$.
  We can take this $\M$ and find the Lemmon filtration, $\M^{\ell}_{A}$.
  By the Filtration lemma, $\neg A$ is satisfied in $\M^{\ell}_{A}$

  From part (c), $\mathcal{F}^{\ell}_{A}$ is serial. From lectures,
  The Lemmon filtration relation $R^\ell$ is transitive (Lemma 9.19), so
  $\mathcal{F}^{\ell}_{A}$ is transitive. Hence $\M^{\ell}_{A}$ is 
  serial and transitive and so validates $\text{Thm}(KTS)$.

  Since $\mathcal{F}^{\ell}_{A}$ is finite with at most $2^n$ worlds (where $n$ is the 
  number of subformulas of $\neg A$), $KTS$ has the strong finite model property.
  There is an algorithm to decide whether $\mathcal{F}^{\ell}_{A}$ validates 
  $KTS$ ($KTS$ has finite many axioms - by Theorem 8.3, it is enough to just check each axiom true for all finite worlds
  taken from finitely possible models build from $\mathcal{F}^{\ell}_{A}$) - so by Theorem 9.8, $KTS$ is decidable. 

  \textit{Algorithm:}\\
  Compute $s(A) = 2^{n}$, where $n$ is the number of subformulas of $A$. \\
  Enumerate all models whose frame is $\mathcal{F}^{\ell}_{A}$. \\
  If we find a model satisfies $\neg A$, halt and print $A \notin L$.\\
  If no models are found that satisfy $\neg A$, halt and print $A \in L$
  \end{enumerate}
\item % Question 4
  \begin{enumerate}
  \item
  \begin{enumerate}
  \item  
  For any $\Gamma, \Delta \in W_C$, \\
  $R_C(\Gamma, \Delta)$ iff [if $\bx A \in \Gamma$ then $A \in \Delta$, for all formulas, $A$].  
  \item 
  ($\Rightarrow$) Assume $R_C(\Gamma, \Delta)$. \\
  Take any $A$. Assume $A \in \Delta$.
  By the truth lemma, $\M_C, \Delta \sat A$.
  By Kripke semantics, $\M_C, \Gamma \sat \dm A$.
  By the truth lemma, $\dm A \in \Gamma$. \\
  ($\Leftarrow$) Assume $A \in \Delta$ implies $\dm A \in \Gamma$ for any $A$. \\
  Assume $\bx A \in \Gamma$. \\
  Suppose for contradiction, $A \notin \Delta$. \\
  $\Delta$ is a MCS, so by MCS properties (lemma 7.20), $\neg A \in \Delta$. 
  So by our first assumption, $\dm \neg A \in \Gamma$.
  But $\dm B$ is an abbreviation for $\neg \bx \neg B$, 
  so we have $\neg \bx \neg \neg A \in \Gamma$. 
  By the truth lemma, $\M_C, \Gamma \sat \neg \bx \neg \neg A$. 
  By equivalence in Kriple semantics $\M_C, \Gamma \sat \neg \bx A$. 
  And by the truth lemma we have $\neg \bx A \in \Gamma$ .
  $\Gamma$ is a MCS, so by properties of MCSs, $\bx A \notin \Gamma$. Contradiction. \\
  So $A \in \Delta$. Hence $R(\Gamma,\Delta)$.
  \item
  Take any worlds $\Gamma, \Delta_1, \Delta_2 \in W_C$. \\
  Assume $R_C(\Gamma, \Delta_1)$ and $R_C(\Gamma, \Delta_2)$.\\
  To show $R_C(\Delta_1, \Delta_2)$. \\
  Assume $\bx A \in \Delta_1$ \\
  From part ii, since $R_C(\Gamma, \Delta_1)$, we have $\dm \bx A \in \Gamma$. 
  Since $\Gamma$ is an MCS, by Lemma 7.19, $\Gamma \vdash_C \dm \bx A$.
  Now, $\dm \bx p \imp \bx p$ is an axiom of $C$, so by sub, 
  $\vdash_C \dm \bx A \imp \bx A$. 
  Since $\Gamma \vdash_C \dm \bx A$ and  $\vdash_C \dm \bx A \imp \bx A$, 
  we have $\Gamma \vdash_C \bx A$, which by MCS properties, gives $\bx A \in \Gamma$. 
  But since $R(\Gamma, \Delta_2)$, $A \in \Delta_2$. \\
  Hence $R_C(\Delta_1, \Delta_2)$.
  \item
  The \textit{Completeness Theorem} states that the 
  if a class of frames $\C$ contains the canonical frame $\F_H$,
  then $H$ is complete over $\C$. 

  In part iii, we showed 
  that the class of balloon-like frames contains $\F_C$, 
  and so our Hilbert system is complete over the class.
  \end{enumerate}
  \item 
  \begin{enumerate}
  \item  
  $f$ is monotonic iff for all sets $U,V \subseteq W$:\\
  if $U \subseteq V$ then $f(U) \subseteq f(V)$. 
  \item
  $U \subseteq W$ is a fixed point of $f$ iff $f(U) = U$.
  \item 
  $f^0(\emptyset) = \emptyset$ by def of $f$.
  But $\emptyset \subseteq f^1(\emptyset)$, as the emptyset is a subset of all sets.
  Since $f$ is monotonic, we have $f^1(\emptyset) \subseteq f^2(\emptyset)$. 
  By monotonicity of $f$ we have $f^2(\emptyset) \subseteq f^3(\emptyset)$
  and $f^3(\emptyset) \subseteq f^4(\emptyset)$, ...

  As $W$ is finite, there must be some $m$ such that $f^m(\emptyset) = f^{m+1}(\emptyset)$. 
  Hence $Z = f^m(\emptyset)$ is a fixed point of $f$. 
  Since $f^0(\emptyset) \cup f^1(\emptyset) \cup ... \cup f^m(\emptyset) = Z$ and 
  $f^m(\emptyset) \cup f^{m+1}(\emptyset) \cup ... = Z$, 
  we have $Z = Z \cup Z = \cup_{n \in \mathbb{N}} f^{n}(\emptyset)$.
  \end{enumerate}
  \item
  \begin{enumerate}
  \item  Let $A = p \wedge \bx q$. 
  Then $\bb{\mu q A}_h = \text{LFP}(A^h_q)$ where $A^h_q(U) = \bb{A}_{h[p \mapsto U]}$.
  \begin{itemize}
  \item 
  Let $h_0 = h[q \mapsto \emptyset]$. Then $A^h_q(\emptyset)$ \\
  $ =\bb{p \wedge \bx q}_{h_0}$ 
  $ = \bb{p}_{h_0} \cap \bb{\bx q}_{h_0}$
  $ = \{2,3,4,5\} \cap \bx \bb{q}_{h_0}$ \\
  $ = \{2,3,4,5\} \cap \bx \emptyset = \{2,3,4,5\} \cap \{5\} = \{5\}$
  \item 
  Let $h_1 = h[q \mapsto \{5\}]$. Then $A^h_q(A^h_q(\emptyset))$ \\
  $ = \bb{p \wedge \bx q}_{h_1}$ 
  $ = \bb{p}_{h_1} \cap \bb{\bx q}_{h_1}$
  $ = \{2,3,4,5\} \cap \bx \bb{q}_{h_1}$ \\
  $ = \{2,3,4,5\} \cap \bx \{5\}$
  $ = \{2,3,4,5\} \cap \{4,5\} = \{4,5\}$.
  \item 
  Let $h_2 = h[q \mapsto \{4,5\}]$. Then $A^h_q(A^h_q(A^h_q(\emptyset)))$ \\
  $ = \bb{p \wedge \bx q}_{h_2}$ 
  $ = \bb{p}_{h_2} \cap \bb{\bx q}_{h_2}$
  $ = \{2,3,4,5\} \cap \bx \bb{q}_{h_2}$ \\
  $ = \{2,3,4,5\} \cap \bx \{4,5\}$
  $ = \{2,3,4,5\} \cap \{4,5\} = \{4,5\}$.
  \end{itemize}
  So $\text{LFP}(A^h_q) = \cup_{n \in \mathbb{N}} (A^h_q)^n(\emptyset) = \{4,5\}$.
  So $\bb{\mu q A}_h = \{4,5\}$.

  \item 
  $\bb{\nu q A}_h = \text{GFP}(A^h_q)$. Let $W = \{1,2,3,4,5\}$.
  \begin{itemize}
  \item 
  Let $h_0 = h[q \mapsto W]$. \\
  Then:
  $A^h_q(W) = \bb{p \wedge \bx q}_{h_0}$ 
  $ = \bb{p}_{h_0} \cap \bb{\bx q}_{h_0}$
  $ = \{2,3,4,5\} \cap \bx \bb{q}_{h_0}$
  $ = \{2,3,4,5\} \cap \bx W$
  $ = \{2,3,4,5\} \cap W = \{2,3,4,5\}$.
  \item 
  Let $h_1 = h[q \mapsto \{2,3,4,5\}]$. \\
  Then:
  $A^h_q(W) = \bb{p \wedge \bx q}_{h_1}$ 
  $ = \bb{p}_{h_0} \cap \bb{\bx q}_{h_1}$
  $ = \{2,3,4,5\} \cap \bx \bb{q}_{h_1}$
  $ = \{2,3,4,5\} \cap \bx \{2,3,4,5\}$
  $ = \{2,3,4,5\} \cap \{3,4,5\} = \{3,4,5\}$.
  \item 
  Let $h_2 = h[q \mapsto \{3,4,5\}]$. \\
  Then:
  $A^h_q(W) = \bb{p \wedge \bx q}_{h_2}$ 
  $ = \bb{p}_{h_0} \cap \bb{\bx q}_{h_2}$
  $ = \{2,3,4,5\} \cap \bx \bb{q}_{h_2}$
  $ = \{2,3,4,5\} \cap \bx \{3,4,5\}$
  $ = \{2,3,4,5\} \cap \{3,4,5\} = \{3,4,5\}$.

  \end{itemize}
  So $\text{LFP}(A^h_q) = \cap_{n \in \mathbb{N}} (A^h_q)^n(W) = \{3,4,5\}$.
  So $\bb{\nu q A}_h = \{3,4,5\}$.

  \end{enumerate}
 
  

  \end{enumerate}



\end{enumerate}

\end{document}
