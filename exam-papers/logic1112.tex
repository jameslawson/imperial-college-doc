\documentclass[a4paper, draft, 12pt]{article}

% BEGIN PACKAGE IMPORT
% =============================================================================
\usepackage{examsty}
\usepackage{logicsty}
% =============================================================================

% BEGIN SOLUTIONS
% =============================================================================



\begin{document}
\textbf{Modal and Temporal Logic, 2011-2012}\\
\textit{James Lawson}

\begin{enumerate} 
\item % Question 1
\begin{enumerate}
\item % (a)
  \begin{enumerate}
  \item % (i)
  $\{1, 2, 3\}$.
  1 because 1 sees 2, and $\neg p$ is true at 2.
  2 because 2 sees 4, and $\neg p$ is true at 4.
  3 because 3 sees 4, and $\neg p$ is true at 4.
  Not 4 (it has no accessible worlds).
  \item % (ii)
  $\{ 2, 3 \}$.
  $\bx p$ is true at world 4 only (all other worlds can see some world where $p$ is false.) 
  So only the worlds that can access 4 satisfy $\dm \bx p$. These are worlds 2, 3. 
  \item % (iii)
  $\{2, 4\}$.
  Look for worlds where either $\neg \bx \bx p$ is true, or $\bx p$ is true, or both.\\
  $\neg \bx \bx p$ is true at $t$ iff there is some $u$ s.t. $R(t,u)$ where 
  $\bx p$ is false at $u$. 

  Not 1 - no $u$ as above and $\bx p$ is false at 1.
  2, because $\neg \bx \bx p$ is true - (2 can access 4, and $\bx p$ is false at 4).
  Not 3 - no $u$ as above and $\bx p$ is false at 3.
  4, because $\bx p$ is true.
  \end{enumerate}
\item 
  \begin{enumerate}
  \item May be true.
  \item May be true.
  \item Definitely true.
  \item Definitely false.
  \end{enumerate}
\item 
  \begin{enumerate}
  \item
  $$ 
  \begin{array}{lll}
    p \vee \bx (\bx p \imp \neg p) &\equiv \bx (\bx p \imp \neg p) \vee p  & A \vee B \equiv B \vee A  \\
          &\equiv \neg \dm \neg (\bx p \imp \neg p) \vee p & \neg \dm \neg A \equiv A \\
          &\equiv \neg \dm \neg (\neg [\bx p \wedge \neg\neg p]) \vee p & A \imp B \equiv \neg(A \wedge \neg B) \\
          & \equiv \neg \dm [\bx p \wedge p] \vee p & \neg\neg A \equiv A \\
          &\equiv \dm [\bx p \wedge p] \imp p & A \imp B \equiv \neg A \vee B \\
  \end{array}
  $$

  \item
  $\neg (\dm [\bx p \wedge p] \wedge \neg p)$
  \item
  Let $A = \dm [\bx p \wedge p] \wedge \neg p$ \\
  We will apply Sahlqvist's algorithm to find the \textit{Sahlqvist's correspondant} of $A$, $\alpha[t]$ and 
  thus have $\F, u \sat \neg A$ iff $\F \sat \forall t \neg \alpha(t)$. 
  \textit{Boxed atoms}: $\bx p$.  \textit{Negative formulas}: $\neg p$. We want to find a lazy assignment that 
  makes boxed atoms true without concern for negative formulas. 
  Suppose arbt world $t$ sees some world $u$. We need to make 
  $p$ true at all worlds, $v$ that $u$ can see (should it see any). 
  So \textit{lazy assignment} is $h^{o}(p) = \{x \;|\; R(u,x)\}$.

  Now we take the \textit{standard translation} of $A$:
  $$ 
  \begin{array}{ll}
    A^{t} &= (\dm [\bx p \wedge p])^t \wedge (\neg p)^t  \\
          &= \exists u (R(t,u) \wedge [\bx p \wedge p]^u) \wedge \neg p^t  \\
          &= \exists u (R(t,u) \wedge [(\bx p)^u \wedge p^u]) \wedge \neg p^t  \\
          &= \exists u (R(t,u) \wedge [(\bx p)^u \wedge P(u)]) \wedge \neg P(t)  \\
  \end{array}
  $$
  Our lazy assignment allows us to replace $(\bx p)^u$ with $\top^u = \top$, 
  replace $P(t)$ with $R(u, t)$, and $P(u)$ with $R(u, u)$. This gives:\\
  $\exists u (R(t,u) \wedge [\top \wedge R(u, u)]) \wedge \neg R(u, t)$. \\
  We move $\exists u$ to preserve equivalence:\\
  $\exists u (R(t,u) \wedge [\top \wedge R(u, u)] \wedge \neg R(u, t)) = \alpha[t]$\\
  So $\forall t \neg \alpha(t) = \forall t \neg \exists u (R(t,u) \wedge [\top \wedge R(u, u)] \wedge \neg R(u, t))$
  which is equivalent to $\forall t \forall u ([R(t,u) \wedge R(u,u)] \imp R(u,t))$. 

  Since $(\F, h), t \sat A$ iff $\F \sat (\F, h^{o}), t \sat A$ iff $\F \sat \alpha[t]$. \\
  we have: $(\F, h), t \sat \neg A$ iff $\F \sat \forall t \neg \alpha[t]$ for any $\F, h, t$. 
  That is, $B$ is valid in $\F$ iff $\F$ satisfies (using first-order semantics) the sentence $\forall t \neg \alpha[t]$.

  \end{enumerate}

\end{enumerate}
\item % Question 2
\begin{enumerate}
\item % (a)
  \begin{enumerate}
  \item % (i)
  $\F \times \N = (W^{\times}, R^{\times})$ where $W^{\times} = \{ (x,y) \;|\; x \in W, y \in \mathbb{N} \}$, 
  and $R^{\times}$ is a relation on $W^{\times}$ where $(x,u)R^{\times}(y,v)$ iff $xRy$ and $u < v$.  
  \item % (ii)
  Take any world $(x,u) \in W^{\times}$. Assume $(x,u)R^{\times}(x,u)$. Then by def of $R^{\times}$, $u < u$. Contradiction ($(\mathbb{N}, <)$ is irreflexive). Hence for all worlds $w \in W^{\times}$, $wR^{\times}w$ is false.
  \item % (iii)
  Take any worlds $(x,u), (y,v), (z,w) \in W^{\times}$. \\
  Assume $(x,u)R^{\times}(y,v)$ and $(y,v)R^{\times}(z,w)$. 
  Then by def of $R^{\times}$, we have $R(x,y)$, $R(y,z)$, but $\F$ is transitive so we have $R(x,z)$. 
  Also, by def of $R^{\times}$, we have $u < v$ and $v < w$. So $u < w$ ($(\mathbb{N}, <)$ is transitive). 
  Since $R(x,z)$, and $u < w$, we have $(x,u)R^{\times}(z,w)$. 
  \end{enumerate}
\item % (b)
  Let $A$ be any modal formula valid in flow of times. 
  From (a)(ii) and (a)(iii), $\F \times \N$ is a flow of time so $A$ is valid in it. 
  From lectures, there is a p-morphism from $\F \times \N$ to $\F$. So 
  $\F$ is a p-morphic image of $\F \times \N$.
  From lectures, p-morphisms preserve validity forwards, that is, if $\F'$ is a 
  p-morphic image of $\F$, then any modal formula valid in $\F$ is valid in $\F'$. 
  So $A$ is valid in $\F$. 
\item % (c)
  \begin{enumerate}
  \item $\bot \Un q \imp \top \Un q$
  \item $q \Si \top \imp q \Un \top $
  \end{enumerate}
\item % (d)
  Assume $(A \Un [A \wedge (A \Un B)])$ holds at $t$ for some arbitrary model $\M = (\N,h)$. \\
  By semantics of Until, there is some time, $f$, in the future of $t$
  where $\M, f \sat A \wedge (A \Un B)$
  and for all $f'$ such that $t< f' < f$, we have $\M, f' \sat A$.

  Because $\M, f \sat A \wedge (A \Un B)$, there is some future time of $f$, $l$,
  $\M, l \sat B$ and for all $l'$ such that $f< l' < l$, we have $\M, l' \sat A$.

  So we must have $\M, t \sat A \Un B$. There is some time in the future of $t$, 
  namely $l$, where $B$ is true. And for all times $t'$ with $t < t' < l$, 
  we have $\M, t' \sat A$ - given: $t < t' < f$ satisfies $A$, $t' = f$ satisfies A and $f < t' < l$ satisfies A.
\end{enumerate}
\item % Question 3
\begin{enumerate}
\item % (a)
  \begin{enumerate}
  \item % (i)
  The \textit{logic} of a class $\C$ is the set of all modal formulas that are valid in all the frames in $\C$
  \item % (ii)
  $L$ is \textit{sound} if every modal formula proven by $L$ is valid in all frames of $\C$. \\
  $L$ is \textit{complete} if it can prove every modal formula in the logic of $\C$. \\
  \end{enumerate}
  \item % (b)
  If $C_1 \subseteq C_2$ then $\text{Log}(C_2) \subseteq \text{Log}(C_1)$. \\
  Proof. Take some formula $A$, in $\text{Log}(C_2)$. To show $A$ in $\text{Log}(C_1)$. \\
  If $A \in \text{Log}(C_2)$ then $A$ is valid in all the frames of $C_2$. 
  All of the frames in $C_1$ are in $C_2$, so $A$ is valid in all the frames of $C_1$. 
  Hence $A \in \text{Log}(C_1)$. 
  \item % (c)
  $\bx p \imp p$ is valid in reflexive frames. In lectures we proved this several times 
  e.g. using direct proof, using Sahqvist algorithm to get correspondant $\forall t R(t,t)$. 
  However $\bx p \imp p$ is not valid in serial frames. To see this,
  find some model $\M$ constructed from some serial frame $\F$ and verify that $\bx p \imp p$ is not valid in $\M$. 
  Take serial frame $\F = (\mathbb{N}, <)$ and form model $\M$ with $h(p) = \{w \in \mathbb{N} | n > 1 \}$. 
  Then at world 1, we have $\bx p$, but not $p$, 
  hence $\M, 1 \notsat \bx p \imp p$. So $\bx p \imp p$ not valid over serial frames.
  \item % (d)
  \begin{enumerate}
  \item % (i)
  Assume without proof: $\M, w \sat A \imp B$ iff [if $\M, w \sat A$ then $\M, w \sat B$].\\
  Assume $\M, w \sat \bx p \wedge \dm q$. To show $\M, w \sat \dm (p \wedge q)$. \\
  We have $\M, w \sat \dm q$, so $w$ has some accessible world, $u$, with $\M, u \sat q$. 
  But since $\M, w \sat \bx p$, we must have $\M, u \sat p$. So $w$ has 
  some accessible world that satisfies $p$ and $q$. Hence $\M, w \sat \dm (p \wedge q)$.
  \item % (ii)
  1. $\vdash_K (p \wedge \neg [p \wedge q]) \imp \neg q \qquad$ taut. \\
  2. $\vdash_K \bx [(p \wedge \neg [p \wedge q]) \imp \neg q] \qquad$ UG(1). \\
  3. $\vdash_K \bx (A \imp B) \imp (\bx A \imp \bx B) \qquad$ (instance of normality). \\
  4. $\vdash_K \bx ((p \wedge \neg [p \wedge q]) \imp \neg q) 
  \imp (\bx (p \wedge \neg [p \wedge q]) \imp \bx \neg q) \qquad$ SUB(3). \\
  5. $\vdash_K \bx (p \wedge \neg [p \wedge q]) \imp \bx \neg q \qquad$ MP(2,4). \\
  6. $\vdash_K  (\bx p \wedge \bx \neg [p \wedge q]) \imp \bx \neg q \qquad$ RofE(5). 
  $\bx (A \wedge B) \equiv \bx A \wedge \bx B$ \\
  7. $\vdash_K  (\bx p \wedge \neg \bx \neg q) \imp (\neg \bx \neg (p \wedge q))  \qquad$ RofE(6). $6 \equiv 7$\\
  8. $\vdash_K  (\bx p \wedge \dm q) \imp \dm (p \wedge q))  \qquad$ RofE(7). $\dm A \equiv \neg \bx \neg A$
  \end{enumerate}
\end{enumerate}
\item % Question 4
\begin{enumerate}
\item % (a)
  \begin{enumerate}
  \item % (i)
  $\Phi$ is \textit{maximally consistent} with respect to $L$ iff 
  \begin{itemize}
  \item it is $L$-\textit{consistent}, there are no formulas $A_1, ..., A_n \in \Phi$, 
  $n \geqslant 0$, such that $\vdash_{L} (A_1 \wedge ... \wedge A_n) \imp \bot$.
  \item it is \textit{maximal}, there is no larger set that is $L$-consistent.
  \end{itemize}
  \item % (ii)
  First we will show $\Phi$ is consistent. \\
  Suppose it was not.
  Then there are formulas $A_1, ..., A_n \in \Phi$ 
  such that $\vdash_{L} (A_1 \wedge ... \wedge A_n) \imp \bot$. 
  Since $K4$ is sound, $(A_1 \wedge ... \wedge A_n) \imp \bot$ is valid 
  in the class of transitive frames. Since $\M$ 
  was based on a transitive frame,we must have $\M, w \sat (A_1 \wedge ... \wedge A_n) \imp \bot$.

  But since $A_1, ..., A_n \in \Phi$, we must have:
  $\M, w \sat A_1, ..., \M, w \sat A_n$, by def. of $\Phi$. 
  So $\M, w \sat A_1 \wedge ... \wedge A_n$. So $\M, w \sat \bot$ which gives a contradiction.
  Hence $\Phi$ is consistent. 

  Now to show it is maximally consistent. Suppose some formula $A$ is not in $\Phi$ 
  and $\Phi' = \Phi \cup\{A\}$ gives a consistent set. Since $A$ is not in $\Phi$, 
  we must have $\M, w \sat \neg A$, by def of $\Phi$. So $\neg A$ is in $\Phi$ 
  and so is also in $\Phi'$.

  But, having $A, \neg A \in \Phi'$ along with the proof: \\
  (1). $(p \wedge \neg p) \imp \bot \qquad$ (taut) \\
  (2). $(A \wedge \neg A) \imp \bot \qquad$ (sub) \\
  shows that $\Phi'$ is inconsistent. Contradiction. \\
  Hence $\Phi$ is maximally consistent.
  \end{enumerate}
\item % (b)
  \begin{enumerate}
  \item % (i)
  Assume $R'(X,Y)$. To show $R'(Y,X)$. \\
  By def. of $R'(X,Y)$, there is some $x \in X, y \in Y$ 
  such that $R(x,y)$. But since $R$ is symmetric, 
  we also have $R(y,x)$. 
  Since we have some $y \in Y, x \in X$ such that $R(y,x)$, by def. of $R'$, 
  $R'(Y,X)$. Hence $R'$ is symmetric.
  \item % (ii)
  Let $\M = (\{1,2,3,4\}, \{(2,1), (4,3)\}, h)$ where $h(p) = \{1\}$. Let $A = \bx p$. 
  Here $R$ is transitive, but $R'$ resulting from the filtration of $\M$ wrt $A$ is \textit{not}. 
  The filtration is $(W',R', h') = (\{X,Y,Z\}, (Z,Y), (Y,X), h')$ where $h'(p) = \{X\}$. Here 
  $R'$ is not transitive because there are relations $R'(Z,Y)$ and $R'(Y,X)$ but not $R'(Z,X)$. 
  \end{enumerate}
  \item % (c)
  \begin{enumerate}
  \item % (i)
  $L$ has the \textit{finite model property} iff 
  for any formula $A$ where $A \notin L$
  there is a finite model, $\M$ 
  such that for all $B \in L$, $B$ is valid in $\M$, 
  and for some world $t$ of $\M$, we have $M,t \sat \neg A$.

  $L$ is \textit{decidable} iff 
  there is some algorithm/program that given any modal formula $A$ as an input, 
  outputs \textit{yes} when $A \in L$ and outputs \textit{no} when $A \notin L$. 
  \item % (ii)
  Assume we have an algorithm to check a model validates a logic. \\
  Then if modal logic has the finite model property, then it is decidable. 
  We run two algorithms in parallel. The first enumerates all possible 
  theorems, the second enumerates all possible finite models. 
  
  For each finite model enumerated, check it validates $L$.
  and see whether $\M$ satisfies $\neg A$. If so, then it is a non-theorem. 
  We halt and print no. If $L$ has the finite model property, 
  then every non-theorem will eventually be printed. Otherwise $A$ 
  is a theorem, it will never be found by the first algorithm,
  and will eventually be printed by the first algorithm - at this point 
  we halt and print yes.
  \end{enumerate} 


\end{enumerate}
\end{enumerate}

\end{document}
